\documentclass{article}

\usepackage{times}
\usepackage{mathpartir}
\usepackage{amsmath,amsfonts,amssymb}
\usepackage{xcolor}
\usepackage{ifmtarg}

\newtheorem{lemma}{Lemma}

%%%%%% Macros here %%%%%%

\input brazilmacros


%% Syntax

\newcommand{\C}{C}     % computation
\newcommand{\K}{K}     % continuation variable
\newcommand{\X}{X}     % TT (computation) variable
\newcommand{\rgn}{r}   % region
\newcommand{\h}{h}     % handler
\newcommand{\B}{B}     % Brazil term
\newcommand{\R}{R}     % Result
\newcommand{\KK}{{\cal K}} % continuation term with hole


\newcommand{\val}{\mathsf{val}\,} % val e
\newcommand{\letin}[1]{\mathsf{let}\; #1 \;\mathsf{in}\;} % let x = c1 in c2
\newcommand{\op}[3]{\mathsf{op}_{#1}(#2, #3)} % operation
\newcommand{\inhabitPat}[2]{\mathsf{inhabit}(#1, #2)} % the inhabit pattern
\newcommand{\inhabitOp}[1]{\mathsf{inhabit}\,#1} % the inhabit operation
\newcommand{\inhabitRes}[3]{\mathsf{inhabit}(#1, #2, #3)} % the inhabit result
\newcommand{\opPat}[3][i]{\mathsf{op}_{#1}(#2, #3)} % the operation pattern
\newcommand{\opOp}[2][i]{\mathsf{op}_{#1}\,#2} % the operation operation
\newcommand{\opRes}[4][i]{\mathsf{op}_{#1}(#2, #3, #4)} % the operation result
\newcommand{\withhandle}[1]{\mathsf{with}\;#1\;\mathsf{handle}\;} % handle
\newcommand{\abs}[1]{\mathsf{abs}\;#1\;\mathsf{in}\;} % abstraction
\newcommand{\new}[2]{\mathsf{new}(#1,#2)} % new(r,T)
\newcommand{\fun}[1]{\mathsf{fun}\;#1\Rightarrow} % tt-level function
\newcommand{\ttapp}[2]{#1\,#2} % application
\newcommand{\ttlam}[2]{\lambda #1 \,{:}\, #2 \,.\,} % Kind of Brazilian lambda
\newcommand{\kapp}[2]{#1{}[#2]} % continuation application
\newcommand{\bterm}[2]{\langle\,#1\, :\, #2\,\rangle} % brazil (typed) term as data
\newcommand{\bty}[1]{\langle\,#1\,\rangle} % brazil type as data

\newcommand{\handler}[6][n]{\mathsf{handler}\; \val #2 \mapsto #3 \mid (\opPat{#4}{#5} \mapsto #6_{#1})_{i=1}^{#1}}

\newcommand{\makeApp}[2]{\mathsf{app}(#1,#2)} % introduce a Brazil application

\newcommand{\debruijn}[1]{\mathsf{debruijn}\,#1} % ugly hack


\newcommand{\cont}[2][\G,\D]{\mathsf{cont}(#1,#2)}     % continuation
\newcommand{\hfcont}{\hat{\cal K}}     % handler-free continuation
\newcommand{\hole}{\diamond}
\newcommand{\tuple}[1]{(#1)}
\newcommand{\generictuple}[1][n]{\tuple{e_1,\ldots,\e_{#1}}}

\newcommand{\pp}{\mathop{\text{\footnotesize{+}{+}}}}
\newcommand{\tupleappend}[2]{#1 \pp #2}


\newcommand{\pat}{P}
\newcommand{\match}[2]{\mathsf{match}\; #1\; \mathsf{with}\;#2}
\newcommand{\genericPats}[1][n]{(\pat_i \Rightarrow \C_i)_{i=1}^n}
\newcommand{\genericmatch}[1][\e]{\match{#1}{\genericPats}}

\renewcommand{\c}{c} % constant
\newcommand{\prim}[2][f]{#1(#2)} % primitive  application
\newcommand{\genericprim}{\prim{\e_1,\ldots,\e_n}}

\newcommand{\inj}[2][i]{\mathsf{inj}_{#1}\, #2}

%%% Operational semantics

\newcommand{\xcT}{\vec{\x}{:}\vec{\T}}
\newcommand{\xT}{{\vec{\x}}{\vec{\T}}}


\newcommand{\evalto}[3][\G]{#1 \vdash #2 \ \Downarrow\  #3}

\newcommand{\resultok}[2][\G]{#1 \vdash #2 \ \mathsf{ok}}
\newcommand{\eok}[2][\G]{#1 \vdash #2 \ \ \mathsf{ok}}
\newcommand{\cok}[2][\G]{#1 \vdash #2 \ \ \mathsf{ok}}
\newcommand{\kok}[3][\G]{#1 \vdash #3 \ : \ #2\to\mathsf{ok}}

\newcommand{\typicalhandler}{\handler{\X}{\C_v}{\X}{\K}{\C}}

\renewcommand{\S}{{\cal S}}
\newcommand{\GS}{\Gamma; \S}


%% Copied from brazil.tex

%% Revised from brazil.tex

\newcommand{\ws}{\vec{w}}
\newcommand{\maybeDownarrow}[1]{\if\relax\detokenize{#1}\relax \else \Downarrow #1 \fi}
\renewcommand{\istypealg}[3][\ws]{#2 \vdash #3 \Leftarrow \mathsf{type} \maybeDownarrow{#1}} % well formed type
\renewcommand{\isfibalg}[4][\ws]{#2 \vdash #3 \Leftarrow \mathsf{fibered} \maybeDownarrow{#1}} % is a fibered type
\renewcommand{\eqtypealg}[4][\ws]{#2 \vdash #3 \thickapprox #4 \maybeDownarrow{#1}} % equal types
\renewcommand{\eqtypepath}[4][\ws]{#2 \vdash #3 \thicksim #4 \maybeDownarrow{#1}} % equal normal-form types
\renewcommand{\eqtermalg}[5][\ws]{#2 \vdash #3 \thickapprox #4 \Leftarrow #5 \maybeDownarrow{#1}} % equal terms of normalized type
\renewcommand{\eqtermext}[5][\ws]{#2 \vdash #3 \simeq #4 \Leftarrow #5 \maybeDownarrow{#1}} % equal terms w/o eta
\renewcommand{\eqpath}[5][\ws]{#2 \vdash #3 \thicksim #4 \maybeDownarrow{#1}}  % equal paths

\newcommand{\equationsin}[1]{\mathsf{equation}\; #1 \;\mathsf{in}\;} % let x = c1 in c2


\begin{document}

\title{TT}
\author{Andrej Bauer \and Matija Pretnar \and Christopher A. Stone}
\maketitle

\section{Abstract syntax}
\label{sec:abstract-syntax}

\begin{equation*}
  \begin{array}{rl@{\qquad}l}
  \text{Expression $\e$}
    \bnf    & \X          & \text{Variable} \\
    \bnfor  & \fun{\X} \C  & \text{Function} \\
    \bnfor  & h           & \text{Handler} \\
    \bnfor  & \cont{\KK} & \text{Continuation value} \\
    \bnfor  & \bterm{\B}{\T}           & \text{Brazilian term} \\
    \bnfor  & \bty{\T}           & \text{Brazilian type} \\
    \bnfor  & \generictuple   & \text{Tuple}\\
    \bnfor  & \c              & \text{TT Constant}\\
    \bnfor  & \inj{\e}        & \text{Coproduct}\\
    \\
    \text{Computation $\C$}
      \bnf  & \val \e                & \text{Pure expression} \\
    \bnfor  & \ttapp{\e_1}{\e_2}   & \text{Application} \\
    \bnfor  & \letin{\X = \C_1} \C_2  & \text{$\mathsf{let}$-binding} \\
    \bnfor  & \opOp{\e} & \text{Operation} \\
    \bnfor  & \withhandle{\e} \C & \text{Handling} \\
%    \bnfor  & \abs{\rgn}{\C} & \text{abstraction} \\
%    \bnfor  & \new{\rgn}{\T} & \text{new variable} \\
    \bnfor  & \kapp{\e_1}{\e_2}   & \text{Invoke a continuation} \\
    \bnfor  & \ascribe{\e_1}{\e_2} & \text{Type ascription} \\
    \bnfor  & \genericprim & \text{Primitive operations}\\
    \bnfor  & \genericmatch& \text{Pattern-match}\\
    \bnfor  & \debruijn{n} & \text{Build Brazilian term: variable} \\
    \bnfor  & \ttlam{\x}{\e} \C   & \text{Build Brazilian term: abstraction} \\
    \bnfor  & \makeApp{\e_1}{\e_2} & \text{Build Brazilian term: application} \\
    \\
    \text{Continuation $\KK$} \bnf    & \hole                & \text{Hole} \\
    \bnfor  & \letin{\X = \KK} \C_2  & \text{$\mathsf{let}$-binding} \\
    \bnfor  & \withhandle{\e} \KK & \text{Handling} \\
    \bnfor  & \ttlam{\x}{\T}{\KK} & \text{Abstraction} \\
%    \bnfor  & \abs{\rgn}{\KK} & \text{abstraction} \\
    \\

  \text{Handler $\h$}
  \bnf & \multicolumn{2}{l}{\typicalhandler}\\
  \\
  \text{Pattern $\pat$}
  \bnf & \tuple{\X_1,\ldots,\X_n}\\
  \bnfor & \inj{\X}\\
  \bnfor & \c\\
\end{array}
\end{equation*}



\section{Operational semantics}
\label{sec:oper-semant}

Results:
%
\begin{equation*}
  \text{Result $R$}
  \begin{aligned}[t]
    &\bnf   {} && \val \e \\
    &\bnfor {} && \opRes{\D}{\e}{\KK} \\
  \end{aligned}
\end{equation*}
%
Judgments:
%
\begin{align*}
  &\evalto[\G]{C}{R} &&\text{$C$ evaluates to result $R$ in context $\G$} \\
  &\resultok[\G]{R}  &&\text{$R$ is a valid result in context $\G$} \\
  &\eok[\G]{\e} &&\text{$\e$ is a valid expression in context $\G$} \\
  &\cok[\G]{\C} &&\text{$\C$ is a valid computation in context $\G$} \\
  &\kok[\G]{\D}{\K} &&\text{$\K$ is a valid continuation in context $\G$,
                              with its hole inside additional binders $\D$} \\
\end{align*}
%
\paragraph{Generic Computations}
\begin{mathpar}
  \infer[eval-val]
  { }
  { \evalto{\val \e}{\val \e}}

  \infer[eval-app]
  {
    \evalto{\subst{\C}{\X}{\e}}{R}
  }
  { \evalto
    {\ttapp{(\fun{\X}{\C})}{\e}}
    {R}
  }

  \infer[eval-let-val]
  {
    \evalto{\C_1}{\val \e}
    \\
    \evalto{\subst{\C_2}{\X}{\e}}{R}
  }
  { \evalto
    {\letin{\X = \C_1} \C_2}
    {R}
  }

  \infer[eval-let-op]
  {
    \evalto
    {\C_1}
    {\opRes{\D}{\e}{\KK}}
  }
  { \evalto
    {\letin{\X = \C_1} \C_2}
    {\opRes{\D}{\e}{\letin{\X = \KK} \C_2}}
  }

  \infer[eval-kapp]
  {
    \evalto[\G,\D]{\KK[\hole:=\e]}{\R}
  }
  { \evalto[\G,\D]
    {\kapp{\cont{\KK}}{\e}}
    {\R}
  }

  \infer[eval-match-tuple]
  {
    P_j = \tuple{\X_1,\ldots,\X_m}\\
    \evalto{\substs{\C_j}{\e_1/\X_1,\ldots,\e_m/\X_m}}{\R}
  }
  {
    \evalto{\genericmatch[{\generictuple[m]}]}
           {\R}
  }

  \infer[eval-match-inj]
  {
    P_j = \inj[k]{X}\\
    \evalto{\subst{\C_j}{\X}{\e}}{\R}
  }
  {
    \evalto{\genericmatch[{\inj[k]{\e}}]}
           {\R}
  }

  \infer[eval-match-const]
  {
    P_j = \c\\
    \evalto{\C_j}{\R}
  }
  {
    \evalto{\genericmatch[\c]}
           {\R}
  }
\end{mathpar}


\paragraph{Operations and Handlers}
\begin{mathpar}
  \infer[eval-op]
  {
  }
  {
    \evalto
    {\opOp{\e}}
    {\opRes{\ctxempty}{\e}{\hole}}
  }

  \infer[eval-handle-val]
  {
    \evalto{\C}{\val \e}
    \\
    \evalto{\subst{\C_v}{\X}{\e}}{\R}
  }
  { \evalto
    {\withhandle{\bigl(\typicalhandler\bigr)}{\C}}
    {\R}
  }

\infer[eval-handle-op-val]
  {
    \h = \typicalhandler
    \\\\
    \evalto{\C}{\opRes{\D}{\e}{\KK_1}}
    \\\\
    \evalto[\G,\D]
    {\substs{\C_i}{\e/\X, \cont{\withhandle{\h}{\KK_1}}/\K}}
    {\val \e}
    \\\\
    \eok{\e}
  }
  { \evalto
    {\withhandle{\h}{\C}}
    {\val \e}
  }

  \infer[eval-handle-op-op]
  {
    \h = \typicalhandler
    \\\\
    \evalto{\C}{\opRes{\D}{\e}{\KK_1}}
    \\\\
    \evalto[\G,\D]
    {\substs{\C_i}{\e/\X, \cont{\withhandle{\h}{\KK_1}}/\K}}
    {\opRes[j]{\D'}{\e'}{\KK_2}}
  }
  { \evalto
    {\withhandle{\h}{\C}}
    {\opRes[j]{(\D,\D')}{\e'}{\KK_2}}
  }
\end{mathpar}

\paragraph{Built-In Functions}
\begin{mathpar}
  \infer[eval-prim]
  {
    \e = \prim[{[\![f]\!]}]{\e_1,\ldots,\e_n}\\
    \bigl[\eok{\e}\bigr]
  }
  {
    \evalto{\genericprim}{\val \e}
  }

  \infer[eval-ascribe]
  {
  }
  {
    \evalto{(\ascribe{\bterm{\B}{\T}}{\T})}{\val \bterm{\B}{\T}}
  }

  \infer[eval-make-var]
  {
    \G = x_{m{-}1}{:}\T_{m{-}1}\ldots,x_0{:}T_0\\
    n < m\\
  }
  {
    \evalto
    {\debruijn{n}}
    {\val \bterm{\x_n}{\T_n}}
  }

  \infer[eval-make-lambda-val]
  {
    \evalto[\ctxextend{\x}{\T}]
    {\C}
    {\val \bterm{\B}{\U}}
  }
  { \evalto
    {\ttlam{\x}{\T} \C}
    {\val (\ttlam{\x}{\T} B)}
  }

  \infer[eval-make-lambda-op]
  {
    \evalto[\ctxextend{\x}{\T}]
    {\C}
    {\opRes{\D}{\e}{\KK}}
  }
  { \evalto
    {\ttlam{\x}{\T} \C}
    {\opRes{(\x{:}\T,\Delta)}{\e}{\ttlam{\x}{\T}{\KK}}}
  }

  \\

  \infer[eval-make-app]
  {
    \eqtypealg{\G}{\T_1}{\T_2}
  }
  { \evalto
      {\makeApp{\bterm{\B_1}{\Prod{\x}{\T_1}{\T_2}}}{\bterm{B_2}{\T_3}}}
      {\bterm{\equationsin{\ws}{\app{B_1}{\x}{\T_1}{\T_2}{B_2}}}
             {\subst{\T_2}{\x}{\B_2}}}
  }
\end{mathpar}

This last rule isn't quite right, because if, for example, $\T_1$ and $\T_3$ are Pi types, it's doubtful that Brazil will think
to look for two names $\B''_1\equiv \B''_3$ in the hint database such that $\T_1 = \El{\gamma}{\B''_1}$ and $\T_3 = \El{\gamma}{\B''_3}$

\subsection*{Type Equivalence Search}

\paragraph{General Type equality}

\begin{mathpar}
  \infer[\rulename{chk-tyeq-refl}]
  { }
  { \eqtypealg[()]{\G}{\T}{\T} }

  \infer[\rulename{chk-tyeq-hnf}]
  { \whnfs{\G}{\T}{\T'}\\
    \whnfs{\G}{\U}{\U'}\\
    \eqtypepath{\G}{\T'}{\U'}
  }
  {
    \eqtypealg{\GH}{\T}{\U}
  }

  \infer[\rulename{chk-tyeq-op}]
  {
    \evalto[\G]{\mathsf{equivTy}(\bty{\T_1},\bty{\T_2})}{\val \ws}
    \\
    \ws =
         \tuple{\ \bterm{\B_1}{\JuEqual{\U_1}{\e'_1}{\e''_1}},
                \ldots,
                \bterm{\B_n}{\JuEqual{\U_n}{\e'_n}{\e''_n}}\ }
    \\
    \eqtypealg[]{\ctxs{\G}{\eqhint{\e'_i}{\e''_i}_{i=1}^{n}}}{\T_1}{\T_2}\\
  }
  {
    \eqtypealg{\G}{\T_1}{\T_2}
  }

\end{mathpar}

\paragraph{Equality of head-normal forms}
\begin{mathpar}
  \infer[\rulename{chk-tyeq-path-refl}]
  { }
  { \eqtypepath[()]{\G}{\T}{\T} }

  \infer[\rulename{chk-tyeq-el}]
  { \alpha = \beta \\
    \eqtermalg{\G}{\e_1}{\e_2}{\Universe{\alpha}}
  }
  {
    \eqtypepath{\G}{\El{\alpha}{\e_1}}{\El{\beta}{\e_2}}
  }

  \infer[\rulename{chk-tyeq-prod}]
  { \eqtypealg[\ws_1]{\GH}{\T_1}{\U_1} \\
    \eqtypealg[\ws_2]{\ctxs{(\ctxextend{\G}{\x}{\T_1})}{\H}}{\T_2}{\U_2}
  }
  { \eqtypepath[\tupleappend{\ws_1}{\ws_2}]{\GH}
    {\Prod{\x}{\T_1}{\T_2}}
    {\Prod{\x}{\U_1}{\U_2}}
  }

  \infer[\rulename{chk-tyeq-paths}]
  {\eqtypealg[\ws_1]{\GH}{\T}{\U}\\
   \eqtermalg[\ws_2]{\GH}{\e_1}{\e'_1}{\T}\\
   \eqtermalg[\ws_3]{\GH}{\e_2}{\e'_2}{\T}
  }
  {\eqtypepath[\tupleappend{\tupleappend{\ws_1}{\ws_2}}{\ws_3}]{\GH}{\PrEqual{\T}{\e_1}{\e_2}}
                  {\PrEqual{\U}{\e'_1}{\e'_2}}}

  \infer[\rulename{chk-tyeq-id}]
  {\eqtypealg{\G}{\T}{\U}\\
   \eqtermalg{\G}{\e_1}{\e'_1}{\T}\\
   \eqtermalg{\G}{\e_2}{\e'_2}{\T}
  }
  {\eqtypepath{\G}{\JuEqual{\T}{\e_1}{\e_2}}
                   {\JuEqual{\U}{\e'_1}{\e'_2}}}

\iffalse
  \infer[\rulename{chk-tyeq-path-op}]
  {
    \evalto[\GH]{\mathsf{equivTy}(\bty{\T_1},\bty{\T_2})}{\val \ws}
    \\
    \ws =
         \tuple{\ \bterm{\B_1}{\JuEqual{\U_1}{\e'_1}{\e''_1}},
                \ldots,
                \bterm{\B_n}{\JuEqual{\U_n}{\e'_n}{\e''_n}}\ }
    \\
    \eqtypealg[]{\ctxs{\G}{\eqhint{\e'_i}{\e''_i}_{i=1}^{n}}}{\T_1}{\T_2}\\
  }
  {
    \eqtypepath{\GH}{\T_1}{\T_2}
  }
\fi

\end{mathpar}
%
The reflexivity rule is not just an optimization, but also handles
equivalence of base types and equivalence of universes.


\subsection{Term equality}
\label{sec:bidirectional-term-equality}

For algorithmic purposes we should try to apply reflexivity and hints before doing
anything else:
%
\begin{mathpar}
  \infer[\rulename{chk-eq-refl}]
  { }
  { \eqtermalg[()]{\G}{\e}{\e}{\T} }
\end{mathpar}
%
Otherwise, we check whether extensionality applies:
%
\begin{mathpar}
  \infer[\rulename{chk-eq-ext}]
  {
    \whnfs{\G}{\T}{\T'} \\
    \eqtermext{\G}{\e_1}{\e_2}{\T'}
  }
  {
    \eqtermalg{\G}{\e_1}{\e_2}{\T}
  }
\end{mathpar}
%
If that doesn't work either, we look for a handler
%
\begin{mathpar}
  \infer[\rulename{chk-tyeq-op}]
  {
    \evalto[\G]{\mathsf{equiv}(\bterm{\e_1}{\T},\bterm{\e_2}{\T},\bty{\T})}{\ws}
    \\
    \ws =
         \tuple{\ \bterm{\B_1}{\JuEqual{\U_1}{\e'_1}{\e''_1}},
                \ldots,
                \bterm{\B_n}{\JuEqual{\U_n}{\e'_n}{\e''_n}}\ }
    \\
    \eqtermalg[]{\ctxs{\G}{\eqhint{\e'_i}{\e''_i}_{i=1}^{n}}}{\e_1}{\e_2}{\T}\\
  }
  {
    \eqtermalg{\G}{\e_1}{\e_2}{\T}
  }
\end{mathpar}
although arguably we should wait until we have head-normalized the terms.

\paragraph{Extensionality}

\begin{mathpar}
  \infer[\rulename{chk-eq-ext-prod}]
  {
    \eqtermalg{\ctxextend{\G}{\x}{\T}}{(\app{\e_1}{\x}{\T}{\U}{\x})}{(\app{\e_2}{\x}{\T}{\U}{\x})}{\U}
  }
  {
    \eqtermext{\G}{\e_1}{\e_2}{\Prod{\x}{\T}{\U}}
  }

  \infer[\rulename{chk-eq-ext-unit}]
  {
  }
  {
    \eqtermext[()]{\G}{\e_1}{\e_2}{\Unit}
  }

  \infer[\rulename{chk-eq-ext-K}]
  {
  }
  {
    \eqtermext[()]{\G}{\e_1}{\e_2}{\JuEqual{\T}{\e_3}{\e_4}}
  }

  \infer[\rulename{chk-eq-ext-whnf}]
  {
    \whnfs{\G}{\e_1}{\e'_1}\\
    \whnfs{\G}{\e_2}{\e'_2}\\\\
    \eqpath{\G}{\e'_1}{\e'_2}{\U}
  }
  {
    \eqtermext{\G}{\e_1}{\e_2}{\T}
  }
\end{mathpar}
%
In \rulename{chk-eq-ext-whnf}, we might want to check whether $\e'_1$ and $\e'_2$ are the
same expressions before invoking the general comparison function.

\renewcommand{\GH}{\G}
\paragraph{Whnf equivalence}
\begin{mathpar}
  \infer[\rulename{chk-eq-whnf-reflexivity}]
  {
  }
  {
    \eqpath[()]{\GH}{\e}{\e}{\T}
  }


  \infer[\rulename{chk-eq-whnf-var}]
  {
    %(\x{:}\T)\in \G
  }
  {
    \eqpath[()]{\GH}{\x}{\x}{\T}
  }

  \infer[\rulename{chk-eq-whnf-app}]
  {
    \eqtypealg[\ws_1]{\GH}{\T_1}{\U_1}\\
    \eqtypealg[\ws_2]{\ctxextend{\G}{\x}{\T_1}}{\T_2}{\U_2}\\\\
    \eqpath[\ws_3]{\GH}{\e_1}{\e'_1}{\Prod{\x}{\T_1}{\T_2}}\\
    \eqtermalg[\ws_4]{\GH}{\e_2}{\e'_2}{\T_1}\\
  }
  {
    \eqpath[\tupleappend{\tupleappend{\tupleappend{\ws_1}{\lambda\x{:}\T_2.\ws_2}}{\ws_3}}{\ws_4}]
           {\GH}{(\app{\e_1}{\x}{\T_1}{\T_2}{\e_2})}
                {(\app{\e'_1}{\x}{\U_1}{\U_2}{\e'_2})}
                {\subst{\T_2}{\x}{\e_2}}
  }
\end{mathpar}
By $\lambda\x{:}\T_2.\ws_2$ we mean lambda-abstracting each individual element of tuple $\ws_2$.

\begin{mathpar}
  \infer[\rulename{chk-eq-whnf-idpath}]
  {
    \eqtypealg[\ws_1]{\GH}{\T}{\U}\\
    \eqtermalg[\ws_2]{\GH}{\e_1}{\e_2}{\T}
  }
  {
    \eqpath[\tupleappend{\ws_1}{\ws_2}]
           {\GH}{\prRefl{\T}{\e_1}}{\prRefl{\U}{\e_2}}
                {\PrEqual{\T}{\e_1}{\e_1}}
  }

  \infer[\rulename{chk-eq-whnf-j}]
  {
   \eqtypealg[\ws_1]{\GH}{\T}{\T'}\\
   \eqtypealg[\ws_2]
     {\ctxs{(\ctxextend{\ctxextend{\ctxextend{\G}{x}{\T}}{y}{\T}}{p}{\PrEqual{\T}{x}{y}})}{\H}}
     {\U}
     {\U'}
   \\
   \eqtermalg[\ws_3]
     {\ctxs{(\ctxextend{\G}{z}{\T})}{\H}}
     {\e_1}
     {\e'_1}
     {\substs{P}{z/x, z/y, (\prRefl{\T}{z})/p}}
     \\
   \eqtermalg[\ws_4]{\GH}{\e_3}{\e'_3}{\T}\\
   \eqtermalg[\ws_5]{\GH}{\e_4}{\e'_4}{\T}\\
   \eqpath[\ws_6]{\GH}{\e_2}{\e'_2}{\PrEqual{\T}{\e_3}{\e_4}}\\
   \ws = \ws_1 \pp (\lambda{\x}{:}{\T}.\lambda{\y}{:}{\T}.\lambda{p}{:}{\PrEqual{\T}{\x}{\y}}.\ws_2) \pp
        \lambda z{:}\T.\ws_3 \pp \ws_4 \pp \ws_5 \pp \ws_6\\
  }
  {\eqpath
     {\GH}
     {\PrElim
        {\T}
        {\abst{x\,y\,p}{\U}}
        {\abst{z}{\e_1}}
        {\e_2}{\e_3}{\e_4}
     }
     {\PrElim
        {\T'}
        {\abst{x\,y\,p}{\U'}}
        {\abst{z}{\e'_1}}
        {\e'_2}{\e'_3}{\e'_4}
     }
     {\substs{P}{\e_2/x, \e_3/y, \e_4/p}}
  }

  \infer[\rulename{chk-eq-whnf-refl}]
  {
    \eqtypealg[\ws_1]{\GH}{\T}{\U}\\
    \eqtermalg[\ws_2]{\GH}{\e_1}{\e_2}{\T}
  }
  {
    \eqpath[\tupleappend{\ws_1}{\ws_2}]
           {\GH}{\juRefl{\T}{\e_1}}{\juRefl{\U}{\e_2}}
                {\JuEqual{\T}{\e_1}{\e_1}}
  }
\end{mathpar}

\paragraph{Whnf equivalence of names}

\begin{mathpar}
  \infer[\rulename{chk-eq-whnf-prod}]
  {\alpha = \alpha' \\
   \beta = \beta' \\\\
   \eqtermalg[\ws_1]{\GH}{\e_1}{\e'_1}{\Universe{\alpha}}\\
   \eqtermalg[\ws_2]{\ctxs{(\ctxextend{\G}{\x}{\El{\alpha}{\e_1}})}{\H}}{\e_2}{\e'_2}{\Universe{\beta}} \\
  }
  {\eqpath[\tupleappend{\ws_1}{\ws_2}]
    {\GH}
    {(\nProd{\alpha}{\beta}{\x}{\e_1}{\e_2})}
    {(\nProd{\alpha'}{\beta'}{\x}{\e'_1}{\e'_2})}
    {\Universe{\gamma}}
  }

\infer[\rulename{chk-eq-whnf-universe}]
  {
    \alpha = \beta \\
  }
  {
    \eqpath[()]{\GH}{\nUniverse{\alpha}}{\nUniverse{\beta}}{\Universe{\gamma}}
  }
%% Subsumed by reflexivity rule
  %\infer[\rulename{chk-eq-whnf-unit}]
  %{
  %}
  %{
    %\eqpath{\GH}{\nUnit}{\nUnit}{\Universe{\zero}}
  %}

\infer[\rulename{chk-eq-whnf-paths}]
  {
    \alpha = \alpha'\\
    \eqtermalg[\ws_1]{\GH}{\e_1}{\e'_1}{\Universe{\alpha}}\\
    \eqtermalg[\ws_2]{\GH}{\e_2}{\e'_2}{\El{\alpha}{\e_1}}\\
    \eqtermalg[\ws_3]{\GH}{\e_3}{\e'_3}{\El{\alpha}{\e_2}}
  }
  {\eqpath[\tupleappend{\tupleappend{\ws_1}{\ws_2}}{\ws_3}]
          {\GH}{\nPrEqual{\alpha}{\e_1}{\e_2}{\e_3}}{\nPrEqual{\alpha'}{\e'_1}{\e'_2}{\e'_3}}{\Universe{\alpha}}}

  \infer[\rulename{chk-eq-whnf-id}]
  {
    \alpha = \alpha'\\
    \eqtermalg[\ws_1]{\GH}{\e_1}{\e'_1}{\Universe{\alpha}}\\
    \eqtermalg[\ws_2]{\GH}{\e_2}{\e'_2}{\El{\alpha}{\e_1}}\\
    \eqtermalg[\ws_3]{\GH}{\e_3}{\e'_3}{\El{\alpha}{\e_2}}
  }
  {\eqpath[\tupleappend{\tupleappend{\ws_1}{\ws_2}}{\ws_3}]
          {\GH}{\nJuEqual{\alpha}{\e_1}{\e_2}{\e_3}}{\nJuEqual{\alpha'}{\e'_1}{\e'_2}{\e'_3}}{\Universe{\alpha}}}

  \infer[\rulename{chk-eq-whnf-coerce}]
  {
    \alpha = \alpha'\\
    \eqtermalg{\GH}{\e_1}{\e'_1}{\Universe{\alpha}} \\
  }
  {
    \eqpath{\GH}{\coerce{\alpha}{\beta}{\e_1}}
                {\coerce{\alpha'}{\beta'}{\e'_1}}
                {\Universe{\beta}}
  }

\end{mathpar}


\section{Well-Formedness}

\subsection*{Expressions}

\begin{mathpar}

  \infer[ok-brazil-term]
  {
    \isterm{\G}{\B}{\T}
  }
  {
    \eok{\bterm{\B}{\T}}
  }


  \infer[ok-brazil-type]
  {
    \istype{\G}{\T}
  }
  {
    \eok{\bty{\T}}
  }

  \infer[ok-tt-var]
  {
  }
  {
    \eok{\X}
  }

  \infer[ok-cont]
  {
    \kok{\D}{\KK}
  }
  {
    \eok{\cont{\KK}}
  }

  \infer[ok-fun]
  {
    \cok{\C}
  }
  {
    \eok{\fun{\X} \C}
  }

  \infer[ok-const]
  {
  }
  {
    \eok{\c}
  }

  \infer[ok-tuple]
  {
    \eok{\e_1} \\
    \cdots \\
    \eok{\e_n} \\
  }
  {
    \eok{\generictuple}
  }

  \infer[ok-inj]
  {
    \eok{\e}
  }
  {
    \eok{\inj{\e}}
  }

  \infer[ok-handler]
  {
    \cok{\C_v} \\
    \cok{\C_1} \\
    \cdots \\
    \cok{\C_n}
  }
  {
    \eok{\typicalhandler}
  }

\end{mathpar}



\subsection*{Results}

\begin{mathpar}

  \infer[ok-result-val]
  {
    \eok{\e}
  }
  {
    \resultok{\val{\e}}
  }

  \infer[ok-result-op]
  {
    \istype{\G,\D}{\T} \\
    \kok{\D}{\K} \\
  }
  {
    \resultok{\opRes{\D}{\T}{\K}}
  }

\end{mathpar}

\subsection*{Continuations}

\begin{mathpar}

  \infer[ok-hole]
  {
  }
  {
    \kok[\G]{\ctxempty}{\hole}
  }

  \infer[ok-let-cont]
  {
    \kok{\D}{\K_1} \\
    \cok{\C_2} \\
  }
  {
    \kok{\D}{(\letin{\X = \K_1} \C_2)}
  }

  \\

  \infer[ok-handle-cont]
  {
    \eok{\e_1} \\
    \kok{\D}{\K_2} \\
  }
  {
    \kok{\D}{(\withhandle{\e_1} \K_2)}
  }

  \infer[ok-lam-cont]
  {
    \kok[\ctxextend{\x}{\T_1}]{\D}{\K_2}
  }
  {
    \kok{(\ctxextend[\D]{\x}{\T_1})}{(\ttlam{\x}{\T_1} \K_2)}
  }
\end{mathpar}

\subsection*{Computations}

\begin{mathpar}

  \infer[ok-val]
  {
    \eok{\e}
  }
  {
    \cok{\val{\e}}
  }

  \infer[ok-app]
  {
    \eok{\e_1} \\
    \eok{\e_2} \\
  }
  {
    \cok{\ttapp{\e_1}{\e_2}}
  }

  \infer[ok-let]
  {
    \cok{\C_1} \\
    \cok{\C_2} \\
  }
  {
    \cok{(\letin{\X = \C_1} \C_2)}
  }

  \infer[ok-op]
  {
    \eok{\e} \\
  }
  {
    \cok{\opOp{\e}}
  }


  \infer[ok-handle]
  {
    \eok{\e_1} \\
    \cok{\C_2} \\
  }
  {
    \cok{(\withhandle{\e_1} \C_2)}
  }

  \infer[ok-cont-app]
  {
    \eok{\e_1}\\
    \eok{\e_2}
  }
  {
    \cok{\kapp{\e_1}{\e_2}}
  }

  \infer[ok-ascribe]
  {
    \eok{\e_1}\\
    \eok{\e_2}\\
  }
  {
    \cok{\ascribe{\e_1}{\e_2}}
  }

  \infer[ok-prim]
  {
    \eok{\e_1}\\
    \cdots
    \eok{\e_n}
  }
  {
    \cok{\prim{\e_1,\ldots,\e_n}}
  }

  \infer[ok-match]
  {
    \eok{\e}\\
    \cok{\C_1} \\ \cdots\\ \cok{\C_n}
  }
  {
    \cok{\genericmatch}
  }

  \infer[ok-make-var]
  {
  }
  {
    \cok{(\debruijn{n})}
  }

  \infer[ok-make-app]
  {
    \eok{\e_1} \\
    \eok{\e_2} \\
  }
  {
    \cok{\makeApp{\e_1}{\e_2}}
  }

  \infer[ok-make-lam1]
  {
    \eok{\e_1} \\
    \cok{\C_2} \\
  }
  {
    \cok{(\ttlam{\x}{\e_1} \C_2)}
  }

  \infer[ok-make-lam2]
  {
    \eok{\e_1} \\
    \cok[\ctxextend{\x}{\T_1}]{\C_2} \\
  }
  {
    \cok{(\ttlam{\x}{\bty{\T_1}} \C_2)}
  }
\end{mathpar}



\begin{lemma}[Substitution]
  \mbox{}
  \begin{enumerate}
    \item If $\cok{\C}$ and $\eok{\e}$ then $\cok{\subst{\C}{\X}{\e}}$.
    \item If $\kok{\D}{\KK}$ and $\eok{\e}$ then $\kok{\D}{\subst{\KK}{\X}{\e}}$.
  \end{enumerate}
\end{lemma}

\begin{lemma}[Continuation Invocation]
  If $\kok{\D}{\KK}$ and $\eok[\G,\D]{\e}$ then $\cok{\KK[\hole:=\e]}$.
\end{lemma}

\begin{lemma}[Preservation]
   If $\cok{\C}$ and $\evalto{\C}{\R}$ then $\resultok{\R}$.
\end{lemma}


\end{document}
