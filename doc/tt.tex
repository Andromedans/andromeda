\documentclass{article}

\usepackage{times}
\usepackage{mathpartir}
\usepackage{amsmath,amsfonts,amssymb}
\usepackage{xcolor}

%%%%%% Macros here %%%%%%

%% Syntax
\newcommand{\bnf}{\ \mathrel{{:}{:}{=}}\ }
\newcommand{\bnfor}{\ \mid\ \ }

\newcommand{\x}{x}     % brazil variable
\newcommand{\y}{y}     % another variable
\newcommand{\z}{z}     % another variable
\newcommand{\C}{C}     % computation
\newcommand{\K}{K}     % continuation variable
\newcommand{\X}{X}     % TT (computation) variable
\newcommand{\e}{e}     % expression
\newcommand{\rgn}{r}   % region
\newcommand{\h}{h}     % handler
\newcommand{\T}{T}     % type
\newcommand{\U}{U}     % another type
\newcommand{\B}{B}     % Brazil term
\newcommand{\R}{R}     % Result

\newcommand{\val}{\mathsf{val}\,} % val e
\newcommand{\letin}[1]{\mathsf{let}\; #1 \;\mathsf{in}\;} % let x = c1 in c2
\newcommand{\op}[3]{\mathsf{op}_{#1}(#2, #3)} % operation
\newcommand{\inhabitPat}[2]{\mathsf{inhabit}(#1, #2)} % the inhabit pattern
\newcommand{\inhabitOp}[1]{\mathsf{inhabit}\,#1} % the inhabit operation
\newcommand{\inhabit}[3]{\mathsf{inhabit}(#1, #2, #3)} % the inhabit result
\newcommand{\withhandle}[1]{\mathsf{with}\;#1\;\mathsf{handle}\;} % handle
\newcommand{\abs}[1]{\mathsf{abs}\;#1\;\mathsf{in}\;} % abstraction
\newcommand{\new}[2]{\mathsf{new}(#1,#2)} % new(r,T)
\newcommand{\fun}[1]{\mathsf{fun}\;#1\Rightarrow} % tt-level function
\newcommand{\app}[2]{#1\,#2} % application
\newcommand{\lam}[2]{\lambda #1 \,{:}\, #2 \,.\,} % Kind of Brazilian lambda
\newcommand{\kapp}[2]{#1[#2]} % continuation application

\newcommand{\handler}[5]{\mathsf{handler}\; \val #1 \mapsto #2 \mid \inhabitPat{#3}{#4} \mapsto #5}

\newcommand{\makeApp}[2]{\mathsf{app}(#1,#2)} % introduce a Brazil application

\newcommand{\debruijn}[1]{\mathsf{debruijn}\,#1} % ugly hack

\newcommand{\subst}[3]{#1[#3/#2]} % substitution
\newcommand{\substs}[2]{#1[#2]} % substitution of many things

\newcommand{\cont}{{\cal K}}     % continuation
\newcommand{\hole}{\diamond}

%%% Operational semantics

\newcommand{\emptyG}{\bullet} % empty context
\newcommand{\emptyH}{\circ} % empty handlers

\newcommand{\G}{\Gamma}
\newcommand{\D}{\Delta}
\newcommand{\xcT}{\vec{\x}{:}\vec{\T}}
\newcommand{\xT}{{\vec{\x}}{\vec{\T}}}
\renewcommand{\H}{\mathcal{H}} % A stack of handlers
\newcommand{\GH}{\G, \H} % combined context & handler

\newcommand{\ctxempty}{\bullet}
\newcommand{\ctxextend}[3][\G]{#1, #2 {:} #3}

\newcommand{\evalto}[3][\G]{#1 \vdash #2 \ \Downarrow\  #3}

\newcommand{\resultok}[2][\G]{#1 \vdash #2 \ \mathsf{ok}}
\newcommand{\eok}[2][\G]{#1 \vdash #2 \ \ \mathsf{ok}}
\newcommand{\cok}[2][\G]{#1 \vdash #2 \ \ \mathsf{ok}}
\newcommand{\kok}[2][\G]{#1 \vdash #2 \ \ \mathsf{ok}}

\newcommand{\istype}[2]{#1 \vdash #2\;\mathsf{type}} % well formed type
\newcommand{\isterm}[3]{#1 \vdash\,#2\,:\,#3} % well formed term

\newcommand{\typeOf}{\mathsf{typeOf}\,}

\newcommand{\typicalhandler}{\handler{\X}{\C_1}{\X}{\K}{\C_2}}

%% Copied from brazil.tex
\newcommand{\Prod}[2]{\mathop{\textstyle\prod_{(#1 {:} #2)}}} % dependent product
\newcommand{\brazilapp}[5]{#1\mathbin{@^{#2{:}#3.#4}} #5} % application
\newcommand{\equationin}[3]{\mathsf{equation}\; #1 \,{:}\, #2 \,{\equiv}\, #3 \; \mathsf{in} \;} % use equation hint
\newcommand{\rewritein}[3]{\mathsf{rewrite}\; #1 \,{:}\, #2 \,{\equiv}\, #3 \; \mathsf{in} \;} % use rewrite hint
\newcommand{\piClose}[2]{\mathsf{max}(#1,#2)}   % the universe containing a pi with given domain and codomain
\newcommand{\uClose}[1]{\mathsf{succ}(#1)}  % the universe containing the given universe's name.
\newcommand{\Universe}[1]{\mathbb{U}_{#1}}
\newcommand{\nJuEqual}[4]{\mathsf{id}^{#1}_{#2}(#3,#4)} % Judgmental Equality type
\newcommand{\coerce}[3]{\mathsf{coerce}^{#1{\mapsto}#2} \, #3}
\newcommand{\El}[2]{\mathsf{El}^{#1}\, #2} % the type named by the term
\newcommand{\ascribe}[2]{#1 \,{:}{:}\, #2} % type ascription





\begin{document}

\title{TT}
\author{Andrej Bauer \and Matija Pretnar \and Christopher A. Stone}
\maketitle

\section{Abstract syntax}
\label{sec:abstract-syntax}

\begin{equation*}
  \text{Expression $\e$}
  \begin{aligned}[t]
    &\bnf   {} && \X          && \text{variable} \\
    &\bnfor {} && \fun{\X} \C  && \text{function} \\
    &\bnfor {} && h           && \text{handler} \\
    &\bnfor {} && B           && \text{Brazilian term} \\
    &\bnfor {} && T           && \text{Brazilian type} \\
  \end{aligned}
\end{equation*}
%
\begin{equation*}
  \text{Computation $\C$}
  \begin{aligned}[t]
    &\bnf   {} && \val \e                && \text{pure expression} \\
    &\bnfor {} && \app{\e_1}{\e_2}   && \text{application} \\
    &\bnfor {} && \letin{\X = \C_1} \C_2  && \text{$\mathsf{let}$-binding} \\
    &\bnfor {} && \inhabitOp{\e} && \text{the inhabitation operation} \\
    &\bnfor {} && \withhandle{\e} \C && \text{handling} \\
%    &\bnfor {} && \abs{\rgn}{\C} && \text{abstraction} \\
%    &\bnfor {} && \new{\rgn}{\T} && \text{new variable} \\
    &\bnfor {} && \kapp{\K}{\e_2}   && \text{Invoke a continuation variable} \\
    &\bnfor {} && \ascribe{\e_1}{\e_2} && \text{Type Ascription} \\
    &\bnfor {} && \debruijn{n} && \text{Build Brazilian term: variable} \\
    &\bnfor {} && \lam{\x}{\e} \C   && \text{Build Brazilian term: abstraction} \\
    &\bnfor {} && \makeApp{\e_1}{\e_2} && \text{Build Brazilian term: application} \\
  \end{aligned}
\end{equation*}
%
\begin{equation*}
  \text{(Capturing!) Continuation $\cont$}
  \begin{aligned}[t]
    &\bnf   {} && \hole                && \text{hole} \\
    &\bnfor {} && \letin{\X = \cont_1} \C_2  && \text{$\mathsf{let}$-binding} \\
    &\bnfor {} && \withhandle{\e} \cont && \text{handling} \\
%    &\bnfor {} && \abs{\rgn}{\cont} && \text{abstraction} \\
    &\bnfor {} && \lam{\x}{\T} \cont   && \text{$\lambda$-abstraction} \\
  \end{aligned}
\end{equation*}

%
\begin{equation*}
  \text{Handler $\h$}
  \bnf (\typicalhandler)\\
\end{equation*}



\section{Operational semantics}
\label{sec:oper-semant}

Results:
%
\begin{equation*}
  \text{Result $R$}
  \begin{aligned}[t]
    &\bnf   {} && \val \e \\
    &\bnfor {} && \inhabit{\D}{\T}{\cont} \\
  \end{aligned}
\end{equation*}
%
Judgement:
%
\begin{align*}
  &\evalto[\G]{C}{R} &&\text{$C$ evaluates to result $R$ in context $\G$} \\
  &\resultok[\G]{R}  &&\text{$R$ is a valid result in context $\G$} \\
  &\eok[\G]{\e} &&\text{$\e$ is a valid expression in context $\G$} \\
  &\cok[\G]{\C} &&\text{$\C$ is a valid computation in context $\G$} \\
  &\kok[\G]{\K} &&\text{$\K$ is a valid continuation in context $\G$} \\
\end{align*}
%
\paragraph{Control Flow}
\begin{mathpar}

  \infer[eval-val]
  { }
  { \evalto{\val \e}{\val \e}}

  \infer[eval-app]
  {
    \evalto{\subst{\C}{\X}{\e}}{R}
  }
  { \evalto
    {\app{(\fun{\X}{\C})}{\e}}
    {R}
  }

  \infer[eval-let-val]
  {
    \evalto{\C_1}{\val \e}
    \\
    \evalto{\subst{\C_2}{\X}{\e}}{R}
  }
  { \evalto
    {\letin{\X = \C_1} \C_2}
    {R}
  }

  \infer[eval-let-op]
  {
    \evalto
    {\C_1}
    {\inhabit{\D}{\T}{\cont}}
  }
  { \evalto
    {\letin{\X = \C_1} \C_2}
    {\inhabit{\D}{\T}{\letin{\X = \cont} \C_2}}
  }

  \infer[eval-ascribe]
  {
    \typeOf{\e} = \T\\
  }
  {
    \evalto{(\ascribe{\e}{\T})}{\val \e}
  }


\end{mathpar}

\paragraph{Operations and Handlers}
\begin{mathpar}
  \infer[eval-inhabit]
  {
  }
  {
    \evalto
    {\inhabitOp{\T}}
    {\inhabit{\ctxempty}{\T}{\hole}}
  }

  \infer[eval-handle-val]
  {
    \evalto{\C}{\val \e}
    \\
    \evalto{\subst{\C_1}{\x}{\e}}{R}
  }
  { \evalto
    {\withhandle{(\typicalhandler)}{\C}}
    {R}
  }

  \infer[eval-handle-op]
  {
    \evalto{\C}{\inhabit{\D}{\T}{\cont}}
    \\\\
    \cont' = \bigl(\withhandle{\h}{\letin{\X' = \ascribe{\hole}{\T}}{\kapp{\cont}{\X'}}}\bigr)\\
    \\\\
    \h = \typicalhandler
    \\\\
    \evalto[\G,\D]
    {\substs{\C_2}{\T/\X, \cont'/\K}}
    {R}
    \\\\
    \resultok{R}\\
  }
  { \evalto
    {\withhandle{\h}{\C}}
    {R}
  }
\end{mathpar}

\paragraph{Computations that build Brazil terms}
\begin{mathpar}
  \infer[eval-make-var]
  {
    \G = x_{m{-}1}{:}\T_{m{-}1}\ldots,x_0{:}T_0\\
    n < m\\
  }
  {
    \evalto
    {\debruijn{n}}
    {\x_n}
  }

  \infer[eval-make-lambda-val]
  {
    \evalto[\ctxextend{\x}{\T}]
    {\C}
    {\val B}
  }
  { \evalto
    {\lam{\x}{\T} \C}
    {\val (\lam{\x}{\T} B)}
  }

  \infer[eval-make-lambda-op]
  {
    \evalto[\ctxextend{\x}{\T}]
    {\C}
    {\inhabit{\D}{\U}{\cont}}
  }
  { \evalto
    {\lam{\x}{\T} \C}
    {\inhabit{(\x{:}\T,\Delta)}{\U}{\lam{\x}{\T}{\cont}}}
  }

  \\

  \infer[eval-make-app]
  {
    \mathsf{typeOf}(B_1) = \Prod{\x}{\T_1}{\T_2}\\
    \mathsf{typeOf}(B_2) = \T_3 \\
    \\\\
    \mathsf{nameOf}(\T_1) = B'_1 : \Universe{\alpha}\\
    \mathsf{nameOf}(\T_3) = B'_3 : \Universe{\beta}\\
    \\\\
    \gamma = \piClose{\alpha}{\beta}\\
    \gamma' = \uClose{\gamma}\\
    \\\\
    \B''_1 = \coerce{\alpha}{\gamma}{\B'_1}\\
    \B''_3 = \coerce{\beta}{\gamma}{\B'_3}\\
    \\\\
    \evalto
      {\bigl(\letin{\X = \inhabitOp{(\nJuEqual{\gamma'}{\Universe{\gamma}}{\B''_1}{\B''_3})}}
         {\val (\equationin{\X}{\B''_1}{\B''_3}{(\brazilapp{B_1}{\x}{\T_1}{\T_2}{B_2})})}\bigr)}
      {R}
  }
  { \evalto
      {\makeApp{B_1}{B_2}}
      {R}
  }
\end{mathpar}

This last rule isn't quite right, because if, for example, $\T_1$ and $\T_3$ are Pi types, it's doubtful that Brazil will think
to look for two names $\B''_1\equiv \B''_3$ in the hint database such that $\T_1 = \El{\gamma}{\B''_1}$ and $\T_3 = \El{\gamma}{\B''_3}$

\section{Well-Formedness}

\subsection*{Expressions}

\begin{mathpar}

  \infer[ok-brazil-term]
  {
    \isterm{\G}{B}{\T}
  }
  {
    \eok{B}
  }


  \infer[ok-brazil-type]
  {
    \istype{\G}{\T}
  }
  {
    \eok{\T}
  }

  \infer[ok-var]
  {
  }
  {
    \eok{\X}
  }

  \infer[ok-fun]
  {
    \cok{\C}
  }
  {
    \eok{\fun{\X} \C}
  }

  \infer[ok-handler]
  {
    \cok{\C_1} \\
    \cok{\C_2} \\
  }
  {
    \eok{(\typicalhandler)}
  }
\end{mathpar}

\subsection*{Continuations}

\begin{mathpar}

  \infer[ok-hole]
  {
  }
  {
    \kok{\hole}
  }

  \infer[ok-let-cont]
  {
    \kok{\K_1} \\
    \cok{\C_2} \\
  }
  {
    \kok{(\letin{\X = \K_1} \C_2)}
  }

  \\

  \infer[ok-handle-cont]
  {
    \eok{\e_1} \\
    \kok{\K_2} \\
  }
  {
    \kok{(\withhandle{\e_1} \K_2)}
  }

  \infer[ok-lam-cont]
  {
    \kok[\ctxextend{\x}{\T_1}]{\K_2}
  }
  {
    \kok{(\lam{\x}{\T_1} \K_2)}
  }
\end{mathpar}

\subsection*{Computations}

\begin{mathpar}

  \infer[ok-val]
  {
    \eok{\e}
  }
  {
    \cok{\val{\e}}
  }

  \infer[ok-app]
  {
    \eok{\e_1} \\
    \eok{\e_2} \\
  }
  {
    \cok{\app{\e_1}{\e_2}}
  }

  \infer[ok-let]
  {
    \cok{\C_1} \\
    \cok{\C_2} \\
  }
  {
    \cok{(\letin{\X = \C_1} \C_2)}
  }

  \infer[ok-inhabit]
  {
    \eok{\e} \\
  }
  {
    \cok{\inhabitOp{\e}}
  }


  \infer[ok-handle]
  {
    \eok{\e_1} \\
    \cok{\C_1} \\
  }
  {
    \cok{(\withhandle{\e_1} \C_2)}
  }

  \infer[ok-cont-app]
  {
    \eok{\e_2}
  }
  {
    \cok{\kapp{\K_1}{\e_2}}
  }

  \\

  \infer[ok-make-var]
  {
  }
  {
    \cok{(\debruijn{n})}
  }

  \infer[ok-make-lam]
  {
    \eok{\e_1} \\
    \cok{\C_2} \\
  }
  {
    \cok{(\lam{\x}{\e_1} \C_2)}
  }

  \infer[ok-make-app]
  {
    \eok{\e_1} \\
    \eok{\e_2} \\
  }
  {
    \cok{\makeApp{\e_1}{\e_2}}
  }

\end{mathpar}



\subsection*{Results}

\begin{mathpar}

  \infer[ok-result-val]
  {
    \eok{\e}
  }
  {
    \resultok{\val{\e}}
  }

  \infer[ok-result-inhabit]
  {
    \eok[\G,\D]{\T} \\
    \kok{\K} \\
  }
  {
    \resultok{\inhabit{\D}{\T}{\K}}
  }

\end{mathpar}

\section{Operational semantics with a Stack of Handlers}
\label{sec:oper-semant-stack}

\paragraph{Syntax}

\begin{equation*}
  \text{(Capturing!) Continuation $\cont$}
  \begin{aligned}[t]
    &\bnf   {} && \hole                && \text{hole} \\
    &\bnfor {} && \letin{\X = \cont_1} \C_2  && \text{$\mathsf{let}$-binding} \\
%    &\bnfor {} && \abs{\rgn}{\cont} && \text{abstraction} \\
    &\bnfor {} && \lam{\x}{\T} \cont   && \text{$\lambda$-abstraction} \\
  \end{aligned}
\end{equation*}

\paragraph{Judgement}

\renewcommand{\S}{{\cal S}}
\newcommand{\GS}{\Gamma; \S}
\newcommand{\emptystack}{\bullet}

\begin{align*}
  &\evalto[\GS]{C}{\e} &&\text{$C$ evaluates to $e$ in context $\G$} \\
\end{align*}
%
\paragraph{Control Flow}
\begin{mathpar}

  \infer[eval-val]
  { }
  { \evalto[\G;\emptystack]{\val \e}{\e}}

  \infer[eval-val]
  { \h = \typicalhandler \\
    \evalto[\GS]{\subst{\C_1}{\X}{\e_1}}{\e}}
  { \evalto[\G;(\S,\h)]{\val \e_1}{\e}}

  \infer[eval-app]
  {
    \evalto[\GS]{\subst{\C}{\X}{\e_1}}{\e}
  }
  { \evalto[\GS]
    {\app{(\fun{\X}{\C})}{\e_1}}
    {\e}
  }

  \infer[eval-let-val]
  {
    \evalto[\GS]{\C_1}{\e_1}
    \\
    \evalto[\GS]{\subst{\C_2}{\X}{\e_1}}{\e}
  }
  { \evalto[\GS]
    {\letin{\X = \C_1} \C_2}
    {\e}
  }

  \infer[eval-ascribe]
  {
    \typeOf{\e} = \T\\
  }
  {
    \evalto{(\ascribe{\e}{\T})}{\val \e}
  }

\end{mathpar}

\paragraph{Operations and Handlers}
\begin{mathpar}
  \infer[eval-inhabit]
  {
    \h = \typicalhandler\\
    \cont' = \bigl(\withhandle{\h}{\letin{\X' = \ascribe{\hole}{\T}}{\kapp{\cont_\D}{\X'}}}\bigr)\\
    \evalto[(\G,\D);\S]{\substs{\C_2}{\T/\X, \cont'/\K}}{\e}\\
    \eok[\G]{\e}
  }
  {
    \evalto[\G;(\S,\h)]
    {{\cont_\D}[\inhabitOp{\T}]}
    {\e}
  }

  \infer[eval-handle]
  {
    \evalto[\G;(S,\h)]{\C}{\e}
  }
  { \evalto[\GS]
    {\withhandle{\h}{\C}}
    {\e}
  }

\end{mathpar}

\paragraph{Computations that build Brazil terms}
\begin{mathpar}
  \infer[eval-make-var]
  {
    \G = x_{m{-}1}{:}\T_{m{-}1}\ldots,x_0{:}T_0\\
    n < m\\
  }
  {
    \evalto[\GS]
    {\debruijn{n}}
    {\x_n}
  }

  \infer[eval-make-lambda-val]
  {
    \evalto[(\ctxextend{\x}{\T});\S]
    {\C}
    {B}
  }
  { \evalto[\GS]
    {\lam{\x}{\T} \C}
    {\lam{\x}{\T} B}
  }

  \infer[eval-make-app]
  {
    \mathsf{typeOf}(B_1) = \Prod{\x}{\T_1}{\T_2}\\
    \mathsf{typeOf}(B_2) = \T_3 \\
    \\\\
    \mathsf{nameOf}(\T_1) = B'_1 : \Universe{\alpha}\\
    \mathsf{nameOf}(\T_3) = B'_3 : \Universe{\beta}\\
    \\\\
    \gamma = \piClose{\alpha}{\beta}\\
    \gamma' = \uClose{\gamma}\\
    \\\\
    \B''_1 = \coerce{\alpha}{\gamma}{\B'_1}\\
    \B''_3 = \coerce{\beta}{\gamma}{\B'_3}\\
    \\\\
    \evalto[\GS]
      {\bigl(\letin{\X = \inhabitOp{(\nJuEqual{\gamma'}{\Universe{\gamma}}{\B''_1}{\B''_3})}}
         {\val (\equationin{\X}{\B''_1}{\B''_3}{(\brazilapp{B_1}{\x}{\T_1}{\T_2}{B_2})})}\bigr)}
      {R}
  }
  { \evalto[\GS]
      {\makeApp{B_1}{B_2}}
      {R}
  }
\end{mathpar}

This last rule isn't quite right, because if, for example, $\T_1$ and $\T_3$ are Pi types, it's doubtful that Brazil will think
to look for two names $\B''_1\equiv \B''_3$ in the hint database such that $\T_1 = \El{\gamma}{\B''_1}$ and $\T_3 = \El{\gamma}{\B''_3}$

\end{document}
