\documentclass{article}

\usepackage{times}
\usepackage{mathpartir}
\usepackage{amsmath,amsfonts,amssymb}
\usepackage{xcolor}

%%%%%% Macros here %%%%%%

%% Syntax
\newcommand{\bnf}{\ \mathrel{{:}{:}{=}}\ }
\newcommand{\bnfor}{\ \mid\ \ }

\newcommand{\x}{x}     % variable
\newcommand{\y}{y}     % another variable
\newcommand{\C}{C}     % computation
\newcommand{\e}{e}     % expression
\newcommand{\rgn}{r}   % region
\newcommand{\h}{h}     % handler
\newcommand{\T}{T}     % type

\newcommand{\val}{\mathsf{val}\,} % val e
\newcommand{\letin}[1]{\mathsf{let}\; #1 \;\mathsf{in}\;} % let x = c1 in c2
\newcommand{\op}[3]{\mathsf{op}_{#1}(#2, #3)} % operation
\newcommand{\inhabit}[2]{\mathsf{inhabit}(#1, #2)} % the inhabit operation
\newcommand{\withhandle}[1]{\mathsf{with}\;#1\;\mathsf{handle}\;} % handle
\newcommand{\cont}[2]{(#1 \,.\, #2)} % continuation
\newcommand{\abs}[1]{\mathsf{abs}\;#1\;\mathsf{in}\;} % abstraction
\newcommand{\new}[2]{\mathsf{new}(#1,#2)} % new(r,T)
\newcommand{\fun}[1]{\mathsf{fun}\;#1\Rightarrow} % tt-level function
\newcommand{\app}[2]{#1\,#2} % application
\newcommand{\lam}[2]{\lambda #1 \,{:}\, #2 \,.\,} % Kind of Brazilian lambda

\newcommand{\handler}[5]{\mathsf{handler}\; \val #1 \mapsto #2 \mid \inhabit{#3}{#4} \mapsto #5}

\newcommand{\makeApp}[2]{\mathsf{app}(#1,#2)} % introduce a Brazil application

\newcommand{\debruijn}[1]{\mathsf{debruijn}\,#1} % ugly hack

\newcommand{\subst}[3]{#1[#3/#2]} % substitution
\newcommand{\substs}[2]{#1[#2]} % substitution of many things

%%% Operational semantics

\newcommand{\emptyG}{\bullet} % empty context
\newcommand{\emptyH}{\circ} % empty handlers

\newcommand{\G}{\Gamma}
\renewcommand{\H}{\mathcal{H}} % A stack of handlers
\newcommand{\GH}{\G, \H} % combined context & handler

\newcommand{\ctxextend}[3][\G]{#1, #2 {:} #3}

\newcommand{\evalto}[3][\GH]{#1 \vdash #2 \Downarrow #3}
\newcommand{\resultok}[2][\GH]{#1 \vdash #2 \ \mathsf{ok}}
\begin{document}

\title{TT}
\author{Andrej Bauer \and Matija Pretnar \and Christopher A. Stone}
\maketitle

\section{Abstract syntax}
\label{sec:abstract-syntax}

\begin{equation*}
  \text{Expression $\e$}
  \begin{aligned}[t]
    &\bnf   {} && \x          && \text{variable} \\
    &\bnfor {} && \fun{x} \C  && \text{function} \\
    &\bnfor {} && h           && \text{handler} \\
    &\bnfor {} && B           && \text{Brazilian term} \\
    &\bnfor {} && T           && \text{Brazilian types}
  \end{aligned}  
\end{equation*}
%
\begin{equation*}
  \text{Computation $\C$}
  \begin{aligned}[t]
    &\bnf   {} && \val \e                && \text{pure expression} \\
    &\bnfor {} && \app{\e_1}{\e_2}   && \text{application} \\
    &\bnfor {} && \letin{\x = \C_1} \C_2  && \text{$\mathsf{let}$-binding} \\
    &\bnfor {} && \inhabit{\e}{\cont{\x}{\C}} && \text{the inhabitation operation} \\
    &\bnfor {} && \debruijn{n} && \text{ugly hack to be deleted later} \\
    &\bnfor {} && \withhandle{\e} \C && \text{handling} \\
%    &\bnfor {} && \abs{\rgn}{\C} && \text{abstraction} \\
%    &\bnfor {} && \new{\rgn}{\T} && \text{new variable} \\
    &\bnfor {} && \lam{\x}{\e} \C   && \text{$\lambda$-abstraction} \\
    &\bnfor {} && \makeApp{\e_1}{\e_2}   && \text{Brazilian application}
  \end{aligned}  
\end{equation*}
%
\begin{equation*}
  \text{Handler $\h$}
  \bnf \handler{\x}{\C_1}{\x}{k}{\C_2}
\end{equation*}



\section{Operational semantics}
\label{sec:oper-semant}

Results:
%
\begin{equation*}
  \text{Result $R$}
  \begin{aligned}[t]
    &\bnf   {} && \val \e \\
    &\bnfor {} && \lam{\vec{x}}{\vec{\T}} \inhabit{\e}{\cont{\x}{\C}} \\
  \end{aligned}  
\end{equation*}
%
Judgement:
%
\begin{align*}
  &\evalto[\GH]{C}{e} &&\text{$C$ evaluates to $e$ in context $\GH$} \\
  &\resultok[\GH]{R}  &&\text{$R$ is a valid result in context $\GH$}
\end{align*}
%
Rules:
\begin{mathpar}

  \infer[eval-val-1]
  { }
  { \evalto[\G, \emptyH]{\val \e}{\val \e}}
  
  \infer[eval-val-2]
  { }
  { \evalto[\G, (h, \H)]{\val \e}{\val \e}}

  \infer[eval-app]
  {
    \evalto{\subst{\C}{\x}{\e}}{R}
  }
  { \evalto
    {\app{(\fun{\x}{\C})}{\e}}
    {R}
  }

  \infer[eval-let-val]
  {
    \evalto{\C_1}{\val \e}
    \\
    \evalto{\subst{\C_2}{\x}{\e}}{R}
  }
  { \evalto
    {\letin{\x = \C_1} \C_2}
    {R}
  }

  \infer[eval-let-op]
  {
    \evalto
    {\C_1}
    {\lam{\vec{\x}}{\vec{\T}} \inhabit{\e}{\cont{y}{\C'_1}}}
    \\
    \evalto[\ctxextend{\vec{\x}}{\vec{\T}}]
    { \inhabit{\e}{\cont{y}{\C'_1}} }
    { R }
  }
  { \evalto
    {\letin{\x = \C_1} \C_2}
    {\lam{\vec{\x}}{\vec{\T}} \inhabit{\e}{\cont{y}{\letin{x = \C'_1} \C_2}} }
  }

  \infer[eval-handle-val]
  {
    \evalto{\C}{\val \e}
    \\
    \evalto{\subst{\C_1}{\x}{\e}}{R}
  }
  { \evalto
    {\withhandle{\h}{\C}}
    {R}
  }

  \infer[eval-handle-op]
  {
    \evalto{\C}{\lam{\vec{\x}}{\vec{\T}} \inhabit{\e}{\cont{y}{\C'}}}
    \\
    \evalto[\ctxextend{\vec{\x}}{\vec{\T}}]
    {\substs{\C_2}{\e/\x, (\fun{y} \withhandle{h}{C'})/k}}
    {R}
    \\\\
    \evalto{\lam{\vec{\x}}{\vec{\T}} R}{R'}
  }
  { \evalto
    {\withhandle{\h}{\C}}
    {R'}
  }

  \infer[eval-lambda-val]
  {
    \evalto[\ctxextend{\x}{\T}]
    {\C}
    {\val \e}
  }
  { \evalto
    {\lam{\x}{\T} \C}
    {\val (\lam{\x}{\T} \e)}
  }

  \infer[eval-lambda-op]
  {
    \evalto[\ctxextend{\x}{\T}]
    {\C}
    {\lam{\vec{\x}}{\vec{\T}} \inhabit{\e}{\cont{y}{\C'}}}
  }
  { \evalto
    {\lam{\x}{\T} \C}
    {\lam{\x}{\T} \lam{\vec{\x}}{\vec{\T}} \inhabit{\e}{\cont{y}{\C'}}}
  }

\end{mathpar}


\end{document}
