\documentclass{article}

\usepackage{times}
\usepackage{mathpartir}
\usepackage{amsmath,amsfonts,amssymb}
\usepackage{xcolor}
\usepackage{ifmtarg}

\newtheorem{lemma}{Lemma}

%%%%%% Macros here %%%%%%

\input brazilmacros


%% Syntax

\newcommand{\C}{C}     % computation
\newcommand{\K}{K}     % continuation variable
\newcommand{\X}{X}     % TT (computation) variable
\newcommand{\rgn}{r}   % region
\newcommand{\h}{h}     % handler
\newcommand{\B}{B}     % Brazil term
\newcommand{\R}{R}     % Result
\newcommand{\KK}{{\cal K}} % continuation term with hole


\newcommand{\val}{\mathsf{val}\,} % val e
\newcommand{\letin}[1]{\mathsf{let}\; #1 \;\mathsf{in}\;} % let x = c1 in c2
\newcommand{\op}[3]{\mathsf{op}_{#1}(#2, #3)} % operation
\newcommand{\inhabitPat}[2]{\mathsf{inhabit}(#1, #2)} % the inhabit pattern
\newcommand{\inhabitOp}[1]{\mathsf{inhabit}\,#1} % the inhabit operation
\newcommand{\inhabitRes}[3]{\mathsf{inhabit}(#1, #2, #3)} % the inhabit result
\newcommand{\opPat}[3][i]{\mathsf{op}_{#1}(#2, #3)} % the operation pattern
\newcommand{\opOp}[2][i]{\mathsf{op}_{#1}\,#2} % the operation operation
\newcommand{\opRes}[4][i]{\mathsf{op}_{#1}(#2, #3, #4)} % the operation result
\newcommand{\withhandle}[1]{\mathsf{with}\;#1\;\mathsf{handle}\;} % handle
\newcommand{\abs}[1]{\mathsf{abs}\;#1\;\mathsf{in}\;} % abstraction
\newcommand{\new}[2]{\mathsf{new}(#1,#2)} % new(r,T)
\newcommand{\fun}[1]{\mathsf{fun}\;#1\Rightarrow} % tt-level function
\newcommand{\ttapp}[2]{#1\,#2} % application
\newcommand{\ttlam}[2]{\lambda #1 \,{:}\, #2 \,.\,} % Kind of Brazilian lambda
\newcommand{\kapp}[2]{#1{}[#2]} % continuation application
\newcommand{\bterm}[1]{\langle\,#1\,\rangle} % brazil (typed) term as data
\newcommand{\bty}[1]{\langle\,#1\,\rangle} % brazil type as data

\newcommand{\handler}[6][n]{\mathsf{handler}\; \val #2 \mapsto #3 \mid (\opPat{#4}{#5} \mapsto #6_{#1})_{i=1}^{#1}}
\newcommand{\fhandler}[8][n]{\mathsf{handler}\; \val #2 \mapsto #3 \mid (\opPat{#4}{#5} \mapsto #6_{#1})_{i=1}^{#1} \mid \mathsf{finally}\; #7\mapsto #8}

\newcommand{\makeApp}[2]{\mathsf{app}(#1,#2)} % introduce a Brazil application
\newcommand{\debruijn}[1]{\mathsf{debruijn}\,#1} % ugly hack


\newcommand{\cont}[2][\G,\D]{\mathsf{cont}(#1,#2)}     % continuation
\newcommand{\hfcont}{\hat{\cal K}}     % handler-free continuation
\newcommand{\hole}{\diamond}
\newcommand{\tuple}[1]{(#1)}
\newcommand{\generictuple}[1][n]{\tuple{e_1,\ldots,\e_{#1}}}

\newcommand{\pp}{\mathop{\text{\footnotesize{+}{+}}}}
\newcommand{\tupleappend}[2]{#1 \pp #2}


\newcommand{\pat}{P}
\newcommand{\match}[2]{\mathsf{match}\; #1\; \mathsf{with}\;#2}
\newcommand{\genericPats}[1][n]{(\pat_i \Rightarrow \C_i)_{i=1}^n}
\newcommand{\genericmatch}[1][\e]{\match{#1}{\genericPats}}

\renewcommand{\c}{c} % constant
\newcommand{\prim}[2][f]{#1(#2)} % primitive  application
\newcommand{\genericprim}{\prim{\e_1,\ldots,\e_n}}

\newcommand{\inj}[2][i]{\mathsf{inj}_{#1}\, #2}

%%% Operational semantics

\newcommand{\xcT}{\vec{\x}{:}\vec{\T}}
\newcommand{\xT}{{\vec{\x}}{\vec{\T}}}


\newcommand{\evalto}[3][\G]{#1 \vdash #2 \ \Downarrow\  #3}

\newcommand{\resultok}[2][\G]{#1 \vdash #2 \ \mathsf{ok}}
\newcommand{\eok}[2][\G]{#1 \vdash #2 \ \ \mathsf{ok}}
\newcommand{\cok}[2][\G]{#1 \vdash #2 \ \ \mathsf{ok}}
\newcommand{\kok}[3][\G]{#1 \vdash #3 \ : \ #2\to\mathsf{ok}}

\newcommand{\typicalhandler}{\handler{\X}{\C_v}{\X}{\K}{\C}}
\newcommand{\finallyhandler}{\fhandler{\X}{\C_v}{\X}{\K}{\C}{\X_f}{\C_f}}

\newcommand{\assertEquivType}[3]{\mathsf{check}\;#1\equiv #2\;\mathsf{using}\;#3\;\mathsf{in}\;}

\newcommand{\qqquad}{\qquad\quad}     % three quads of space
\newcommand{\qqqquad}{\qquad\qquad}   % four quads of space

%% Copied from brazil.tex

%% Revised from brazil.tex

\newcommand{\ws}{\vec{w}}
\newcommand{\maybeDownarrow}[1]{\if\relax\detokenize{#1}\relax \else \Downarrow #1 \fi}
\renewcommand{\istypealg}[3][\ws]{#2 \vdash #3 \Leftarrow \mathsf{type} \maybeDownarrow{#1}} % well formed type
\renewcommand{\isfibalg}[4][\ws]{#2 \vdash #3 \Leftarrow \mathsf{fibered} \maybeDownarrow{#1}} % is a fibered type
\renewcommand{\eqtypealg}[4][\ws]{#2 \vdash #3 \thickapprox #4 \maybeDownarrow{#1}} % equal types
\renewcommand{\eqtypepath}[4][\ws]{#2 \vdash #3 \thicksim #4 \maybeDownarrow{#1}} % equal normal-form types
\renewcommand{\eqtermalg}[5][\ws]{#2 \vdash #3 \thickapprox #4 \Leftarrow #5 \maybeDownarrow{#1}} % equal terms of normalized type
\renewcommand{\eqtermext}[5][\ws]{#2 \vdash #3 \simeq #4 \Leftarrow #5 \maybeDownarrow{#1}} % equal terms w/o eta
\renewcommand{\eqpath}[5][\ws]{#2 \vdash #3 \thicksim #4 \maybeDownarrow{#1}}  % equal paths

\newcommand{\equationsin}[1]{\mathsf{equation}\; #1 \;\mathsf{in}\;} % let x = c1 in c2



\newcommand{\eqtypealgOp}[2]{\mathsf{equivTy}(\bty{#1},\bty{#2})} % equal types
\newcommand{\eqtypepathOp}[2]{\mathsf{equivWhnfTy}(\bty{#1}, \bty{#2})} % equal normal-form types
\newcommand{\eqtermalgOp}[3]{\mathsf{equivTerm}({#1},{#2},#3)} % equal terms of normalized type
\newcommand{\eqtermextOp}[3]{\mathsf{equivExtTerm}(#1,#2,\bty{#3})} % equal terms of normalized type
\newcommand{\eqpathOp}[2]{\mathsf{equivWhnf}(\bterm{#1},\bterm{#2})} % equal normalized term

\begin{document}

\title{TT}
\author{Andrej Bauer \and Matija Pretnar \and Christopher A. Stone}
\maketitle

\section{Abstract syntax}
\label{sec:abstract-syntax}

\begin{equation*}
  \begin{array}{rl@{\qquad}l}
  \text{Expression $\e$}
    \bnf    & \X          & \text{Variable} \\
    \bnfor  & \fun{\X} \C  & \text{Function} \\
    \bnfor  & h           & \text{Handler} \\
    \bnfor  & \cont{\KK} & \text{Continuation value} \\
    \bnfor  & \bterm{\B}         & \text{Brazilian term} \\
    \bnfor  & \bty{\T}           & \text{Brazilian type} \\
    \bnfor  & \generictuple   & \text{Tuple}\\
    \bnfor  & \c              & \text{TT Constant}\\
    \bnfor  & \inj{\e}        & \text{Coproduct}\\[5pt]

    \text{Computation $\C$}
      \bnf  & \val \e                & \text{Pure expression} \\
    \bnfor  & \ttapp{\e_1}{\e_2}   & \text{Application} \\
    \bnfor  & \letin{\X = \C_1} \C_2  & \text{$\mathsf{let}$-binding} \\
    \bnfor  & \opOp{\e} & \text{Operation} \\
    \bnfor  & \withhandle{\e} \C & \text{Handling} \\
%    \bnfor  & \abs{\rgn}{\C} & \text{abstraction} \\
%    \bnfor  & \new{\rgn}{\T} & \text{new variable} \\
    \bnfor  & \kapp{\e_1}{\e_2}   & \text{Invoke a continuation} \\
    \bnfor  & \ascribe{\e_1}{\e_2} & \text{Type ascription} \\
    \bnfor  & \genericprim & \text{Primitive operations}\\
    \bnfor  & \genericmatch& \text{Pattern-match}\\
    \bnfor  & \assertEquivType{\T_1}{\T_2}{\e}{\C} & \text{Run-time Equivalence Check}\\
    \bnfor  & \debruijn{n} & \text{Build Brazilian term: variable} \\
    \bnfor  & \ttlam{\x}{\e} \C   & \text{Build Brazilian term: abstraction} \\
    \bnfor  & \makeApp{\e_1}{\e_2} & \text{Build Brazilian term: application} \\[5pt]

    \text{Continuation $\KK$} \bnf    & \hole                & \text{Hole} \\
    \bnfor  & \letin{\X = \KK} \C_2  & \text{$\mathsf{let}$-binding} \\
    \bnfor  & \withhandle{\e} \KK & \text{Handling} \\
    \bnfor  & \ttlam{\x}{\T}{\KK} & \text{Abstraction} \\[5pt]
%    \bnfor  & \abs{\rgn}{\KK} & \text{abstraction} \\

  \text{Handler $\h$}
  \bnf & \multicolumn{2}{l}{\typicalhandler}\\
  \bnfor & \multicolumn{2}{l}{\finallyhandler}\\[5pt]

  \text{Pattern $\pat$}
  \bnf & \tuple{\X_1,\ldots,\X_n}\\
  \bnfor & \inj{\X}\\
  \bnfor & \c\\
\end{array}
\end{equation*}



\section{Operational semantics}
\label{sec:oper-semant}

Results:
%
\begin{equation*}
  \text{Result $R$}
  \begin{aligned}[t]
    &\bnf   {} && \val \e \\
    &\bnfor {} && \opRes{\D}{\e}{\KK} \\
  \end{aligned}
\end{equation*}
%
Judgments:
%
\begin{align*}
  &\evalto[\G]{C}{R} &&\text{$C$ evaluates to result $R$ in context $\G$} \\
  &\resultok[\G]{R}  &&\text{$R$ is a valid result in context $\G$} \\
  &\eok[\G]{\e} &&\text{$\e$ is a valid expression in context $\G$} \\
  &\cok[\G]{\C} &&\text{$\C$ is a valid computation in context $\G$} \\
  &\kok[\G]{\D}{\K} &&\text{$\K$ is a valid continuation in context $\G$,
                              with its hole inside additional binders $\D$} \\
\end{align*}
%
\paragraph{Generic Computations}
\begin{mathpar}
  \infer[eval-val]
  { }
  { \evalto{\val \e}{\val \e}}

  \infer[eval-app]
  {
    \evalto{\subst{\C}{\X}{\e}}{R}
  }
  { \evalto
    {\ttapp{(\fun{\X}{\C})}{\e}}
    {R}
  }

  \infer[eval-let-val]
  {
    \evalto{\C_1}{\val \e}
    \\
    \evalto{\subst{\C_2}{\X}{\e}}{R}
  }
  { \evalto
    {\letin{\X = \C_1} \C_2}
    {R}
  }

  \infer[eval-let-op]
  {
    \evalto
    {\C_1}
    {\opRes{\D}{\e}{\KK}}
  }
  { \evalto
    {\letin{\X = \C_1} \C_2}
    {\opRes{\D}{\e}{\letin{\X = \KK} \C_2}}
  }

  \infer[eval-kapp]
  {
    \evalto[\G,\D]{\KK[\hole:=\e]}{\R}
  }
  { \evalto[\G,\D]
    {\kapp{\cont{\KK}}{\e}}
    {\R}
  }

  \infer[eval-match-tuple]
  {
    P_j = \tuple{\X_1,\ldots,\X_m}\\
    \evalto{\substs{\C_j}{\e_1/\X_1,\ldots,\e_m/\X_m}}{\R}
  }
  {
    \evalto{\genericmatch[{\generictuple[m]}]}
           {\R}
  }

  \infer[eval-match-inj]
  {
    P_j = \inj[k]{X}\\
    \evalto{\subst{\C_j}{\X}{\e}}{\R}
  }
  {
    \evalto{\genericmatch[{\inj[k]{\e}}]}
           {\R}
  }

  \infer[eval-match-const]
  {
    P_j = \c\\
    \evalto{\C_j}{\R}
  }
  {
    \evalto{\genericmatch[\c]}
           {\R}
  }
\end{mathpar}


\paragraph{Operations and Handlers}
\begin{mathpar}
  \infer[eval-op]
  {
  }
  {
    \evalto
    {\opOp{\e}}
    {\opRes{\ctxempty}{\e}{\hole}}
  }

  \infer[eval-handle-val]
  {
    \evalto{\C}{\val \e}
    \\
    \evalto{\subst{\C_v}{\X}{\e}}{\R}
  }
  { \evalto
    {\withhandle{\bigl(\typicalhandler\bigr)}{\C}}
    {\R}
  }

\infer[eval-handle-op-val]
  {
    \h = \typicalhandler
    \\\\
    \evalto{\C}{\opRes{\D}{\e}{\KK_1}}
    \\\\
    \evalto[\G,\D]
    {\substs{\C_i}{\e/\X, \cont{\withhandle{\h}{\KK_1}}/\K}}
    {\val \e'}
    \\\\
    \eok{\e'}
  }
  { \evalto
    {\withhandle{\h}{\C}}
    {\val \e'}
  }

\infer[eval-handle-finally]
  {
    \h = \finallyhandler\\
    \h' = \typicalhandler\\
    \evalto{(\letin{\X_f = \withhandle{\h'}{\C}} {\C_f})}{\R}
  }
  {
    \evalto
    {\withhandle{\h}{\C}}
    {\R}
  }


  \infer[eval-handle-op-op]
  {
    \h = \typicalhandler
    \\\\
    \evalto{\C}{\opRes{\D}{\e}{\KK_1}}
    \\\\
    \evalto[\G,\D]
    {\substs{\C_i}{\e/\X, \cont{\withhandle{\h}{\KK_1}}/\K}}
    {\opRes[j]{\D'}{\e'}{\KK_2}}
  }
  { \evalto
    {\withhandle{\h}{\C}}
    {\opRes[j]{(\D,\D')}{\e'}{\KK_2}}
  }
\end{mathpar}

\paragraph{Built-In Functions}
\begin{mathpar}
  \infer[eval-prim]
  {
    \e = \prim[{[\![f]\!]}]{\e_1,\ldots,\e_n}\\
    \bigl[\eok{\e}\bigr]
  }
  {
    \evalto{\genericprim}{\val \e}
  }

  \infer[eval-ascribe]
  {
    \typeof{\Gamma}{\B} = \U\\
    \evalto
      {\left(\begin{array}{l}
       \letin{\X_1 = \eqtypealgOp{\T}{\U}}\\
       \quad \assertEquivType{\T}{\U}{\X_1}\\
       \qquad \val \bterm{\ascribe{\B}{\T}}\\
       \end{array}\right)}
      {\R}
  }
  {
    \evalto{\ascribe{\bterm{\B}}{\bty\T}}{\R}
  }

  \infer[eval-make-var]
  {
    \G = x_{m{-}1}{:}\T_{m{-}1}\ldots,x_0{:}T_0\\
    n < m\\
  }
  {
    \evalto
    {\debruijn{n}}
    {\val \bterm{\x_n}}
  }

  \infer[eval-make-lambda-val]
  {
    \evalto[\ctxextend{\G}{\x}{\T}]
      {\C}
      {\val \bterm{\B}}\\
    \typeof{\Gamma}{\B} = U\\
  }
  { \evalto
    {\ttlam{\x}{\T} \C}
    {\val \ \bterm{\lam{\x}{\T}{\U} B}}
  }

  % For hint abstraction
  \infer[eval-make-lambda-val-tuple]
  {
    \evalto[\ctxextend{\G}{\x}{\T}]
      {\C}
      {\val \tuple{\bterm{\B_1},\ldots,\bterm{\B_n}}}\\
    (\U_i = \typeof{\Gamma}{\B_i})_{i=1}^n\\
  }
  { \evalto
    {\ttlam{\x}{\T} \C}
    {\val \tuple{\bterm{\lam{\x}{\T}{\U_1} B_1},\ldots,\bterm{\lam{\x}{\T}{\U_n} B_n}}}
  }

  \infer[eval-make-lambda-op]
  {
    \evalto[\ctxextend{\G}{\x}{\T}]
    {\C}
    {\opRes{\D}{\e}{\KK}}
  }
  { \evalto
    {\ttlam{\x}{\T} \C}
    {\opRes{(\x{:}\T,\Delta)}{\e}{\ttlam{\x}{\T}{\KK}}}
  }

  \infer[eval-assert-type]
  {
    \ws = \tuple{\bterm{\B_1},\ldots,\bterm{\B_n}}\\
    (\JuEqual{\U_i}{\e'_i}{\e''_i} = \typeof{\G}{\B_i})_{i=1}^n\\
    \eqtypealg[]{\ctxs{\G}{\eqhint{\e'_i}{\e''_i}_{i=1}^n}}{\T_1}{\T_2}\\
    \evalto{\C}{\R}\\
  }
  {
    \evalto
    {\assertEquivType{\T_1}{\T_2}{\ws}{\C}}
    {\R}
  }

\end{mathpar}


\begin{mathpar}
  \infer[eval-make-app]
  {
    \evalto
      {\left(\begin{array}{l}
          \typeof{\Gamma}{\B_1} = \Prod{\x}{\T_1}{\T_2}\\
          \quad \typeof{\Gamma}{\B_2} = T_3\\
          \qquad \letin{\X = \mathsf{equivTy}(\bty{\T_1},\bty{\T_3})}\\
          \qqquad \assertEquivType{\T_1}{\T_3}{\X}\\
          \qqqquad \bterm{\equationsin{\ws}{\app{B_1}{\x}{\T_1}{\T_2}{B_2}}}
       \end{array}\right)}
     {\R}
  }
  { \evalto
      {\makeApp{\bterm{\B_1}}{\bterm{B_2}}}
      {\R}
  }
\end{mathpar}



%This last rule isn't quite right, because if, for example, $\T_1$ and $\T_3$ are Pi types, it's doubtful that Brazil will think
%to look for two names $\B''_1\equiv \B''_3$ in the hint database such that $\T_1 = \El{\gamma}{\B''_1}$ and $\T_3 = \El{\gamma}{\B''_3}$

\subsection*{Type Equivalence Search}

\paragraph{General Type equality}

\begin{mathpar}
  \infer[\rulename{chk-tyeq-reflexivity}]
  { }
  { \evalto{\eqtypealgOp{\T}{\T}}{\val ()} }

  \infer[\rulename{chk-tyeq-hnf}]
  { \whnfs{\G}{\T}{\T'}\\
    \whnfs{\G}{\U}{\U'}\\\\
    \evalto{\eqtypepathOp{\T'}{\U'}}{\R}
  }
  {
    \evalto{\eqtypealgOp{\T}{\U}}{\R}
  }

\end{mathpar}

\paragraph{Equality of head-normal forms}
\begin{mathpar}
  \infer[\rulename{chk-tyeq-path-reflexivity}]
  { }
  { \evalto{\eqtypepathOp{\T}{\T}}{\val ()} }

  \infer[\rulename{chk-tyeq-el}]
  { \alpha = \beta \\
    \evalto{\eqtermalgOp{\e_1}{\Universe{\alpha}}{\e_2}{\Universe{\alpha}}{\Universe{\alpha}}}{\R}
  }
  {
    \evalto{\eqtypepathOp{\El{\alpha}{\e_1}}{\El{\beta}{\e_2}}}{\R}
  }

  \infer[\rulename{chk-tyeq-prod}]
  %{ \eqtypealg[\ws_1]{\GH}{\T_1}{\U_1} \\
    %\eqtypealg[\ws_2]{\ctxs{(\ctxextend{\G}{\x}{\T_1})}{\H}}{\T_2}{\U_2}
  %}
  { \evalto{\left(\begin{array}{l}
                \letin{\X_1 = \eqtypealgOp{\T_1}{\U_1}}\\
                \quad \assertEquivType{\T_1}{\U_1}{\X_1}\\
                \qquad \letin{\X_2 = \ttlam{\x}{\T_1}{\eqtypealgOp{\T_2}{\U_2}}}\\
                \qqquad {\X_1 \pp \X_2}\\
            \end{array}\right)}
           {\R}
  }
  { \evalto{\eqtypepathOp{\Prod{\x}{\T_1}{\T_2}}{\Prod{\x}{\U_1}{\U_2}}}{\R}
  }

  \infer[\rulename{chk-tyeq-paths}]
  {\evalto{\left(\begin{array}{l}
             \letin{\X_1 = \eqtypealgOp{\T}{\U}}\\
             \quad \assertEquivType{\T}{\U}{\X_1}\\
             \qquad \letin{\X_2 = \eqtermalgOp{\e_1}{\e'_1}{\T}}\\
             \qqquad \letin{\X_3 = \eqtermalgOp{\e_2}{\e'_2}{\T}}\\
             \qqqquad \X_1 \pp \X_2 \pp \X_3
           \end{array}\right)}
       {\R}
  }
  {\evalto{\eqtypepathOp{\PrEqual{\T}{\e_1}{\e_2}}
                        {\PrEqual{\U}{\e'_1}{\e'_2}}}
          {\R}}

  \infer[\rulename{chk-tyeq-id}]
  {\evalto{\left(\begin{array}{l}
             \letin{\X_1 = \eqtypealgOp{\T}{\U}}\\
             \quad \assertEquivType{\T}{\U}{\X_1}\\
             \qquad \letin{\X_2 = \eqtermalgOp{\e_1}{\e'_1}{\T}}\\
             \qqquad \letin{\X_3 = \eqtermalgOp{\e_2}{\e'_2}{\T}}\\
             \qqqquad \X_1 \pp \X_2 \pp \X_3
           \end{array}\right)}{\R}
  }
  {\evalto{\eqtypepathOp{\JuEqual{\T}{\e_1}{\e_2}}
                        {\JuEqual{\U}{\e'_1}{\e'_2}}}
          {\R}}
\end{mathpar}
%
The reflexivity rule is not just an optimization, but also handles
equivalence of base types and equivalence of universes.


\subsection{Term equality}
\label{sec:bidirectional-term-equality}

\begin{mathpar}
  \infer[\rulename{chk-eq-refl}]
  { }
  { \evalto{\eqtermalgOp{\e}{\e}{\T}}{\val ()} }

  \infer[\rulename{chk-eq-ext}]
  {
    \whnfs{\G}{\T}{\T'} \\
    \evalto{\eqtermextOp{\e_1}{\e_2}{\T'}}{\R}
  }
  {
    \evalto{\eqtermalgOp{\e_1}{\e_2}{\T}}
           {\R}
  }
\end{mathpar}

\paragraph{Extensionality}

\begin{mathpar}
  \infer[\rulename{chk-eq-ext-prod}]
  {
    \evalto
      {\ttlam{\x}{\T} \!
        \left(\begin{array}{l}
            \letin{\X_1 = \makeApp{\e_1}{\bterm\x}}\\
            \quad \letin{\X_2 = \makeApp{\e_2}{\bterm\x}}\\
            \qquad \eqtermalgOp{\X_1}{\X_2}{\bty{\U}}\\
           \end{array}\right)}
      {\R}
  }
  {
    \evalto
      {\eqtermextOp{\e_1}{\e_2}{\Prod{\x}{\T}{\U}}}
      {\R}
  }
\end{mathpar}

\begin{mathpar}
  \infer[\rulename{chk-eq-ext-unit}]
  {
  }
  {
    \evalto{\eqtermextOp{\e_1}{\e_2}{\Unit}}{\val ()}
  }

  \infer[\rulename{chk-eq-ext-K}]
  {
  }
  {
    \evalto{\eqtermextOp{\e_1}{\e_2}{\JuEqual{\T}{\e_3}{\e_4}}}{\val ()}
  }

  \infer[\rulename{chk-eq-ext-whnf}]
  {
    \whnfs{\G}{\e_1}{\e'_1}\\
    \whnfs{\G}{\e_2}{\e'_2}\\\\
    \evalto{\eqpathOp{\e'_1}{\e'_2}}{\R}
  }
  {
    \evalto{\mathsf{equivExtTerm}(\e_1,\e_2,\e_3)}{\R}
  }
\end{mathpar}
%
In \rulename{chk-eq-ext-whnf}, we might want to check whether $\e'_1$ and $\e'_2$ are the
same expressions before invoking the general comparison function.

\renewcommand{\GH}{\G}
\paragraph{Whnf equivalence}
\begin{mathpar}
  \infer[\rulename{chk-eq-whnf-reflexivity}]
  {
  }
  {
    \evalto{\eqpathOp{\B}{\B}}{\val ()}
  }


  \infer[\rulename{chk-eq-whnf-app}]
  {
    \evalto
      {\left(\begin{array}{l}
        \letin{\X_1 = \eqtypealgOp{\T_1}{\U_1}}\\
        \ \assertEquivType{\T_1}{\T_2}{\X_1}\\
        \ \ \letin{\X_2 = \ttlam{\x}{\T_1}\!
                  \begin{array}[t]{l}
                    \letin{X' = \eqtypealgOp{\T_2}{\U_2}}\\
                    \ \assertEquivType{\T_2}{\U_2}{\X'}\\
                    \ \ \val \X'\\
                  \end{array}}\\
        \ \ \ \letin{\X_3 = \eqpathOp{\B_1}{\B'_1}}\\
        \ \ \ \ \letin{\X_4 = \eqtermalgOp{\B_2}{\B'_2}{\T_1}}\\
        \ \ \ \ \ \X_1 \pp \X_2 \pp \X_3 \pp \X_4
       \end{array}\right)}
      {\R}
  }
  {
    \evalto{\eqpathOp
             {\app{\B_1}{\x}{\T_1}{\T_2}{\B_2}}
             {\app{\B'_1}{\x}{\U_1}{\U_2}{\B'_2}}}
         {\R}
  }
\end{mathpar}

\begin{mathpar}
  \infer[\rulename{chk-eq-whnf-idpath}]
  {
    \evalto
      {\left(\begin{array}{l}
          \letin{\X_1 = \eqtypealgOp{\T}{\U}}\\
          \quad \assertEquivType{\T}{\U}{\X_1}\\
          \qquad \letin{\X_2 = \eqtermalgOp{\B_1}{\B_2}{\T}}\\
          \qqquad \X_1 \pp \X_2\\
        \end{array}\right)}
      {\R}
  }
  {
    \evalto
      {\eqpathOp{\prRefl{\T}{\B_1}}
                {\prRefl{\U}{\B_2}}}
      {\R}
  }

  \infer[\rulename{chk-eq-whnf-j}]
  {
    \evalto
      {\left(\begin{array}{l}
          \letin{\X_1 = \eqtypealgOp{\T}{\T'}}\\
          \assertEquivType{\T}{\T'}{\X_1}\\
          \letin{\X_2 = \ttlam{\x}{\T} \ttlam{\y}{\T} \ttlam{p}{\PrEqual{\T}{x}{y}}}\\
          \qquad\qquad \letin{\X' = \eqtypealgOp{\U}{\U'}}\\
          \qquad\qquad \assertEquivType{\U}{\U'}{\X'}\\
          \qquad\qquad \val \X'\\
          \letin{\X_3 = \ttlam{\z}{\T} \eqtermalgOp{\B_1}{\B'_1}{\substs{P}{z/x, z/y, (\prRefl{\T}{z})/p}}}\\
          \letin{\X_4 = \eqtermalgOp{\B_3}{\B'_3}{\T}}\\
          \letin{\X_5 = \eqtermalgOp{\B_4}{\B'_4}{\T}}\\
          \letin{\X_6 = \eqpathOp{\B_2}{\B'_2}}\\
          \X_1 \pp \X_2 \pp \X_3 \pp \X_4 \pp \X_5 \pp X_6\\
       \end{array}\right)}
      {\R}\\
  }
  {
    \evalto
     {\eqpathOp{\PrElim
                  {\T}
                  {\abst{x\,y\,p}{\U}}
                  {\abst{z}{\B_1}}
                  {\B_2}{\B_3}{\B_4}
               }
               {\PrElim
                  {\T'}
                  {\abst{x\,y\,p}{\U'}}
                  {\abst{z}{\B'_1}}
                  {\B'_2}{\B'_3}{\B'_4}
               }}
     {\R}
  }

  \infer[\rulename{chk-eq-whnf-refl}]
  {
    \evalto
      {\left(\begin{array}{l}
          \letin{\X_1 = \eqtypealgOp{\T}{\U}}\\
          \quad \assertEquivType{\T}{\U}{\X_1}\\
          \qquad \letin{\X_2 = \eqtermalgOp{\B_1}{\B_2}{\T}}\\
          \qqquad \X_1 \pp \X_2\\
        \end{array}\right)}
      {\R}
  }
  {
    \evalto
      {\eqpathOp{\juRefl{\T}{\B_1}}
                {\juRefl{\U}{\B_2}}}
      {\R}
  }
\end{mathpar}

\paragraph{Whnf equivalence of names}

\begin{mathpar}
  \infer[\rulename{chk-eq-whnf-prod}]
  {\alpha = \alpha' \\
   \beta = \beta' \\\\
   \evalto
     {\left(\begin{array}{l}
        \letin{\X_1 = \eqtermalgOp{\B_1}{\B'_1}{\Universe{\alpha}}}\\
        \quad \assertEquivType{\El{\alpha}{\B_1}}{\El{\alpha}{\B'_1}}{\X_1}\\
        \qquad \letin{\X_2 = \ttlam{\x}{\El{\alpha}{\B_1}}\eqtermalgOp{\e_2}{\B'_2}{\Universe{\beta}}} \\
      \end{array}\right)}
     {\R}
  }
  {
    \evalto
      {\eqpathOp
          {\nProd{\alpha}{\beta}{\x}{\B_1}{\B_2}}
          {\nProd{\alpha'}{\beta'}{\x}{\B'_1}{\B'_2}}}
      {\R}
  }

\infer[\rulename{chk-eq-whnf-universe}]
  {
    \alpha = \beta \\
  }
  {
    \evalto
      {\eqpathOp{\nUniverse{\alpha}}{\nUniverse{\beta}}}
      {\val ()}
  }
%% Subsumed by reflexivity rule
  %\infer[\rulename{chk-eq-whnf-unit}]
  %{
  %}
  %{
    %\eqpath{\GH}{\nUnit}{\nUnit}{\Universe{\zero}}
  %}

\infer[\rulename{chk-eq-whnf-paths}]
  {
    \alpha = \alpha'\\
    \evalto
      {\left(\begin{array}{l}
          \letin{\X_1 = \eqtermalgOp{\B_1}{\B'_1}{\Universe{\alpha}}}\\
          \quad \assertEquivType{\El{\alpha}{\B_1}}{\El{\alpha}{\B'_1}}{\X_1}\\
          \qquad \letin{\X_2 = \eqtermalgOp{\B_2}{\B'_2}{\El{\alpha}{\B_1}}}\\
          \qqquad \letin{\X_3 = \eqtermalgOp{\B_3}{\B'_3}{\El{\alpha}{\B_1}}}\\
          \qqqquad \X_1 \pp \X_2 \pp \X_3\\
       \end{array}\right)}
      {\R}
  }
  {
    \evalto
      {\eqpathOp{\nPrEqual{\alpha}{\B_1}{\B_2}{\B_3}}{\nPrEqual{\alpha'}{\B'_1}{\B'_2}{\B'_3}}}
      {\R}
    }

  \infer[\rulename{chk-eq-whnf-id}]
  {
    \alpha = \alpha'\\
    \evalto
      {\left(\begin{array}{l}
          \letin{\X_1 = \eqtermalgOp{\B_1}{\B'_1}{\Universe{\alpha}}}\\
          \quad \assertEquivType{\El{\alpha}{\B_1}}{\El{\alpha}{\B'_1}}{\X_1}\\
          \qquad \letin{\X_2 = \eqtermalgOp{\B_2}{\B'_2}{\El{\alpha}{\B_1}}}\\
          \qqquad \letin{\X_3 = \eqtermalgOp{\B_3}{\B'_3}{\El{\alpha}{\B_1}}}\\
          \qqqquad \val (\X_1 \pp \X_2 \pp \X_3)\\
       \end{array}\right)}
      {\R}
  }
  {
    \evalto
      {\eqpathOp{\nJuEqual{\alpha}{\B_1}{\B_2}{\B_3}}{\nJuEqual{\alpha'}{\B'_1}{\B'_2}{\B'_3}}}
      {\R}
  }


  \infer[\rulename{chk-eq-whnf-coerce}]
  {
    \alpha = \alpha'\\
    \beta = \beta'\\
    \evalto{\eqtermalgOp{\B_1}{\B'_1}{\Universe{\alpha}}}{\R} \\
  }
  {
    \evalto
      {\eqpathOp{\coerce{\alpha}{\beta}{\B_1}}
                {\coerce{\alpha'}{\beta'}{\B'_1}}}
      {\R}
  }

\end{mathpar}


\section{Well-Formedness}

\subsection*{Expressions}

\begin{mathpar}

  \infer[ok-brazil-term]
  {
    \isterm{\G}{\B}{\T}
  }
  {
    \eok{\bterm{\B}}
  }


  \infer[ok-brazil-type]
  {
    \istype{\G}{\T}
  }
  {
    \eok{\bty{\T}}
  }

  \infer[ok-tt-var]
  {
  }
  {
    \eok{\X}
  }

  \infer[ok-cont]
  {
    \kok{\D}{\KK}
  }
  {
    \eok{\cont{\KK}}
  }

  \infer[ok-fun]
  {
    \cok{\C}
  }
  {
    \eok{\fun{\X} \C}
  }

  \infer[ok-const]
  {
  }
  {
    \eok{\c}
  }

  \infer[ok-tuple]
  {
    \eok{\e_1} \\
    \cdots \\
    \eok{\e_n} \\
  }
  {
    \eok{\generictuple}
  }

  \infer[ok-inj]
  {
    \eok{\e}
  }
  {
    \eok{\inj{\e}}
  }

  \infer[ok-handler]
  {
    \cok{\C_v} \\
    \cok{\C_1} \\
    \cdots \\
    \cok{\C_n}
  }
  {
    \eok{\typicalhandler}
  }

\end{mathpar}



\subsection*{Results}

\begin{mathpar}

  \infer[ok-result-val]
  {
    \eok{\e}
  }
  {
    \resultok{\val{\e}}
  }

  \infer[ok-result-op]
  {
    \istype{\G,\D}{\T} \\
    \kok{\D}{\K} \\
  }
  {
    \resultok{\opRes{\D}{\T}{\K}}
  }

\end{mathpar}

\subsection*{Continuations}

\begin{mathpar}

  \infer[ok-hole]
  {
  }
  {
    \kok[\G]{\ctxempty}{\hole}
  }

  \infer[ok-let-cont]
  {
    \kok{\D}{\K_1} \\
    \cok{\C_2} \\
  }
  {
    \kok{\D}{(\letin{\X = \K_1} \C_2)}
  }

  \\

  \infer[ok-handle-cont]
  {
    \eok{\e_1} \\
    \kok{\D}{\K_2} \\
  }
  {
    \kok{\D}{(\withhandle{\e_1} \K_2)}
  }

  \infer[ok-lam-cont]
  {
    \kok[\ctxextend{\x}{\T_1}]{\D}{\K_2}
  }
  {
    \kok{(\ctxextend[\D]{\x}{\T_1})}{(\ttlam{\x}{\T_1} \K_2)}
  }
\end{mathpar}

\subsection*{Computations}

\begin{mathpar}

  \infer[ok-val]
  {
    \eok{\e}
  }
  {
    \cok{\val{\e}}
  }

  \infer[ok-app]
  {
    \eok{\e_1} \\
    \eok{\e_2} \\
  }
  {
    \cok{\ttapp{\e_1}{\e_2}}
  }

  \infer[ok-let]
  {
    \cok{\C_1} \\
    \cok{\C_2} \\
  }
  {
    \cok{(\letin{\X = \C_1} \C_2)}
  }

  \infer[ok-op]
  {
    \eok{\e} \\
  }
  {
    \cok{\opOp{\e}}
  }


  \infer[ok-handle]
  {
    \eok{\e_1} \\
    \cok{\C_2} \\
  }
  {
    \cok{(\withhandle{\e_1} \C_2)}
  }

  \infer[ok-cont-app]
  {
    \eok{\e_1}\\
    \eok{\e_2}
  }
  {
    \cok{\kapp{\e_1}{\e_2}}
  }

  \infer[ok-ascribe]
  {
    \eok{\e_1}\\
    \eok{\e_2}\\
  }
  {
    \cok{\ascribe{\e_1}{\e_2}}
  }

  \infer[ok-prim]
  {
    \eok{\e_1}\\
    \cdots
    \eok{\e_n}
  }
  {
    \cok{\prim{\e_1,\ldots,\e_n}}
  }

  \infer[ok-match]
  {
    \eok{\e}\\
    \cok{\C_1} \\ \cdots\\ \cok{\C_n}
  }
  {
    \cok{\genericmatch}
  }

  \infer[ok-make-var]
  {
  }
  {
    \cok{(\debruijn{n})}
  }

  \infer[ok-make-app]
  {
    \eok{\e_1} \\
    \eok{\e_2} \\
  }
  {
    \cok{\makeApp{\e_1}{\e_2}}
  }

  \infer[ok-make-lam1]
  {
    \eok{\e_1} \\
    \cok{\C_2} \\
  }
  {
    \cok{(\ttlam{\x}{\e_1} \C_2)}
  }

  \infer[ok-make-lam2]
  {
    \eok{\e_1} \\
    \cok[\ctxextend{\x}{\T_1}]{\C_2} \\
  }
  {
    \cok{(\ttlam{\x}{\bty{\T_1}} \C_2)}
  }

  \infer[eval-assert-type]
  {
    \eok{\e}\\\\
    \text{if}\ (\eqtype{\G}{\T_1}{\T_2})\ \text{then}\ (\cok{\C})\\
  }
  {
    \cok{\assertEquivType{\T_1}{\T_2}{\e}{\C}}
  }

\end{mathpar}



\begin{lemma}[Substitution]
  \mbox{}
  \begin{enumerate}
    \item If $\eok{\e}$ and $\eok{\e'}$ then $\eok{\subst{\e}{\X}{\e'}}$.
    \item If $\cok{\C}$ and $\eok{\e'}$ then $\cok{\subst{\C}{\X}{\e'}}$.
    \item If $\kok{\D}{\KK}$ and $\eok{\e'}$ then $\kok{\D}{\subst{\KK}{\X}{\e'}}$.
  \end{enumerate}
\end{lemma}

\begin{lemma}[Continuation Invocation]
  If $\kok{\D}{\KK}$ and $\eok[\G,\D]{\e}$ then $\cok{\KK[\hole:=\e]}$.
\end{lemma}

\begin{lemma}[Preservation]
   If $\isctx{\G}$ and $\cok{\C}$ and $\evalto{\C}{\R}$ then $\resultok{\R}$.
\end{lemma}


\end{document}
