\documentclass{article}

\usepackage{times}
\usepackage{mathpartir}
\usepackage{amsmath,amsfonts,amssymb}
\usepackage{xcolor}

\newtheorem{lemma}{Lemma}

%%%%%% Macros here %%%%%%

%contexts
%
\newcommand{\emptyG}{\bullet} % empty context
\newcommand{\G}{\Gamma}
\newcommand{\D}{\Delta}
\newcommand{\emptyH}{\circ} % empty handlers

%% Syntax
\newcommand{\bnf}{\mathrel{{:}{:}{=}}}
\newcommand{\bnfor}{|}

\newcommand{\x}{x}     % brazil variable
\newcommand{\y}{y}     % another variable
\newcommand{\z}{z}     % another variable
\newcommand{\C}{C}     % computation
\newcommand{\K}{K}     % continuation variable
\newcommand{\X}{X}     % TT (computation) variable
\newcommand{\e}{e}     % expression
\newcommand{\rgn}{r}   % region
\newcommand{\h}{h}     % handler
\newcommand{\T}{T}     % type
\newcommand{\U}{U}     % another type
\newcommand{\B}{B}     % Brazil term
\newcommand{\R}{R}     % Result
\newcommand{\KK}{{\cal K}} % continuation term with hole


\newcommand{\val}{\mathsf{val}\,} % val e
\newcommand{\letin}[1]{\mathsf{let}\; #1 \;\mathsf{in}\;} % let x = c1 in c2
\newcommand{\op}[3]{\mathsf{op}_{#1}(#2, #3)} % operation
\newcommand{\inhabitPat}[2]{\mathsf{inhabit}(#1, #2)} % the inhabit pattern
\newcommand{\inhabitOp}[1]{\mathsf{inhabit}\,#1} % the inhabit operation
\newcommand{\inhabitRes}[3]{\mathsf{inhabit}(#1, #2, #3)} % the inhabit result
\newcommand{\opPat}[3][i]{\mathsf{op}_{#1}(#2, #3)} % the operation pattern
\newcommand{\opOp}[2][i]{\mathsf{op}_{#1}\,#2} % the operation operation
\newcommand{\opRes}[4][i]{\mathsf{op}_{#1}(#2, #3, #4)} % the operation result
\newcommand{\withhandle}[1]{\mathsf{with}\;#1\;\mathsf{handle}\;} % handle
\newcommand{\abs}[1]{\mathsf{abs}\;#1\;\mathsf{in}\;} % abstraction
\newcommand{\new}[2]{\mathsf{new}(#1,#2)} % new(r,T)
\newcommand{\fun}[1]{\mathsf{fun}\;#1\Rightarrow} % tt-level function
\newcommand{\app}[2]{#1\,#2} % application
\newcommand{\lam}[2]{\lambda #1 \,{:}\, #2 \,.\,} % Kind of Brazilian lambda
\newcommand{\kapp}[2]{#1{}[#2]} % continuation application
\newcommand{\bterm}[2]{\langle\,#1\, :\, #2\,\rangle} % brazil (typed) term as data
\newcommand{\bty}[1]{\langle\,#1\,\rangle} % brazil type as data

\newcommand{\handler}[6][n]{\mathsf{handler}\; \val #2 \mapsto #3 \mid (\opPat{#4}{#5} \mapsto #6_{#1})_{i=1}^{#1}}

\newcommand{\makeApp}[2]{\mathsf{app}(#1,#2)} % introduce a Brazil application

\newcommand{\debruijn}[1]{\mathsf{debruijn}\,#1} % ugly hack

\newcommand{\subst}[3]{#1[#3/#2]} % substitution
\newcommand{\substs}[2]{#1[#2]} % substitution of many things

\newcommand{\cont}[2][\G,\D]{\mathsf{cont}(#1,#2)}     % continuation
\newcommand{\hfcont}{\hat{\cal K}}     % handler-free continuation
\newcommand{\hole}{\diamond}
\newcommand{\tuple}[1]{(#1)}
\newcommand{\generictuple}[1][n]{\tuple{e_1,\ldots,\e_{#1}}}


\newcommand{\pat}{P}
\newcommand{\match}[2]{\mathsf{match}\; #1\; \mathsf{with}\;#2}
\newcommand{\genericPats}[1][n]{(\pat_i \Rightarrow \C_i)_{i=1}^n}
\newcommand{\genericmatch}[1][\e]{\match{#1}{\genericPats}}

\renewcommand{\c}{c} % constant
\newcommand{\prim}[2][f]{#1(#2)} % primitive  application
\newcommand{\genericprim}{\prim{\e_1,\ldots,\e_n}}

\newcommand{\inj}[2][i]{\mathsf{inj}_{#1}\, #2}

%%% Operational semantics

\newcommand{\xcT}{\vec{\x}{:}\vec{\T}}
\newcommand{\xT}{{\vec{\x}}{\vec{\T}}}
\renewcommand{\H}{\mathcal{H}} % A stack of handlers
\newcommand{\GH}{\G, \H} % combined context & handler

\newcommand{\ctxempty}{\bullet}
\newcommand{\ctxextend}[3][\G]{#1, #2 {:} #3}

\newcommand{\evalto}[3][\G]{#1 \vdash #2 \ \Downarrow\  #3}

\newcommand{\resultok}[2][\G]{#1 \vdash #2 \ \mathsf{ok}}
\newcommand{\eok}[2][\G]{#1 \vdash #2 \ \ \mathsf{ok}}
\newcommand{\cok}[2][\G]{#1 \vdash #2 \ \ \mathsf{ok}}
\newcommand{\kok}[3][\G]{#1 \vdash #3 \ : \ #2\to\mathsf{ok}}

\newcommand{\istype}[2]{#1 \vdash #2\;\mathsf{type}} % well formed type
\newcommand{\isterm}[3]{#1 \vdash\,#2\,:\,#3} % well formed term

\newcommand{\typeOf}{\mathsf{typeOf}\,}

\newcommand{\typicalhandler}{\handler{\X}{\C_v}{\X}{\K}{\C}}

\renewcommand{\S}{{\cal S}}
\newcommand{\GS}{\Gamma; \S}
\newcommand{\emptystack}{\bullet}


%% Copied from brazil.tex
\newcommand{\Prod}[2]{\mathop{\textstyle\prod_{(#1 {:} #2)}}} % dependent product
\newcommand{\brazilapp}[5]{#1\mathbin{@^{#2{:}#3.#4}} #5} % application
\newcommand{\equationin}[3]{\mathsf{equation}\; #1 \,{:}\, #2 \,{\equiv}\, #3 \; \mathsf{in} \;} % use equation hint
\newcommand{\rewritein}[3]{\mathsf{rewrite}\; #1 \,{:}\, #2 \,{\equiv}\, #3 \; \mathsf{in} \;} % use rewrite hint
\newcommand{\piClose}[2]{\mathsf{max}(#1,#2)}   % the universe containing a pi with given domain and codomain
\newcommand{\uClose}[1]{\mathsf{succ}(#1)}  % the universe containing the given universe's name.
\newcommand{\Universe}[1]{\mathbb{U}_{#1}}
\newcommand{\nJuEqual}[4]{\mathsf{id}^{#1}_{#2}(#3,#4)} % Judgmental Equality type
\newcommand{\coerce}[3]{\mathsf{coerce}^{#1{\mapsto}#2} \, #3}
\newcommand{\El}[2]{\mathsf{El}^{#1}\, #2} % the type named by the term
\newcommand{\ascribe}[2]{#1 \,{:}{:}\, #2} % type ascription





\begin{document}

\title{TT}
\author{Andrej Bauer \and Matija Pretnar \and Christopher A. Stone}
\maketitle

\section{Abstract syntax}
\label{sec:abstract-syntax}

\begin{equation*}
  \begin{array}{rl@{\qquad}l}
  \text{Expression $\e$}
    \bnf    & \X          & \text{Variable} \\
    \bnfor  & \fun{\X} \C  & \text{Function} \\
    \bnfor  & h           & \text{Handler} \\
    \bnfor  & \cont{\KK} & \text{Continuation value} \\
    \bnfor  & \bterm{\B}{\T}           & \text{Brazilian term} \\
    \bnfor  & \bty{\T}           & \text{Brazilian type} \\
    \bnfor  & \generictuple   & \text{Tuple}\\
    \bnfor  & \c              & \text{TT Constant}\\
    \bnfor  & \inj{\e}        & \text{Coproduct}\\
    \\
    \text{Computation $\C$}
      \bnf  & \val \e                & \text{Pure expression} \\
    \bnfor  & \app{\e_1}{\e_2}   & \text{Application} \\
    \bnfor  & \letin{\X = \C_1} \C_2  & \text{$\mathsf{let}$-binding} \\
    \bnfor  & \opOp{\e} & \text{Operation} \\
    \bnfor  & \withhandle{\e} \C & \text{Handling} \\
%    \bnfor  & \abs{\rgn}{\C} & \text{abstraction} \\
%    \bnfor  & \new{\rgn}{\T} & \text{new variable} \\
    \bnfor  & \kapp{\e_1}{\e_2}   & \text{Invoke a continuation} \\
    \bnfor  & \ascribe{\e_1}{\e_2} & \text{Type ascription} \\
    \bnfor  & \genericprim & \text{Primitive operations}\\
    \bnfor  & \genericmatch& \text{Pattern-match}\\
    \bnfor  & \debruijn{n} & \text{Build Brazilian term: variable} \\
    \bnfor  & \lam{\x}{\e} \C   & \text{Build Brazilian term: abstraction} \\
    \bnfor  & \makeApp{\e_1}{\e_2} & \text{Build Brazilian term: application} \\
    \\
    \text{Continuation $\KK$} \bnf    & \hole                & \text{Hole} \\
    \bnfor  & \letin{\X = \KK} \C_2  & \text{$\mathsf{let}$-binding} \\
    \bnfor  & \withhandle{\e} \KK & \text{Handling} \\
    \bnfor  & \lam{\x}{\T}{\KK} & \text{Abstraction} \\
%    \bnfor  & \abs{\rgn}{\KK} & \text{abstraction} \\
    \\

  \text{Handler $\h$}
  \bnf & \multicolumn{2}{l}{\typicalhandler}\\
  \\
  \text{Pattern $\pat$}
  \bnf & \tuple{\X_1,\ldots,\X_n}\\
  \bnfor & \inj{\X}\\
  \bnfor & \c\\
\end{array}
\end{equation*}



\section{Operational semantics}
\label{sec:oper-semant}

Results:
%
\begin{equation*}
  \text{Result $R$}
  \begin{aligned}[t]
    &\bnf   {} && \val \e \\
    &\bnfor {} && \opRes{\D}{\e}{\KK} \\
  \end{aligned}
\end{equation*}
%
Judgments:
%
\begin{align*}
  &\evalto[\G]{C}{R} &&\text{$C$ evaluates to result $R$ in context $\G$} \\
  &\resultok[\G]{R}  &&\text{$R$ is a valid result in context $\G$} \\
  &\eok[\G]{\e} &&\text{$\e$ is a valid expression in context $\G$} \\
  &\cok[\G]{\C} &&\text{$\C$ is a valid computation in context $\G$} \\
  &\kok[\G]{\D}{\K} &&\text{$\K$ is a valid continuation in context $\G$,
                              with its hole inside additional binders $\D$} \\
\end{align*}
%
\paragraph{Generic Computations}
\begin{mathpar}
  \infer[eval-val]
  { }
  { \evalto{\val \e}{\val \e}}

  \infer[eval-app]
  {
    \evalto{\subst{\C}{\X}{\e}}{R}
  }
  { \evalto
    {\app{(\fun{\X}{\C})}{\e}}
    {R}
  }

  \infer[eval-let-val]
  {
    \evalto{\C_1}{\val \e}
    \\
    \evalto{\subst{\C_2}{\X}{\e}}{R}
  }
  { \evalto
    {\letin{\X = \C_1} \C_2}
    {R}
  }

  \infer[eval-let-op]
  {
    \evalto
    {\C_1}
    {\opRes{\D}{\e}{\KK}}
  }
  { \evalto
    {\letin{\X = \C_1} \C_2}
    {\opRes{\D}{\e}{\letin{\X = \KK} \C_2}}
  }

  \infer[eval-kapp]
  {
    \evalto[\G,\D]{\KK[\hole:=\e]}{\R}
  }
  { \evalto[\G,\D]
    {\kapp{\cont{\KK}}{\e}}
    {\R}
  }

  \infer[eval-match-tuple]
  {
    P_j = \tuple{\X_1,\ldots,\X_m}\\
    \evalto{\substs{\C_j}{\e_1/\X_1,\ldots,\e_m/\X_m}}{\R}
  }
  {
    \evalto{\genericmatch[{\generictuple[m]}]}
           {\R}
  }

  \infer[eval-match-inj]
  {
    P_j = \inj[k]{X}\\
    \evalto{\subst{\C_j}{\X}{\e}}{\R}
  }
  {
    \evalto{\genericmatch[{\inj[k]{\e}}]}
           {\R}
  }

  \infer[eval-match-const]
  {
    P_j = \c\\
    \evalto{\C_j}{\R}
  }
  {
    \evalto{\genericmatch[\c]}
           {\R}
  }
\end{mathpar}


\paragraph{Operations and Handlers}
\begin{mathpar}
  \infer[eval-op]
  {
  }
  {
    \evalto
    {\opOp{\e}}
    {\opRes{\ctxempty}{\e}{\hole}}
  }

  \infer[eval-handle-val]
  {
    \evalto{\C}{\val \e}
    \\
    \evalto{\subst{\C_v}{\X}{\e}}{\R}
  }
  { \evalto
    {\withhandle{\bigl(\typicalhandler\bigr)}{\C}}
    {\R}
  }

\infer[eval-handle-op-val]
  {
    \h = \typicalhandler
    \\\\
    \evalto{\C}{\opRes{\D}{\e}{\KK_1}}
    \\\\
    \evalto[\G,\D]
    {\substs{\C_i}{\e/\X, \cont{\withhandle{\h}{\KK_1}}/\K}}
    {\val \e}
    \\\\
    \eok{\e}
  }
  { \evalto
    {\withhandle{\h}{\C}}
    {\val \e}
  }

  \infer[eval-handle-op-op]
  {
    \h = \typicalhandler
    \\\\
    \evalto{\C}{\opRes{\D}{\e}{\KK_1}}
    \\\\
    \evalto[\G,\D]
    {\substs{\C_i}{\e/\X, \cont{\withhandle{\h}{\KK_1}}/\K}}
    {\opRes[j]{\D'}{\e'}{\KK_2}}
  }
  { \evalto
    {\withhandle{\h}{\C}}
    {\opRes[j]{(\D,\D')}{\e'}{\KK_2}}
  }
\end{mathpar}

\paragraph{Built-In Functions}
\begin{mathpar}
  \infer[eval-prim]
  {
    \e = \prim[{[\![f]\!]}]{\e_1,\ldots,\e_n}\\
    \bigl[\eok{\e}\bigr]
  }
  {
    \evalto{\genericprim}{\val \e}
  }

  \infer[eval-ascribe]
  {
  }
  {
    \evalto{(\ascribe{\bterm{\B}{\T}}{\T})}{\val \bterm{\B}{\T}}
  }

  \infer[eval-make-var]
  {
    \G = x_{m{-}1}{:}\T_{m{-}1}\ldots,x_0{:}T_0\\
    n < m\\
  }
  {
    \evalto
    {\debruijn{n}}
    {\val \bterm{\x_n}{\T_n}}
  }

  \infer[eval-make-lambda-val]
  {
    \evalto[\ctxextend{\x}{\T}]
    {\C}
    {\val \bterm{\B}{\U}}
  }
  { \evalto
    {\lam{\x}{\T} \C}
    {\val (\lam{\x}{\T} B)}
  }

  \infer[eval-make-lambda-op]
  {
    \evalto[\ctxextend{\x}{\T}]
    {\C}
    {\opRes{\D}{\e}{\KK}}
  }
  { \evalto
    {\lam{\x}{\T} \C}
    {\opRes{(\x{:}\T,\Delta)}{\e}{\lam{\x}{\T}{\KK}}}
  }

  \\

  \infer[eval-make-app]
  {
    \mathsf{nameOf}(\T_1) = B'_1 : \Universe{\alpha}\\
    \mathsf{nameOf}(\T_3) = B'_3 : \Universe{\beta}\\
    \\\\
    \gamma = \piClose{\alpha}{\beta}\\
    \gamma' = \uClose{\gamma}\\
    \\\\
    \B''_1 = \coerce{\alpha}{\gamma}{\B'_1}\\
    \B''_3 = \coerce{\beta}{\gamma}{\B'_3}\\
    \\\\
    \evalto
      {\bigl(\letin{\X = \inhabitOp{\bty{\nJuEqual{\gamma'}{\Universe{\gamma}}{\B''_1}{\B''_3}}}}
         {\val (\equationin{\X}{\B''_1}{\B''_3}{(\brazilapp{B_1}{\x}{\T_1}{\T_2}{B_2})})}\bigr)}
      {R}
  }
  { \evalto
      {\makeApp{\bterm{B_1}{\Prod{\x}{\T_1}{\T_2}}}{\bterm{B_2}{\T_3}}}
      {R}
  }
\end{mathpar}

This last rule isn't quite right, because if, for example, $\T_1$ and $\T_3$ are Pi types, it's doubtful that Brazil will think
to look for two names $\B''_1\equiv \B''_3$ in the hint database such that $\T_1 = \El{\gamma}{\B''_1}$ and $\T_3 = \El{\gamma}{\B''_3}$

\section{Well-Formedness}

\subsection*{Expressions}

\begin{mathpar}

  \infer[ok-brazil-term]
  {
    \isterm{\G}{\B}{\T}
  }
  {
    \eok{\bterm{\B}{\T}}
  }


  \infer[ok-brazil-type]
  {
    \istype{\G}{\T}
  }
  {
    \eok{\bty{\T}}
  }

  \infer[ok-tt-var]
  {
  }
  {
    \eok{\X}
  }

  \infer[ok-cont]
  {
    \kok{\D}{\KK}
  }
  {
    \eok{\cont{\KK}}
  }

  \infer[ok-fun]
  {
    \cok{\C}
  }
  {
    \eok{\fun{\X} \C}
  }

  \infer[ok-const]
  {
  }
  {
    \eok{\c}
  }

  \infer[ok-tuple]
  {
    \eok{\e_1} \\
    \cdots \\
    \eok{\e_n} \\
  }
  {
    \eok{\generictuple}
  }

  \infer[ok-inj]
  {
    \eok{\e}
  }
  {
    \eok{\inj{\e}}
  }

  \infer[ok-handler]
  {
    \cok{\C_v} \\
    \cok{\C_1} \\
    \cdots \\
    \cok{\C_n}
  }
  {
    \eok{\typicalhandler}
  }

\end{mathpar}



\subsection*{Results}

\begin{mathpar}

  \infer[ok-result-val]
  {
    \eok{\e}
  }
  {
    \resultok{\val{\e}}
  }

  \infer[ok-result-op]
  {
    \istype{\G,\D}{\T} \\
    \kok{\D}{\K} \\
  }
  {
    \resultok{\opRes{\D}{\T}{\K}}
  }

\end{mathpar}

\subsection*{Continuations}

\begin{mathpar}

  \infer[ok-hole]
  {
  }
  {
    \kok[\G]{\ctxempty}{\hole}
  }

  \infer[ok-let-cont]
  {
    \kok{\D}{\K_1} \\
    \cok{\C_2} \\
  }
  {
    \kok{\D}{(\letin{\X = \K_1} \C_2)}
  }

  \\

  \infer[ok-handle-cont]
  {
    \eok{\e_1} \\
    \kok{\D}{\K_2} \\
  }
  {
    \kok{\D}{(\withhandle{\e_1} \K_2)}
  }

  \infer[ok-lam-cont]
  {
    \kok[\ctxextend{\x}{\T_1}]{\D}{\K_2}
  }
  {
    \kok{(\ctxextend[\D]{\x}{\T_1})}{(\lam{\x}{\T_1} \K_2)}
  }
\end{mathpar}

\subsection*{Computations}

\begin{mathpar}

  \infer[ok-val]
  {
    \eok{\e}
  }
  {
    \cok{\val{\e}}
  }

  \infer[ok-app]
  {
    \eok{\e_1} \\
    \eok{\e_2} \\
  }
  {
    \cok{\app{\e_1}{\e_2}}
  }

  \infer[ok-let]
  {
    \cok{\C_1} \\
    \cok{\C_2} \\
  }
  {
    \cok{(\letin{\X = \C_1} \C_2)}
  }

  \infer[ok-op]
  {
    \eok{\e} \\
  }
  {
    \cok{\opOp{\e}}
  }


  \infer[ok-handle]
  {
    \eok{\e_1} \\
    \cok{\C_2} \\
  }
  {
    \cok{(\withhandle{\e_1} \C_2)}
  }

  \infer[ok-cont-app]
  {
    \eok{\e_1}\\
    \eok{\e_2}
  }
  {
    \cok{\kapp{\e_1}{\e_2}}
  }

  \infer[ok-ascribe]
  {
    \eok{\e_1}\\
    \eok{\e_2}\\
  }
  {
    \cok{\ascribe{\e_1}{\e_2}}
  }

  \infer[ok-prim]
  {
    \eok{\e_1}\\
    \cdots
    \eok{\e_n}
  }
  {
    \cok{\prim{\e_1,\ldots,\e_n}}
  }

  \infer[ok-match]
  {
    \eok{\e}\\
    \cok{\C_1} \\ \cdots\\ \cok{\C_n}
  }
  {
    \cok{\genericmatch}
  }

  \infer[ok-make-var]
  {
  }
  {
    \cok{(\debruijn{n})}
  }

  \infer[ok-make-app]
  {
    \eok{\e_1} \\
    \eok{\e_2} \\
  }
  {
    \cok{\makeApp{\e_1}{\e_2}}
  }

  \infer[ok-make-lam1]
  {
    \eok{\e_1} \\
    \cok{\C_2} \\
  }
  {
    \cok{(\lam{\x}{\e_1} \C_2)}
  }

  \infer[ok-make-lam2]
  {
    \eok{\e_1} \\
    \cok[\ctxextend{\x}{\T_1}]{\C_2} \\
  }
  {
    \cok{(\lam{\x}{\bty{\T_1}} \C_2)}
  }
\end{mathpar}



\begin{lemma}[Substitution]
  \mbox{}
  \begin{enumerate}
    \item If $\cok{\C}$ and $\eok{\e}$ then $\cok{\subst{\C}{\X}{\e}}$.
    \item If $\kok{\D}{\KK}$ and $\eok{\e}$ then $\kok{\D}{\subst{\KK}{\X}{\e}}$.
  \end{enumerate}
\end{lemma}

\begin{lemma}[Continuation Invocation]
  If $\kok{\D}{\KK}$ and $\eok[\G,\D]{\e}$ then $\cok{\KK[\hole:=\e]}$.
\end{lemma}

\begin{lemma}[Preservation]
   If $\cok{\C}$ and $\evalto{\C}{\R}$ then $\resultok{\R}$.
\end{lemma}


\end{document}
