\section{The declarative formulation with spines}
\label{sec:declarative-formulation}

We now reformulate the classic theory to one that uses spines instead of applications. A
spine is an application of the form $e_1 \, [e_2, \ldots, e_n]$ where $e_1$ is applied to
a non-empty list of arguments $e_2, \ldots, e_n$. An ordinary application then corresponds
to $e_1 \, [e_2]$, while a spine $e_1 \, [e_2, \ldots, e_n]$ corresponds to a nested
application $((e_1 e_2) \cdots e_n)$. Of course, our spines will be tagged with types.

In this sections all rule names are prefixed with \textsc{s-}, which stands for ``spines''.

\subsection{Syntax}
\label{sec:syntax}

Contexts:
%
\begin{equation*}
  \G
  \begin{aligned}[t]
    \bnf   {}& \ctxempty & & \text{empty context}\\
    \bnfor {}& \ctxextend{\G}{\x}{\T} & & \text{context $\G$ extended with $\x : \T$}
  \end{aligned}
\end{equation*}
%
We shall use vector notation to indicate a repetition of syntactic entities. For instance,
we write $\many{e}$ instead of $e_1, \ldots, e_n$ and $\many{x : A}$ instead of
$x_1 : A_1, \ldots, x_n : A_n$. We also write $e_0, \many{e}$ for a sequence
$e_0, e_1, \ldots, e_n$.

Terms ($\e$) and types $(\T, \U)$:
%
\begin{equation*}
  \e, \T, \U
  \begin{aligned}[t]
    \bnf   {}& \x   &&\text{variable} \\
    \bnfor {}& \Type & & \text{universe}\\
    \bnfor {}& \sProd{\x}{\T} \U & & \text{product}\\
    \bnfor {}& \Equal{\T}{\e_1}{\e_2} & & \text{equality type} \\
    \bnfor {}& \slam{\x}{\T}{\U} \e  &&\text{$\lambda$-abstraction} \\
    \bnfor {}& \sapp{\e_1}{\x}{\T}{\U}{\e_2}  &&\text{application} \\
    \bnfor {}& \refl{\T} \e  &&\text{reflexivity} \\
  \end{aligned}
\end{equation*}
%
We translate the products, abstractions and spines back to their classic versions as
follows:
%
\begin{align*}
  \left(\sProd{\x}{\T} \U \right)^{\mathsf{classic}} &=
  \cProd{\x_1}{\T_1} \cdots \cProd{\x_n}{\T_n} \U
  \\
  \left(
    \xlam{(\x_0{:}\T_0, \many{\x{:}\T})}{\U} \e
  \right)^{\mathsf{classic}} &=
  \clam{\x_0}{\T_0}{(\sProd{\x}{\T})^{\mathsf{classic}}}
  \left(\slam{\x}{\T}{\U} \e\right)^{\mathsf{classic}}
  \\
  \left(
    \xapp{\e_0}{(\x_0{:}\T_0, \many{\x{:}\T}).\U}{(\e_1, \many{e_2})}
  \right)^{\mathsf{classic}} &=
\end{align*}


\subsection{Judgments}
\label{sec:judgments}

\begin{align*}
& \isctx{\G} & & \text{$\G$ is a well formed context} \\
& \istype{\G}{\T} & & \text{$T$ is a well formed type in context $\G$} \\
& \isterm{\G}{\e}{\T} & & \text{$\e$ is a well formed term of type $\T$ in context $\G$} \\
& \eqtype{\G}{T_1}{T_2} & & \text{$T_1$ and $T_2$ are equal types in context $\G$} \\
& \eqterm{\G}{\e_1}{\e_2}{\T} & & \text{$e_1$ and $e_2$ are equal terms of type $\T$ in context $\G$}
\end{align*}

\subsection{Contexts}
\label{sec:contexts}

\begin{mathpar}
  \infer[\rulename{s-ctx-empty}]
  { }
  {\isctx{\ctxempty}}

  \infer[\rulename{s-ctx-extend}]
  {\isctx{\G} \\
   \istype{\G}{\T}
  }
  {\isctx{\ctxextend{\G}{\x}{\T}}}
\end{mathpar}

% In rules that extend the context, we leave implicit the premise that the extended context be well formed.
% XXX Is this covered by the term-var rule?

\subsection{Terms and types}

\paragraph{General rules}
\begin{mathpar}
  \infer[\rulename{s-term-conv}]
  {\isterm{\G}{\e}{\T} \\
   \eqtype{\G}{\T}{\U}
  }
  {\isterm{\G}{\e}{\U}}

  \infer[\rulename{s-term-var}]
  {\isctx{\G} \\
   (\x{:}\T) \in \G
  }
  {\isterm{\G}{\x}{\T}}
\end{mathpar}

\paragraph{Universe}

\begin{mathpar}
  \infer[\rulename{s-ty-type}]
  {\isctx{\G}
  }
  {\istype{\G}{\Type}}
\end{mathpar}

\paragraph{Products}

\begin{mathpar}
  \infer[\rulename{s-ty-prod}]
  {\istype{\G}{\T} \\
   \istype{\ctxextend{\G}{\x}{\T}}{\U}
  }
  {\istype{\G}{\sProd{\x}{\T}{\U}}}

  \infer[\rulename{s-term-abs}]
  {\isterm{\ctxextend{\G}{\x}{\T}}{\e}{\U}}
  {\isterm{\G}{(\slam{\x}{\T}{\U}{\e})}{\sProd{\x}{\T}{\U}}}

  \infer[\rulename{s-term-app}]
  {\isterm{\G}{\e_1}{\sProd{x}{\T} \U} \\
   \isterm{\G}{\e_2}{\T}
  }
  {\isterm{\G}{\sapp{\e_1}{\x}{\T}{\U}{\e_2}}{\subst{\U}{\x}{\e_2}}}
\end{mathpar}

\paragraph{Equality types}
\label{sec:equality}

\begin{mathpar}
  \infer[\rulename{s-ty-eq}]
  {\istype{\G}{\T}\\
   \isterm{\G}{\e_1}{\T}\\
   \isterm{\G}{\e_2}{\T}
  }
  {\istype{\G}{\Equal{\T}{\e_1}{\e_2}}}

  \infer[\rulename{s-term-refl}]
  {\isterm{\G}{\e}{\T}}
  {\isterm{\G}{\refl{\T} \e}{\Equal{\T}{\e}{\e}}}
  \end{mathpar}

\subsection{Equality}

\paragraph{General rules}

\begin{mathpar}
  \infer[\rulename{s-eq-refl}]
  {\isterm{\G}{\e}{\T}}
  {\eqterm{\G}{\e}{\e}{\T}}

  \infer[\rulename{s-eq-sym}]
  {\eqterm{\G}{\e_2}{\e_1}{\T}}
  {\eqterm{\G}{\e_1}{\e_2}{\T}}

  \infer[\rulename{s-eq-trans}]
  {\eqterm{\G}{\e_1}{\e_2}{\T}\\
   \eqterm{\G}{\e_2}{\e_3}{\T}}
  {\eqterm{\G}{\e_1}{\e_3}{\T}}

  \infer[\rulename{s-eq-conv}]
  {\eqterm{\G}{\e_1}{\e_2}{\T}\\
    \eqtype{\G}{\T}{\U}}
  {\eqterm{\G}{\e_1}{\e_2}{\U}}
\end{mathpar}

\paragraph{Equality reflection}
%
\begin{mathpar}
  \infer[\rulename{s-eq-reflection}]
  {\isterm{\G}{\e}{\Equal{\T}{\e_1}{\e_2}}}
  {\eqterm{\G}{\e_1}{\e_2}{\T}}
\end{mathpar}

\paragraph{Computations}

\begin{mathpar}
\infer[\rulename{s-prod-beta}]
  {\eqtype{\G}{\T_1}{\U_1}\\
    \eqtype{\ctxextend{\G}{\x}{\T_1}}{\T_2}{\U_2}\\\\
    \isterm{\ctxextend{\G}{\x}{\T_1}}{\e_1}{\T_2}\\
    \isterm{\G}{\e_2}{\U_1}}
  {\eqterm{\G}{\bigl(\sapp{(\slam{\x}{\T_1}{\T_2}{\e_1})}{\x}{\U_1}{\U_2}{\e_2}\bigr)}
              {\subst{\e_1}{\x}{\e_2}}
              {\subst{\T_2}{\x}{\e_2}}}
\end{mathpar}

\paragraph{Extensionality}

%
\begin{mathpar}
  \infer[\rulename{s-eq-eta}]
  {\isterm{\G}{\e'_1}{\Equal{\T}{\e_1}{\e_2}} \\
    \isterm{\G}{\e'_2}{\Equal{\T}{\e_1}{\e_2}}
  }
  {\eqterm{\G}{\e'_1}{e'_2}{\Equal{\T}{\e_1}{\e_2}}}

  \infer[\rulename{s-prod-eta}]
  {\isterm{\G}{\e_1}{\sProd{\x}{\T}{\U}}\\
   \isterm{\G}{\e_2}{\sProd{\x}{\T}{\U}}\\\\
   \eqterm{\ctxextend{\G}{\x}{\T}}{(\sapp{\e_1}{\x}{\T}{\U}{\x})}
          {(\sapp{\e_2}{\x}{\T}{\U}{\x})}{\U}
  }
  {\eqterm{\G}{\e_1}{\e_2}{\sProd{\x}{\T}{\U}}}
\end{mathpar}

\subsubsection{Congruences}

\paragraph{Type formers}

\begin{mathpar}
  \infer[\rulename{s-cong-prod}]
  {\eqtype{\G}{\T_1}{\U_1}\\
   \eqtype{\ctxextend{\G}{\x}{\T_1}}{\T_2}{\U_2}}
  {\eqtype{\G}{\sProd{\x}{\T_1}{\T_2}}{\sProd{\x}{\U_1}{\U_2}}}

  \infer[\rulename{s-cong-eq}]
  {\eqtype{\G}{\T}{\U}\\
   \eqterm{\G}{\e_1}{\e'_1}{\T}\\
   \eqterm{\G}{\e_2}{\e'_2}{\T}
  }
  {\eqtype{\G}{\Equal{\T}{\e_1}{\e_2}}
              {\Equal{\U}{\e'_1}{\e'_2}}}
\end{mathpar}

\paragraph{Products}

\begin{mathpar}
  \infer[\rulename{s-cong-abs}]
  {\eqtype{\G}{\T_1}{\U_1}\\
    \eqtype{\ctxextend{\G}{\x}{\T_1}}{\T_2}{\U_2}\\
    \eqterm{\ctxextend{\G}{\x}{\T_1}}{\e_1}{\e_2}{\T_2}}
  {\eqterm{\G}{(\slam{\x}{\T_1}{\T_2}{\e_1})}
              {(\slam{\x}{\U_1}{\U_2}{\e_2})}
              {\sProd{\x}{\T_1}{\T_2}}}

  \infer[\rulename{s-cong-app}]
  {\eqtype{\G}{\T_1}{\U_1}\\
   \eqtype{\ctxextend{\G}{\x}{\T_1}}{\T_2}{\U_2}\\\\
   \eqterm{\G}{\e_1}{\e'_1}{\sProd{\x}{\T_1}{\T_2}}\\
   \eqterm{\G}{\e_2}{\e'_2}{\T_1}}
  {\eqterm{\G}{(\sapp{\e_1}{\x}{\T_1}{\T_2}{\e_2})}{(\sapp{\e'_1}{\x}{\U_1}{\U_2}{\e'_2})}{\subst{\T_2}{\x}{\e_2}}}
\end{mathpar}

\paragraph{Equality types}

%
\begin{mathpar}
\infer[\rulename{s-cong-refl}]
{\eqterm{\G}{\e_1}{\e_2}{\T}\\
 \eqtype{\G}{\T}{\U}}
{\eqterm{\G}{\refl{\T} \e_1}{\refl{\U} \e_2}{\Equal{\T}{\e_1}{\e_1}}}
\end{mathpar}

%%%%%%%%%%%%%%%%%%%%%%%%%%%%%%%%%%%%%%%%%%%%%%%%%%%%%%%%%

%%% Local Variables:
%%% mode: latex
%%% TeX-master: "theory"
%%% End:
