\section{The classic declarative formulation}
\label{sec:classic-declarative-formulation}

In this section we give the formulation of type theory in a classic declarative way which
minimizes the number of judgments, is better suited for a semantic account, but is not
susceptible to an algorithmic treatment. We shall arrive at the final version in several
steps.

In this sections all rule names are prefixed with \textsc{c-}, which stands for ``classic''.

\subsection{Syntax}
\label{sec:syntax}

Contexts:
%
\begin{equation*}
  \G
  \begin{aligned}[t]
    \bnf   {}& \ctxempty & & \text{empty context}\\
    \bnfor {}& \ctxextend{\G}{\x}{\T} & & \text{context $\G$ extended with $\x : \T$}
  \end{aligned}
\end{equation*}
%
Terms ($\e$) and types $(\T, \U)$:
%
\begin{equation*}
  \e, \T, \U
  \begin{aligned}[t]
    \bnf   {}&  \x   &&\text{variable} \\
    \bnfor {}& \Type & & \text{universe}\\
    \bnfor {}& \cProd{x}{\T} \U & & \text{product}\\
    \bnfor {}& \Equal{\T}{\e_1}{\e_2} & & \text{equality type} \\
    \bnfor {}&  \clam{\x}{\T_1}{\T_2} \e  &&\text{$\lambda$-abstraction} \\
    \bnfor {}&  \capp{\e_1}{\x}{\T_1}{\T_2}{\e_2}  &&\text{application} \\
    \bnfor {}&  \refl{\T} \e  &&\text{reflexivity} \\
  \end{aligned}
\end{equation*}
%
Note that $\lambda$-abstraction and application are tagged with extra types not usually
seen in type theory. An abstraction $\clam{\x}{\T_1}{\T_2} \e$ specifies not only the type
$\T_1$ of $\x$ but also the type $\T_2$ of $e$, where $\x$ is bound in $\T_2$ and $\e$.
Similarly, an application $\capp{\e_1}{\x}{\T_1}{\T_2}{\e_2}$ specifies that $\e_1$ and
$\e_2$ have types $\cProd{\x}{\T_1} \T_2$ and $\T_2$, respectively. This is necessary
because in the presence of exotic equalities (think ``$\mathsf{nat} \to \mathsf{bool}
\equiv \mathsf{nat} \to \mathsf{nat}$'') we must be \emph{very} careful about
$\beta$-reductions.

The annotations on an application matter also for determining when two
terms are equal. For example, if $X,Y : \Type$, $f : \mathsf{nat}\to X$ and $$e
: \Equal{\Type}{\mathsf{nat}{\to}X}{\mathsf{nat}{\to}Y},$$ then
$(\capp{f}{\_}{\mathsf{nat}}{X} 0) : X$ and $(\capp{f}{\_}{\mathsf{nat}}{Y} 0) : Y$,
so without annotations the two terms would be equal and would have different types.

\subsection{Judgments}
\label{sec:judgments}

\begin{align*}
& \isctx{\G} & & \text{$\G$ is a well formed context} \\
& \istype{\G}{\T} & & \text{$T$ is a well formed type in context $\G$} \\
& \isterm{\G}{\e}{\T} & & \text{$\e$ is a well formed term of type $\T$ in context $\G$} \\
& \eqtype{\G}{T_1}{T_2} & & \text{$T_1$ and $T_2$ are equal types in context $\G$} \\
& \eqterm{\G}{\e_1}{\e_2}{\T} & & \text{$e_1$ and $e_2$ are equal terms of type $\T$ in context $\G$}
\end{align*}

\subsection{Contexts}
\label{sec:contexts}

\begin{mathpar}
  \infer[\rulename{c-ctx-empty}]
  { }
  {\isctx{\ctxempty}}

  \infer[\rulename{c-ctx-extend}]
  {\isctx{\G} \\
   \istype{\G}{\T}
  }
  {\isctx{\ctxextend{\G}{\x}{\T}}}
\end{mathpar}

% In rules that extend the context, we leave implicit the premise that the extended context be well formed.
% XXX Is this covered by the term-var rule?

\subsection{Terms and types}

\paragraph{General rules}
\begin{mathpar}
  \infer[\rulename{c-term-conv}]
  {\isterm{\G}{\e}{\T} \\
   \eqtype{\G}{\T}{\U}
  }
  {\isterm{\G}{\e}{\U}}

  \infer[\rulename{c-term-var}]
  {\isctx{\G} \\
   (\x{:}\T) \in \G
  }
  {\isterm{\G}{\x}{\T}}
\end{mathpar}

\paragraph{Universe}

\begin{mathpar}
  \infer[\rulename{c-ty-type}]
  {\isctx{\G}
  }
  {\istype{\G}{\Type}}
\end{mathpar}

\paragraph{Products}

\begin{mathpar}
  \infer[\rulename{c-ty-prod}]
  {\istype{\G}{\T} \\
   \istype{\ctxextend{\G}{\x}{\T}}{\U}
  }
  {\istype{\G}{\cProd{\x}{\T}{\U}}}

  \infer[\rulename{c-term-abs}]
  {\isterm{\ctxextend{\G}{\x}{\T}}{\e}{\U}}
  {\isterm{\G}{(\clam{\x}{\T}{\U}{\e})}{\cProd{\x}{\T}{\U}}}

  \infer[\rulename{c-term-app}]
  {\isterm{\G}{\e_1}{\cProd{x}{\T} \U} \\
   \isterm{\G}{\e_2}{\T}
  }
  {\isterm{\G}{\capp{\e_1}{\x}{\T}{\U}{\e_2}}{\subst{\U}{\x}{\e_2}}}
\end{mathpar}

\paragraph{Equality types}
\label{sec:equality}

\begin{mathpar}
  \infer[\rulename{c-ty-eq}]
  {\istype{\G}{\T}\\
   \isterm{\G}{\e_1}{\T}\\
   \isterm{\G}{\e_2}{\T}
  }
  {\istype{\G}{\Equal{\T}{\e_1}{\e_2}}}

  \infer[\rulename{c-term-refl}]
  {\isterm{\G}{\e}{\T}}
  {\isterm{\G}{\refl{\T} \e}{\Equal{\T}{\e}{\e}}}
  \end{mathpar}

\subsection{Equality}

\paragraph{General rules}

\begin{mathpar}
  \infer[\rulename{c-eq-refl}]
  {\isterm{\G}{\e}{\T}}
  {\eqterm{\G}{\e}{\e}{\T}}

  \infer[\rulename{c-eq-sym}]
  {\eqterm{\G}{\e_2}{\e_1}{\T}}
  {\eqterm{\G}{\e_1}{\e_2}{\T}}

  \infer[\rulename{c-eq-trans}]
  {\eqterm{\G}{\e_1}{\e_2}{\T}\\
   \eqterm{\G}{\e_2}{\e_3}{\T}}
  {\eqterm{\G}{\e_1}{\e_3}{\T}}

  \infer[\rulename{c-eq-conv}]
  {\eqterm{\G}{\e_1}{\e_2}{\T}\\
    \eqtype{\G}{\T}{\U}}
  {\eqterm{\G}{\e_1}{\e_2}{\U}}
\end{mathpar}

\paragraph{Equality reflection}
%
\begin{mathpar}
  \infer[\rulename{c-eq-reflection}]
  {\isterm{\G}{\e}{\Equal{\T}{\e_1}{\e_2}}}
  {\eqterm{\G}{\e_1}{\e_2}{\T}}
\end{mathpar}

\paragraph{Computations}

\begin{mathpar}
\infer[\rulename{c-prod-beta}]
  {\eqtype{\G}{\T_1}{\U_1}\\
    \eqtype{\ctxextend{\G}{\x}{\T_1}}{\T_2}{\U_2}\\\\
    \isterm{\ctxextend{\G}{\x}{\T_1}}{\e_1}{\T_2}\\
    \isterm{\G}{\e_2}{\U_1}}
  {\eqterm{\G}{\bigl(\capp{(\clam{\x}{\T_1}{\T_2}{\e_1})}{\x}{\U_1}{\U_2}{\e_2}\bigr)}
              {\subst{\e_1}{\x}{\e_2}}
              {\subst{\T_2}{\x}{\e_2}}}
\end{mathpar}

\paragraph{Extensionality}

%
\begin{mathpar}
  \infer[\rulename{c-eq-eta}]
  {\isterm{\G}{\e'_1}{\Equal{\T}{\e_1}{\e_2}} \\
    \isterm{\G}{\e'_2}{\Equal{\T}{\e_1}{\e_2}}
  }
  {\eqterm{\G}{\e'_1}{e'_2}{\Equal{\T}{\e_1}{\e_2}}}

  \infer[\rulename{c-prod-eta}]
  {\isterm{\G}{\e_1}{\cProd{\x}{\T}{\U}}\\
   \isterm{\G}{\e_2}{\cProd{\x}{\T}{\U}}\\\\
   \eqterm{\ctxextend{\G}{\x}{\T}}{(\capp{\e_1}{\x}{\T}{\U}{\x})}
          {(\capp{\e_2}{\x}{\T}{\U}{\x})}{\U}
  }
  {\eqterm{\G}{\e_1}{\e_2}{\cProd{\x}{\T}{\U}}}
\end{mathpar}

\subsubsection{Congruences}

\paragraph{Type formers}

\begin{mathpar}
  \infer[\rulename{c-cong-prod}]
  {\eqtype{\G}{\T_1}{\U_1}\\
   \eqtype{\ctxextend{\G}{\x}{\T_1}}{\T_2}{\U_2}}
  {\eqtype{\G}{\cProd{\x}{\T_1}{\T_2}}{\cProd{\x}{\U_1}{\U_2}}}

  \infer[\rulename{c-cong-eq}]
  {\eqtype{\G}{\T}{\U}\\
   \eqterm{\G}{\e_1}{\e'_1}{\T}\\
   \eqterm{\G}{\e_2}{\e'_2}{\T}
  }
  {\eqtype{\G}{\Equal{\T}{\e_1}{\e_2}}
              {\Equal{\U}{\e'_1}{\e'_2}}}
\end{mathpar}

\paragraph{Products}

\begin{mathpar}
  \infer[\rulename{c-cong-abs}]
  {\eqtype{\G}{\T_1}{\U_1}\\
    \eqtype{\ctxextend{\G}{\x}{\T_1}}{\T_2}{\U_2}\\
    \eqterm{\ctxextend{\G}{\x}{\T_1}}{\e_1}{\e_2}{\T_2}}
  {\eqterm{\G}{(\clam{\x}{\T_1}{\T_2}{\e_1})}
              {(\clam{\x}{\U_1}{\U_2}{\e_2})}
              {\cProd{\x}{\T_1}{\T_2}}}

  \infer[\rulename{c-cong-app}]
  {\eqtype{\G}{\T_1}{\U_1}\\
   \eqtype{\ctxextend{\G}{\x}{\T_1}}{\T_2}{\U_2}\\\\
   \eqterm{\G}{\e_1}{\e'_1}{\cProd{\x}{\T_1}{\T_2}}\\
   \eqterm{\G}{\e_2}{\e'_2}{\T_1}}
  {\eqterm{\G}{(\capp{\e_1}{\x}{\T_1}{\T_2}{\e_2})}{(\capp{\e'_1}{\x}{\U_1}{\U_2}{\e'_2})}{\subst{\T_2}{\x}{\e_2}}}
\end{mathpar}

\paragraph{Equality types}

%
\begin{mathpar}
\infer[\rulename{c-cong-refl}]
{\eqterm{\G}{\e_1}{\e_2}{\T}\\
 \eqtype{\G}{\T}{\U}}
{\eqterm{\G}{\refl{\T} \e_1}{\refl{\U} \e_2}{\Equal{\T}{\e_1}{\e_1}}}
\end{mathpar}

%%%%%%%%%%%%%%%%%%%%%%%%%%%%%%%%%%%%%%%%%%%%%%%%%%%%%%%%%

%%% Local Variables:
%%% mode: latex
%%% TeX-master: "theory"
%%% End:
