\section{Algorithmic version}

Andromeda should be thought of as a programming language for deriving judgments. At the moment the language is untyped. We can hope to have it \emph{simply typed} one day.

\subsection{Syntax} % (fold)
\label{sub:prog-syntax}

Expressions:
%
\begin{equation*}
  \expr
  \begin{aligned}[t]
    \bnf   {}& \cmdType & & \text{universe}\\
    \bnfor {}& \x   &&\text{variable} \\
  \end{aligned}
\end{equation*}
%
Computations:
%
\begin{equation*}
  \cmd
  \begin{aligned}[t]
    \bnf   {}& \cmdReturn \expr              &&\text{pure expressions} \\
    \bnfor {}& \cmdLet{\x}{\cmd_1} \cmd_2    &&\text{let binding} \\
    \bnfor {}& \cmdAscribe{\cmd}{\expr}      &&\text{ascription} \\
    \bnfor {}& \cmdProd{x}{\expr} \cmd       &&\text{product}\\
    \bnfor {}& \cmdEq{\cmd}{\cmd}            &&\text{equality type} \\
    \bnfor {}& \cmdLam{\x}{\expr} \cmd       &&\text{$\lambda$-abstraction} \\
    \bnfor {}& \cmdApp{\expr}{\cmd}          &&\text{application} \\
    \bnfor {}& \cmdRefl \cmd                 &&\text{reflexivity}
  \end{aligned}
\end{equation*}

The result of a computation is a value, which is a pair $(e,T)$ where $e$ and $T$ are terms of type theory, as described in Section~\ref{sec:syntax}. The correctness guarantee which we want is that a computation only ever evaluates to derivable judgments.

% subsection prog-syntax (end)

\subsubsection{Operational semantics} % (fold)
\label{ssub:operational_semantics}

Operational semantics is given by \emph{two} versions of evaluation of computations, called \emph{inference} and \emph{checking}, of the forms:
%
\begin{align*}
  \text{Inference:}&\quad \evali{\ctxenv}{\cmd}{\e}{\T} \\
  \text{Checking:}&\quad  \evalc{\ctxenv}{\cmd}{\T}{\e}
\end{align*}
%
These are read as ``in the given context $\G$ and environment $\env$ command $\cmd$ infers that $\e$ has type $\T$'' and ``in the given context $\G$ and environment $\env$ command $\cmd$ checks that $\e$ has the given type $\T$.''

[EXPLAIN THAT AN ENVIRONMENT MAPS VARIABLES TO VALUES.]

\begin{mathpar}

  \infer[\rulename{ev-check-infer}]
  {\evali{\ctxenv}{\cmd}{\e}{\U} \\
    \eqtypealg{\ctxenv}{\T}{\U}}
  {\evalc{\ctxenv}{\cmd}{\T}{\e}}

  \infer[\rulename{ev-infer-type}]
  {}
  {\evali{\ctxenv}{\cmdReturn \Type}{\Type}{\Type}}

  \infer[\rulename{ev-infer-product}]
  {\evalc \ctxenv {\cmd_1} \Type {\T_1} \\
    \evalc {\ctxextend \G \x {\T_1};\, \env} {\cmd_2} \Type {\T_2}}
  {\evali \ctxenv {\cmdProd \x {\cmd_1} {\cmd_2}} {\cProd \x {\T_1} {\T_2}} \Type}

  \infer[\rulename{ev-infer-eq}]
  {\evali \ctxenv {\cmd_1} {\e_1} \T \\
    \evalc \ctxenv {\cmd_2} \T {\e_2}}
  {\evali \ctxenv {\cmdEq {\cmd_1} {\cmd_2}} {\Equal {\T} {\e_1} {\e_2}} \Type}

  \infer[\rulename{ev-infer-ascription}]
  {\evalc \ctxenv \expr \Type \T \\
    \evalc \ctxenv \cmd \T \e}
  {\evali \ctxenv {\cmdAscribe \cmd \expr} \e \T}

\end{mathpar}

Inference of $\cmdEq{\cmd_1} {\cmd_2}$ keeps the type of the first argument.

Ascription only has an infer rule, and will always switch to a checking phase. It breaks the inward information flow and has to use the check-infer when encountered during checking.

\begin{mathpar}

  \infer[\rulename{ev-infer-var}]
  {\env(\x) = (\e, \T)}
  {\evali{\ctxenv}{\cmdReturn \x}{\e}{\T}}

  % supposedly the user annotated the lambda, so we should use the type \T_1
  % which we get out of \expr in the recursive call, instead of \U_1
  \infer[\rulename{ev-check-$\lambda$-tagged}]
  {\tywhnfs \ctxenv \U {\cProd \x {\U_1} {\U_2}} \\
    \evalc \ctxenv \expr \Type {\T_1} \\
    \eqtypealg \ctxenv {\T_1} {\U_1} \\
    \evalc {\ctxextend \G \x {\T_1};\, \env} \cmd {\U_2} \e}
  {\evalc \ctxenv {\cmdLam \x \expr \cmd} \U {\clam \x {\U_1} {\U_2} \e}}
  % XXX should the result be (x:U1)->U2 or (x:T1)->U2?

  \infer[\rulename{ev-check-$\lambda$-untagged}]
  {\tywhnfs \ctxenv \U {\cProd \x {\U_1} {\U_2}} \\
    \evalc {\ctxextend \G \x {\U_1};\, \env} \cmd {\U_2} \e}
  {\evalc \ctxenv {\cmdLamCurry \x \cmd} \U {\clam \x {\U_1} {\U_2} \e}}

  \infer[\rulename{ev-infer-$\lambda$}]
  {\evalc \ctxenv \expr \Type {\U_1} \\
    \evali {\ctxextend \G \x {\U_1};\, \envextend \env \x \x {\U_1}} \cmd \e {\U_2}}
  {\evali \ctxenv {\cmdLam \x \expr \cmd} {\clam \x {\U_1} {\U_2} \e} {\cProd \x {\U_1} {\U_2}}}

  \infer[\rulename{ev-infer-app}]
  {\evali{\ctxenv}{\expr}{\e_1}{\T} \\
   \tywhnfs{\ctxenv}{\T}{\cProd{\x}{\U_1} \U_2} \\
   \evalc{\ctxenv}{\cmd}{\U_1}{\e_2}
  }
  {\evali
    {\ctxenv}
    {\cmdApp{\expr}{\cmd}}
    {\capp{\e_1}{\x}{\U_1}{\U_2}{\e_2}}
    {\subst{\U_2}{\x}{\e_2}}
  }

  \infer[\rulename{ev-check-app-non-dep}]
  {\evali \ctxenv \cmd {\e_2} \U \\
    \evalc \ctxenv \expr {\cProd \_ \U \T} {\e_1}}
  {\evalc \ctxenv {\cmdApp \expr \cmd} \T
    {\capp{\e_1} \_ \U \T {\e_2}}}

  \infer[\rulename{ev-infer-refl}]
  {\evali{\ctxenv}{\cmd}{\e}{\T}}
  {\evali{\ctxenv}{\cmdRefl \cmd}{\refl{\T}{\e}}{\Equal{\T}{\e}{\e}}}

  \infer[\rulename{ev-check-refl}]
  {\tywhnfs{\ctxenv}{\T}{\Equal{\U}{\e_1}{\e_2}} \\
   \evalc{\ctxenv}{\cmd}{\U}{\e} \\
   \eqtermalg{\ctxenv}{\e}{\e_1}{\U} \\
   \eqtermalg{\ctxenv}{\e}{\e_2}{\U}
 }
  {\evalc{\ctxenv}{\cmdRefl \cmd}{\T}{\refl \T \e}}

  \infer[\rulename{ev-check-let}]
  {\evali \ctxenv {\cmd_1} {\e_1} \U \\
  \evalc {\G;\, \envextend \env \x \e \U} {\cmd_2} \T {\e_2}}
  {\evalc \ctxenv {\cmdLet \x {\cmd_1} \cmd_2} \T {\e_2}}

  \infer[\rulename{ev-infer-let}]
  {\evali \ctxenv {\cmd_1} {\e_1} \U \\
  \evali {\G;\, \envextend \env \x \e \U} {\cmd_2} {\e_2} \T}
  {\evali \ctxenv {\cmdLet \x {\cmd_1} \cmd_2} {\e_2} \T}

\end{mathpar}

TODO: check the freshness (and other side-conditions?).

% subsubsection operational_semantics (end)

%%% Local Variables:
%%% mode: latex
%%% TeX-master: "theory"
%%% End:
