% Universe indices

\newcommand{\NN}{\mathbb{N}} % God-given numbers

\newcommand{\PP}{\mathbb{P}} % the poset of universe indices
\newcommand{\FF}{\mathbb{F}} % the subposet of fibered universe indices
\newcommand{\zero}{o} % this is not o, it is omicron

\newcommand{\piClose}[2]{\mathsf{max}(#1,#2)}   % the universe containing a pi with given domain and codomain
\newcommand{\uClose}[1]{\mathsf{succ}(#1)}  % the universe containing the given universe's name.

% Generic entities
\newcommand{\G}{\Gamma} % a context
\newcommand{\D}{\Delta} % another context
\newcommand{\T}{T} % a type
\newcommand{\U}{U} % another type
\newcommand{\x}{x} % a term variable
\newcommand{\y}{y} % another term variable
\newcommand{\z}{z} % a third term variable
\newcommand{\e}{e} % a term

\newcommand{\rulename}[1]{\text{\textsc{#1}}}

%% Syntax
\newcommand{\bnf}{\ \mathrel{{:}{:}{=}}\ }
\newcommand{\bnfor}{\ \mid\ \ }

%% Syntactic constructs
\newcommand{\ctxempty}{\bullet} % empty context
\newcommand{\ctxextend}[3]{#1,\, #2\, {:}\, #3} % extended context

\newcommand{\subst}[3]{#1[#3/#2]} % substitution
\newcommand{\substs}[2]{#1[#2]} % substitution of many variables

\newcommand{\Universe}[1]{\mathbb{U}_{#1}} % non-fibered universe
\newcommand{\El}[2]{\mathsf{El}^{#1}\, #2} % the type named by the term

\newcommand{\Unit}{\mathsf{Unit}} % The unit type
\newcommand{\Prod}[2]{\mathop{\textstyle\prod_{(#1 {:} #2)}}} % dependent product

\newcommand{\lam}[3]{\lambda #1 {:} #2.{#3}\,.\,} % $\lambda$-abstraction
\newcommand{\app}[5]{#1\mathbin{@^{#2{:}#3.#4}} #5} % application

\newcommand{\abst}[2]{[#1 \,.\, #2]} % abstraction
\newcommand{\ascribe}[2]{#1 \,{:}{:}\, #2} % type ascription

\newcommand{\unitTerm}{\star} % the element of the unit type

\newcommand{\coerce}[3]{\mathsf{coerce}^{#1{\mapsto}#2} \, #3}

\newcommand{\PrEqual}[3]{\mathsf{Paths}_{#1}(#2,#3)} % Propositional Equality type
\newcommand{\JuEqual}[3]{\mathsf{Id}_{#1}(#2,#3)} % Judgmental Equality type

\newcommand{\PrElim}[6]{\mathsf{J}_{#1}(#2, #3, #4, #5, #6)} % Propositional equality eliminator

\newcommand{\prRefl}[1]{{\mathsf{idpath}_{#1}}\ }  % Propositional refl
\newcommand{\juRefl}[1]{{\mathsf{refl}_{#1}}\ }    % Judgmental refl

% names
\newcommand{\nUnit}{\mathsf{unit}} % name of unit type
\newcommand{\nProd}[4]{\pi^{#1,#2} #3\,{:}\,#4 \,.\ } % name of a dependent product
\newcommand{\nUniverse}[1]{u_{#1}}  % name of a universe
\newcommand{\nPrEqual}[4]{\mathsf{paths}^{#1}_{#2}(#3,#4)} % Propositional Equality type
\newcommand{\nJuEqual}[4]{\mathsf{id}^{#1}_{#2}(#3,#4)} % Judgmental Equality type


% orthodox judgments
\newcommand{\isctx}[1]{#1\;\mathsf{ctx}} % well formed context
\newcommand{\istype}[2]{#1 \vdash #2\;\mathsf{type}} % well formed type
\newcommand{\isfib}[2]{#1 \vdash #2\;\mathsf{fibered}} % is a fibered type
\newcommand{\isterm}[3]{#1 \vdash\,#2\,:\,#3} % well formed term

\newcommand{\eqtype}[3]{#1 \vdash #2 \equiv #3} % equal types
\newcommand{\eqterm}[4]{#1 \vdash #2 \equiv #3 : #4} % equal terms

% algorithmic judgments

\newcommand{\chkterm}[3]{#1 \vdash #2 \Leftarrow #3} % checking
\newcommand{\synterm}[3]{#1 \vdash #2 \Rightarrow #3} % synthesis

\newcommand{\ishints}[2]{#1 \vdash #2 \; \mathsf{hints}} % well-formed context and hints

\newcommand{\istypealg}[2]{#1 \vdash #2 \Leftarrow \mathsf{type}} % well formed type
\newcommand{\isfibalg}[2]{#1 \vdash #2 \Leftarrow \mathsf{fibered}} % is a fibered type
\newcommand{\eqtypealg}[3]{#1 \vdash #2 \thickapprox #3} % equal types
\newcommand{\eqtypepath}[3]{#1 \vdash #2 \thicksim #3} % equal normal-form types
\newcommand{\eqtermalg}[4]{#1 \vdash #2 \thickapprox #3 \Leftarrow #4} % equal terms of normalized type
\newcommand{\eqtermext}[4]{#1 \vdash #2 \simeq #3 \Leftarrow #4} % equal terms w/o eta
\newcommand{\eqpath}[4]{#1 \vdash #2 \thicksim #3}  % equal paths

\newcommand{\equationin}[3]{\mathsf{equation}\; #1 \,{:}\, #2 \,{\equiv}\, #3 \; \mathsf{in} \;} % use equation hint
\newcommand{\rewritein}[3]{\mathsf{rewrite}\; #1 \,{:}\, #2 \,{\equiv}\, #3 \; \mathsf{in} \;} % use rewrite hint
\newcommand{\eqhint}[2]{(#1 \,{\equiv}\, #2)} % equality hint
\newcommand{\rwhint}[2]{(#1 \,{\leadsto}\, #2)} % rewrite hint

\renewcommand{\H}{\mathcal{H}}      % term hint list
\newcommand{\hintempty}{\circ}      % empty hint list

\newcommand{\addhinteq}[3]{#1, \eqhint{#2}{#3}} % extended hint list
\newcommand{\addhintrw}[3]{#1, \rwhint{#2}{#3}} % extended hint list
\newcommand{\ctxs}[2]{#1\,;\,#2}
\newcommand{\GH}{\ctxs{\G}{\H}}           % combined context and hints

% normalization

\newcommand{\tywhnf}[4]{#1 \vdash #2 \leadsto #3 / #4} % "one-step type normalization"
\newcommand{\whnf}[4]{#1 \vdash #2 \leadsto #3 / #4} % "one-step term normalization"

\newcommand{\tywhnfs}[3]{#1 \vdash #2 \leadsto^* #3 \not\leadsto } % "type normalization"
\newcommand{\whnfs}[3]{#1 \vdash #2 \leadsto^* #3 \not\leadsto } % "term normalization"

\newcommand{\inferred}[1]{{\color{magenta}{#1}}}

\newcommand{\simplefib}{\mathop{\mathsf{isfib}}\,}
\newcommand{\true}{\mathsf{true}}
\newcommand{\false}{\mathsf{false}}

\newcommand{\nameof}{\mathop{\mathsf{name\_of}}\,}
\newcommand{\typeof}{\mathop{\mathsf{type\_of}}\,}

