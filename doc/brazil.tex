\documentclass{article}

\usepackage{times}
\usepackage{mathpartir}
\usepackage{amsmath,amsfonts}

%% Macros here

% Universe indices

\newcommand{\NN}{\mathbb{N}} % God-given numbers

\newcommand{\PP}{\mathbb{P}} % the poset of universe indices
\newcommand{\FF}{\mathbb{F}} % the subposet of fibered universe indices
\newcommand{\zero}{o} % this is not o, it is omicron

\newcommand{\piClose}[2]{\mathsf{piClose}(#1,#2)}   % the universe containing a pi with given domain and codomain
\newcommand{\uClose}[1]{\mathsf{uClose}(#1)}  % the universe containing the given universe's name.

% Generic entities
\newcommand{\G}{\Gamma} % a context
\newcommand{\T}{T} % a type
\newcommand{\U}{U} % another type
\newcommand{\x}{x} % a term variable
\newcommand{\y}{y} % another term variable
\newcommand{\e}{e} % a term

\newcommand{\rulename}[1]{\text{\textsc{#1}}}

%% Syntax
\newcommand{\bnf}{\ \mathrel{{:}{:}{=}}\ }
\newcommand{\bnfor}{\mid}

%% Syntactic constructs
\newcommand{\ctxempty}{\bullet} % empty context
\newcommand{\ctxextend}[3]{#1, #2 {:} #3} % extended context

\newcommand{\subst}[3]{#1[#3/#2]} % substitution

\newcommand{\Universe}[1]{\mathbb{U}_{#1}} % non-fibered universe
\newcommand{\El}[2]{\mathsf{El}^{#1} #2} % the type named by #2


\newcommand{\Unit}{\mathsf{Unit}} % The unit type
\newcommand{\Prod}[2]{{\textstyle\prod_{(#1 {:} #2)}}} % dependent product

\newcommand{\lam}[2]{\lambda #1 {:} #2 .\,} % $\lambda$-abstraction
\newcommand{\app}[2]{#1\,#2} % application

\newcommand{\unitTerm}{\star} % the element of the unit type

\newcommand{\coerce}[3]{\mathsf{lift}^{#2{\mapsto}#3}\ #1}

\newcommand{\PrEqual}[3]{\mathsf{Paths}_{#1}(#2,#3)} % Propositional Equality type
\newcommand{\JuEqual}[3]{\mathsf{Id}_{#1}(#2,#3)} % Judgmental Equality type

\newcommand{\prRefl}[1]{\mathsf{idpath}_{#1}}  % Propositional refl
\newcommand{\juRefl}[1]{\mathsf{refl}_{#1}}    % Judgmental refl

% names
\newcommand{\nUnit}{\mathsf{unit}} % name of unit type
\newcommand{\nProd}[3]{\pi^{#1} #2 {:} #3 \,.\,} % name of a dependent product
\newcommand{\nUniverse}[1]{u_{#1}}  % name of a universe
\newcommand{\nPrEqual}[3]{\mathsf{paths}_{#1}(#2,#3)} % Propositional Equality type
\newcommand{\nJuEqual}[3]{\mathsf{id}_{#1}(#2,#3)} % Judgmental Equality type


% judgments
\newcommand{\isctx}[1]{#1\;\mathsf{ctx}} % well formed context
\newcommand{\istype}[2]{#1 \vdash #2\;\mathsf{type}} % well formed type
\newcommand{\isfib}[2]{#1 \vdash #2\;\mathsf{fibered}} % is a fibered type
\newcommand{\isterm}[3]{#1 \vdash\,#2\,:\,#3} % well formed term

\newcommand{\eqtype}[3]{#1 \vdash\, #2\, \equiv\, #3} % equal types
\newcommand{\eqterm}[4]{#1 \vdash\, #2\, \equiv #3\,:\,#4} % equal terms

\begin{document}

\title{Brazilian type theory}
\author{Andrej Bauer \and Christopher A. Stone}
\maketitle

\section{Universe indices}
\label{sec:universe-indices}

There is a pointed poset $(\PP, {\leq}, \zero)$ of \emph{universe
  indices} and a subset $\FF \subseteq \PP$ of \emph{fibered indices}.
For example, we may take
%
\begin{equation*}
  \PP = \{0,1\} \times \NN
\end{equation*}
%
with the lexicographic order, $\zero = (0,0)$ and $\FF = \{0\} \times
\NN$. We use lower Greek letters for universe indices.

We assume the following partial functions are given:
\begin{itemize}
  \item $\piClose{\cdot}{\cdot} : \PP \times \PP \rightharpoonup \PP$
  \item $\uClose{\cdot}  : \PP \rightharpoonup \PP$
\end{itemize}
\section{Syntax}
\label{sec:syntax}

Contexts:
%
\begin{align*}
  \G \bnf
  \ctxempty \bnfor
  \ctxextend{\G}{\x}{\T}
\end{align*}
%
Types:
%
\begin{align*}
  \T, \U \bnf
  \Universe{\alpha} \bnfor
  \El{\alpha}{\e} \bnfor
  \Unit \bnfor
  \Prod{x}{\T} \U \bnfor
  \PrEqual{T}{\e_1}{\e_2} \bnfor
  \JuEqual{T}{\e_1}{\e_2}
\end{align*}
%
Terms:
%
\begin{align*}
  \e \bnf
  &\x \bnfor
  \lam{\x}{\T} \e \bnfor
  \app{\e_1}{\e_2} \bnfor
  \unitTerm \bnfor
  \nUnit \bnfor
  \nProd{\alpha,\beta}{\x}{\e_1} \e_2 \bnfor
  \nUniverse{\alpha} \bnfor {} \\
  & \coerce{\e}{\alpha}{\beta} \bnfor
  \nPrEqual{\e_1}{\e_2}{\e_3} \bnfor
  \nJuEqual{\e_1}{\e_2}{\e_3}
\end{align*}

We also need an elim form for propositional equality. Should we have an elim form for definitional equality too, or is it an implicit rule (if something has an Id type, then the definitional equivalence holds)?

\section{Judgments}
\label{sec:judgments}

\begin{align*}
& \isctx{\G} & & \text{$\G$ is a well formed context} \\
& \istype{\G}{\T} & & \text{$\T$ is a type in context $\G$} \\
& \isfib{\G}{\T} & & \text{$\T$ is a fibered type in context $\G$} \\
& \isterm{\G}{\e}{\T} & & \text{$\e$ is a well formed term of type $\T$ in context $\G$} \\
& \eqtype{\G}{\T}{\U} & & \text{$\T$ and $\U$ are equal types in context $\G$} \\
& \eqterm{\G}{\e_1}{\e_2}{\T} & & \text{$e_1$ and $e_2$ are equal terms of type $\T$ in context $\G$}
\end{align*}

\section{Contexts}
\label{sec:contexts}

\begin{mathpar}
  \infer[\rulename{ctx-empty}]
  { }
  {\isctx{\ctxempty}}

  \infer[\rulename{ctx-extend}]
  {\isctx{\G} \\
   \istype{\G}{\T}
  }
  {\isctx{\ctxextend{\G}{\x}{\T}}}
\end{mathpar}

\section{Types}
\label{sec:types}

\begin{mathpar}
  \infer[\rulename{ty-universe}]
  {\isctx{\G} \\
   \alpha \in \PP
  }
  {\istype{\G}{\Universe{\alpha}}}

  \infer[\rulename{ty-prod}]
  {\istype{\G}{\T} \\
   \istype{\ctxextend{\G}{\x}{\T}}{\U}
  }
  {\istype{\G}{\Prod{\x}{\T}{\U}}}

  \infer[\rulename{ty-el}]
  {\isterm{\G}{\e}{\Universe{\alpha}}}
  {\istype{\G}{\El{\alpha}{\e}}}

  \infer[\rulename{ty-unit}]
  {\isctx{\G}}
  {\istype{\G}{\Unit}}

  \infer[\rulename{ty-paths}]
  {\isfib{\G}{\T}\\
   \isterm{\G}{\e_1}{\T}\\
   \isterm{\G}{\e_2}{\T}
  }
  {\istype{\G}{\PrEqual{\T}{\e_1}{\e_2}}}

  \infer[\rulename{ty-id}]
  {\istype{\G}{\T}\\
   \isterm{\G}{\e_1}{\T}\\
   \isterm{\G}{\e_2}{\T}
  }
  {\istype{\G}{\JuEqual{\T}{\e_1}{\e_2}}}


\end{mathpar}

\section{Fibered types}
\label{sec:fibered-types}

\begin{mathpar}
  \infer[\rulename{fib-universe}]
  {\isctx{\G} \\
   \alpha \in \PP
  }
  {\isfib{\G}{\Universe{\alpha}}}

  \infer[\rulename{fib-prod}]
  {\isfib{\G}{\T} \\
   \isfib{\ctxextend{\G}{\x}{\T}}{\U}
  }
  {\isfib{\G}{\Prod{\x}{\T}{\U}}}

  \infer[\rulename{fib-el}]
  {\isterm{\G}{\e}{\Universe{\alpha}} \\
   \alpha \in \FF
  }
  {\isfib{\G}{\El{\alpha}{\e}}}

  \infer[\rulename{fib-unit}]
  {\isctx{\G} \\
   \zero\in\FF }
  {\istype{\G}{\Unit}}

  \infer[\rulename{fib-paths}]
  {\isfib{\G}{\T}\\
   \isterm{\G}{\e_1}{\T}\\
   \isterm{\G}{\e_2}{\T}
  }
  {\isfib{\G}{\PrEqual{\T}{\e_1}{\e_2}}}



\end{mathpar}

\section{Terms}
\label{sec:terms}

\begin{mathpar}
  \infer[\rulename{term-eq}]
  {\isterm{\G}{\e}{\T} \\
   \eqtype{\G}{\T}{\U}
  }
  {\isterm{\G}{\e}{\U}}

  \infer[\rulename{term-var}]
  {\isctx{\G} \\
   (\x{:}\T) \in \G
  }
  {\isterm{\G}{\x}{\T}}

  \infer[\rulename{term-abs}]
  {\isterm{\ctxextend{\G}{\x}{\T}}{\e}{\U}}
  {\isterm{\G}{(\lam{\x}{\T}{\e})}{\Prod{\x}{\T}{\U}}}

  \infer[\rulename{term-app}]
  {\isterm{\G}{\e_1}{\Prod{x}{\T} \U} \\
   \isterm{\G}{\e_2}{\T}
  }
  {\isterm{\G}{\app{\e_1}{\e_2}}{\subst{\U}{\x}{\e_2}}}

  \infer[\rulename{term-star}]
  {\isctx{\G}}
  {\isterm{\G}{\unitTerm}{\Unit}}

  \infer[\rulename{term-coerce}]
  {\isterm{\G}{\e}{\Universe{\alpha}}\\
   \alpha \leq \beta}
  {\isterm{\G}{\coerce{\e}{\alpha}{\beta}}{\Universe{\beta}}}

  \infer[\rulename{term-idpath}]
  {\isterm{\G}{\e}{\T}\\
   \isfib{\G}{\T}}
  {\isterm{\G}{\prRefl{\T}{\e}}{\PrEqual{\T}{\e}{\e}}}


  \infer[\rulename{term-refl}]
  {\isterm{\G}{\e}{\T}}
  {\isterm{\G}{\juRefl{\T}{\e}}{\JuEqual{\T}{\e}{\e}}}


  \infer[\rulename{name-unit}]
  {\isctx{\G}}
  {\isterm{\G}{\nUnit}{\Universe{\zero}}}

  \infer[\rulename{name-prod}]
  {\isterm{\G}{\e_1}{\Universe{\alpha}} \\
   \isterm{\ctxextend{\G}{\x}{\El{\alpha}{\e_1}}}{\e_2}{\Universe{\beta}}
  }
  {\isterm{\G}{(\nProd{\alpha,\beta}{\x}{\e_1} \e_2)}{\Universe{\piClose{\alpha}{\beta}}}}

  \infer[\rulename{name-universe}]
  { }
  {\isterm{\G}{\nUniverse{\alpha}}{\Universe{\uClose{\alpha}}}}

  \infer[\rulename{name-paths}]
  {\isterm{\G}{\e_\T}{\Universe{\alpha}}\\
   \alpha\in\FF\\
   \isterm{\G}{\e_1}{\El{\alpha}{\e_\T}}\\
   \isterm{\G}{\e_2}{\El{\alpha}{\e_\T}}
  }
  {\isterm{\G}{\nPrEqual{\e_T}{\e_1}{\e_2}}{\Universe{\alpha}}}

  \infer[\rulename{name-id}]
  {\isterm{\G}{\e_\T}{\Universe{\alpha}}\\
   \isterm{\G}{\e_1}{\El{\alpha}{\e_\T}}\\
   \isterm{\G}{\e_2}{\El{\alpha}{\e_\T}}
  }
  {\isterm{\G}{\nJuEqual{\e_\T}{\e_1}{\e_2}}{\Universe{\alpha}}}

\end{mathpar}

\section{Type Equality}

\begin{mathpar}
  \infer[\rulename{tyeq-refl}]
  {\istype{\G}{\T}}
  {\eqtype{\G}{\T}{\T}}

  \infer[\rulename{tyeq-sym}]
  {\eqtype{\G}{\U}{\T}}
  {\eqtype{\G}{\T}{\U}}

  \infer[\rulename{tyeq-trans}]
  {\eqtype{\G}{\T}{\T'}\\
   \eqtype{\G}{\T'}{\U}}
  {\eqtype{\G}{\T}{\U}}

  \infer[\rulename{tyeq-el}]
  {\eqterm{\G}{\e_1}{\e_2}{\Universe{\alpha}}}
  {\eqtype{\G}{\El{\alpha}{\e_1}}{\El{\alpha}{\e_2}}}

  \infer[\rulename{tyeq-prod}]
  {\eqtype{\G}{\T_1}{\U_1}\\
   \eqtype{\ctxextend{\G}{\x}{\T_1}}{\T_2}{\U_2}}
  {\eqtype{\G}{\Prod{\x}{\T_1}{\T_2}}{\Prod{\x}{\U_1}{\U_2}}}

  \infer[\rulename{tyeq-paths}]
  {\isfib{\G}{\T}\\
   \isfib{\G}{\U}\\\\
   \eqtype{\G}{\T}{\U}\\
   \eqterm{\G}{\e_1}{\e'_1}{\T}\\
   \eqterm{\G}{\e_2}{\e'_2}{\T}
  }
  {\eqtype{\G}{\PrEqual{\T}{\e_1}{\e_2}}
              {\PrEqual{\U}{\e'_1}{\e'_2}}}

  \infer[\rulename{tyeq-id}]
  {\eqtype{\G}{\T}{\U}\\
   \eqterm{\G}{\e_1}{\e'_1}{\T}\\
   \eqterm{\G}{\e_2}{\e'_2}{\T}
  }
  {\eqtype{\G}{\JuEqual{\T}{\e_1}{\e_2}}
              {\JuEqual{\T}{\e'_1}{\e'_2}}}

  \infer[\rulename{tyeq-el-pi}]
  { }
  {\eqtype{\G}{\El{\piClose{\alpha}{\beta}}{(\nProd{\alpha,\beta}{\x}{\e_1}{\e_2})}}
              {\Prod{\x}{\El{\alpha}{\e_1}}{\El{\beta}{\e_2}}}}

  \infer[\rulename{tyeq-el-unit}]
  { }
  {\eqtype{\G}{\El{\zero}{\nUnit}}{\Unit}}

  \infer[\rulename{tyeq-el-coerce}]
  {\isterm{\G}{\e}{\Universe{\alpha}} }
  {\eqtype{\G}{\El{\beta}(\coerce{\e}{\alpha}{\beta})}
              {\El{\alpha}{e}}}

  \infer[\rulename{tyeq-el-paths}]
  {\isterm{\G}{\e_\T}{\Universe{\alpha}}\\
   \alpha\in\FF\\
   \isterm{\G}{\e_1}{\El{\alpha}{\e_\T}}\\
   \isterm{\G}{\e_2}{\El{\alpha}{\e_\T}}
  }
  {\eqtype{\G}{\El{\alpha}{(\nPrEqual{\e_T}{\e_1}{\e_2})}}
          {\PrEqual{\El{\alpha}{\e_\T}}{\e_1}{\e_2}}}

  \infer[\rulename{tyeq-el-id}]
  {\isterm{\G}{\e_\T}{\Universe{\alpha}}\\
   \isterm{\G}{\e_1}{\El{\alpha}{\e_\T}}\\
   \isterm{\G}{\e_2}{\El{\alpha}{\e_\T}}
  }
  {\eqtype{\G}{\El{\alpha}{(\nJuEqual{\e_\T}{\e_1}{\e_2})}}
          {\JuEqual{\El{\alpha}{\e_\T}}{\e_1}{\e_2}}}

\end{mathpar}

\section{Term Equality}

\begin{mathpar}
%
% PER rules
%
  \infer[\rulename{eq-refl}]
  {\isterm{\G}{\e}{\T}}
  {\eqterm{\G}{\e}{\e}{\T}}

  \infer[\rulename{eq-sym}]
  {\eqterm{\G}{\e_2}{\e_1}{\T}}
  {\eqterm{\G}{\e_1}{\e_2}{\T}}

  \infer[\rulename{eq-trans}]
  {\eqterm{\G}{\e_1}{\e_2}{\T}\\
   \eqterm{\G}{\e_2}{\e_3}{\T}}
  {\eqterm{\G}{\e_1}{\e_3}{\T}}


%
% Subsumption-like rule
%

  \infer[\rulename{eq-eq}]
  {\eqterm{\G}{\e_1}{\e_2}{\T}\\
    \eqtype{\G}{\T}{\U}}
  {\eqterm{\G}{\e_1}{\e_2}{\U}}

%
% Congruence and other rules
%
  \infer[\rulename{eq-abs}]
  {\eqtype{\G}{\T_1}{\U_1}\\
    \eqterm{\ctxextend{\G}{\x}{\T_1}}{\e_1}{\e_2}{\T_2}}
  {\eqterm{\G}{(\lam{\x}{\T_1}{\e_1})}
              {(\lam{\x}{\U_1}{\e_2})}
              {\Prod{\x}{\T_1}{\T_2}}}

  \infer[\rulename{eq-app}]
  {\eqterm{\G}{\e_1}{\e'_1}{\Prod{\x}{\T}{\U}}\\
   \eqterm{\G}{\e_2}{\e'_2}{\T}}
  {\eqterm{\G}{\app{\e_1}{\e_2}}{\app{\e'_1}{\e'_2}}{\subst{\U}{\x}{\e_2}}}

  \infer[\rulename{eq-beta}]
  {\isterm{\ctxextend{\G}{\x}{\T}}{e_1}{\U}\\
    \isterm{\G}{\e_2}{\T}}
  {\eqterm{\G}{\app{(\lam{\x}{\T}{\e_1})}{\e_2}}
              {\subst{\e_1}{\x}{\e_2}}
              {\subst{\U}{\x}{\e_2}}}

  \infer[\rulename{eq-ext}]
  {\isterm{\G}{\e_1}{\Prod{\x}{\T}{\U}}\\
   \isterm{\G}{\e_2}{\Prod{\x}{\T}{\U}}\\
   \eqterm{\ctxextend{\G}{\x}{\T}}{(\app{\e_1}{\x})}
          {(\app{\e_2}{\x})}{\U}
  }
  {\eqterm{\G}{\e_1}{\e_2}{\Prod{\x}{\T}{\U}}}

\infer[\rulename{eq-prod}]
  {\eqterm{\G}{\e_1}{\e'_1}{\Universe{\alpha}}\\
   \eqterm{\ctxextend{\G}{\x}{\El{\alpha}{\e_1}}}{\e_2}{\e'_2}{\Universe{\beta}}}
  {\eqterm{\G}{\nProd{\alpha,\beta}{\x}{\e_1}{\e_2}}{\nProd{\alpha,\beta}{\x}{\e'_1}{\e'_2}}{\Universe{\piClose{\alpha}{\beta}}}}

\infer[\rulename{eq-idpath}]
  {\eqterm{\G}{\e_1}{\e_2}{\T}\\
   \eqtype{\G}{\T}{\U}\\
   \isfib{\G}{\T}\\
   \isfib{\G}{\U}}
  {\eqterm{\G}{\prRefl{\T}{\e_1}}{\prRefl{\U}{\e_2}}{\PrEqual{\T}{\e_1}{\e_1}}}

\infer[\rulename{eq-refl}]
{\eqterm{\G}{\e_1}{\e_2}{\T}\\
 \eqtype{\G}{\T}{\U}}
{\eqterm{\G}{\juRefl{\T}{\e_1}}{\juRefl{\U}{\e_2}}{\JuEqual{\T}{\e_1}{\e_1}}}

\infer[\rulename{eq-coerce}]
  {\eqterm{\G}{\e_1}{\e_2}{\Universe{\alpha}}}
  {\eqterm{\G}{(\coerce{\e_1}{\alpha}{\beta})}
              {(\coerce{\e_2}{\alpha}{\beta})}
              {\Universe{\beta}}}

\infer[\rulename{eq-coerce-refl}]
  {\isterm{\G}{\e}{\Universe{\alpha}}}
  {\eqterm{\G}{(\coerce{\e}{\alpha}{\alpha})}{e}{\Universe{\alpha}}}

\infer[\rulename{eq-coerce-trans}]
  {\isterm{\G}{\e}{\Universe{\alpha}} \\
    \alpha \leq \gamma \leq \beta}
  {\eqterm{\G}{\coerce{(\coerce{\e}{\alpha}{\gamma}}{\gamma}{\beta})}
              {\coerce{\e}{\alpha}{\beta}}
              {\Universe{\beta}}}

\end{mathpar}
\bigskip

Question: Assuming that $\piClose{0}{1} = \piClose{1}{1} = 1$, should we be able to prove that
\[ \eqterm{\ctxempty}{(\nProd{0,1}{\_}{\nUnit}{\nUniverse{1}})}{(\nProd{1,1}{\_}{(\coerce{\nUnit}{0}{1})}{\nUniverse{1}})}
                     {\Universe{2}}  ? \]

This would follow from the reflection rule
\[
  \infer
   {\eqtype{\G}{\El{\alpha}{\e_1}}{\El{\alpha}{\e_2}}}
   {\eqterm{\G}{\e_1}{\e_2}{\Universe{\alpha}}}
\]
because both sides correspond to the type $\Unit\to\Universe{1}$.
But we definitely need the inverse Rule~\rulename{tyeq-el}, for example, to prove that
\[ \eqtype{\ctxempty}
          {\El{\zero}{(\app{(\lam{\x}{\Universe{\zero}}{\x})}{\nUnit})}}
                    {\El{\zero}{\nUnit}}
\]
and I'm a little nervous about making \rulename{tyeq-el} into a bidirectional
inference rule. Certainly if we want an algorithm, we can't have name-equality
in a universe just invoke the algorithm for type equality on the corresponding
$\El{}{}$ types \emph{and} have the algorithm for equality of $\El{}{}$ types
check for name-equality.


\end{document}
