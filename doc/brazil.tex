\documentclass{article}

\usepackage{times}
\usepackage{mathpartir}
\usepackage{amsmath,amsfonts}

%% Macros here

% Universe indices

\newcommand{\NN}{\mathbb{N}} % God-given numbers

\newcommand{\PP}{\mathbb{P}} % the poset of universe indices
\newcommand{\FF}{\mathbb{F}} % the subposet of fibered universe indices
\newcommand{\zero}{o} % this is not o, it is omicron

\newcommand{\piClose}[2]{\mathsf{piClose}(#1,#2)}   % the universe containing a pi with given domain and codomain
\newcommand{\uClose}[1]{\mathsf{uClose}(#1)}  % the universe containing the given universe's name.

% Generic entities
\newcommand{\G}{\Gamma} % a context
\newcommand{\T}{T} % a type
\newcommand{\U}{U} % another type
\newcommand{\x}{x} % a term variable
\newcommand{\y}{y} % another term variable
\newcommand{\e}{e} % a term

\newcommand{\rulename}[1]{\text{\textsc{#1}}}

%% Syntax
\newcommand{\bnf}{\mathrel{{:}{:}{=}}}
\newcommand{\bnfor}{\mid}

%% Syntactic constructs
\newcommand{\ctxempty}{\bullet} % empty context
\newcommand{\ctxextend}[3]{#1, #2 {:} #3} % extended context

\newcommand{\subst}[3]{#1[#3/#2]} % substitution

\newcommand{\Universe}[1]{\mathbb{U}_{#1}} % non-fibered universe
\newcommand{\El}[2]{\mathsf{El}^{#1} #2} % the type named by #2


\newcommand{\Unit}{\mathtt{Unit}} % The unit type
\newcommand{\Prod}[2]{{\textstyle\prod_{(#1 {:} #2)}}} % dependent product

\newcommand{\lam}[2]{\lambda #1 {:} #2 .\,} % $\lambda$-abstraction
\newcommand{\app}[2]{#1\,#2} % application

\newcommand{\unitTerm}{\star} % the element of the unit type

% names
\newcommand{\nUnit}{\mathtt{unit}} % name of unit type
\newcommand{\nProd}[3]{\pi^{#1} #2 {:} #3 \,.\,} % name of a dependent product
\newcommand{\nUniverse}[1]{u_{#1}}  % name of a universe

\newcommand{\coerce}[3]{\mathsf{coerce}^{#2{\mapsto}#3}\ #1}

% judgments
\newcommand{\isctx}[1]{#1\;\mathtt{ctx}} % well formed context
\newcommand{\istype}[2]{#1 \vdash #2\;\mathtt{type}} % well formed type
\newcommand{\isfib}[2]{#1 \vdash #2\;\mathtt{fibered}} % is a fibered type
\newcommand{\isterm}[3]{#1 \vdash\,#2\,:\,#3} % well formed term

\newcommand{\eqtype}[3]{#1 \vdash\, #2\, \equiv\, #3} % equal types
\newcommand{\eqterm}[4]{#1 \vdash\, #2\, \equiv #3\,:\,#4} % equal terms

\begin{document}

\title{Brazilian type theory}
\author{Andrej Bauer \and Christopher A. Stone}
\maketitle

\section{Universe indices}
\label{sec:universe-indices}

There is a pointed poset $(\PP, {\leq}, \zero)$ of \emph{universe
  indices} and a subset $\FF \subseteq \PP$ of \emph{fibered indices}.
For example, we may take
%
\begin{equation*}
  \PP = \{0,1\} \times \NN
\end{equation*}
%
with the lexicographic order, $\zero = (0,0)$ and $\FF = \{0\} \times
\NN$. We use lower Greek letters for universe indices.

We assume the following functions [or should they be partial functions?] are given:
\begin{itemize}
  \item $\piClose{\cdot}{\cdot} : \PP \times \PP \to \PP$
  \item $\uClose{\cdot}  : \PP \to \PP$
\end{itemize}
\section{Syntax}
\label{sec:syntax}

Context:
%
\begin{align*}
  \G \bnf
  \ctxempty \bnfor
  \ctxextend{\G}{\x}{\T}
\end{align*}
%
Types:
%
\begin{align*}
  \T, \U \bnf
  \Universe{\alpha} \bnfor
  \El{\alpha}{\e} \bnfor
  \Unit \bnfor
  \Prod{x}{\T} \U
\end{align*}
%
Terms:
%
\begin{align*}
  \e \bnf
  \x \bnfor
  \lam{\x}{\T} \e \bnfor
  \app{\e_1}{\e_2} \bnfor
  \unitTerm \bnfor
  \nUnit \bnfor
  \nProd{\alpha,\beta}{\x}{\e_1} \e_2 \bnfor
  \nUniverse{\alpha} \bnfor
  \coerce{\e}{\alpha}{\beta}
\end{align*}

\section{Judgments}
\label{sec:judgments}

\begin{align*}
& \isctx{\G} & & \text{$\G$ is a well formed context} \\
& \istype{\G}{\T} & & \text{$\T$ is a type in context $\G$} \\
& \isfib{\G}{\T} & & \text{$\T$ is a fibered type in context $\G$} \\
& \isterm{\G}{\e}{\T} & & \text{$\e$ is a well formed term of type $\T$ in context $\G$} \\
& \eqtype{\G}{\T}{\U} & & \text{$\T$ and $\U$ are equal types in context $\G$} \\
& \eqterm{\G}{\e_1}{\e_2}{\T} & & \text{$e_1$ and $e_2$ are equal terms of type $\T$ in context $\G$}
\end{align*}

\section{Contexts}
\label{sec:contexts}

\begin{mathpar}
  \infer[\rulename{ctx-empty}]
  { }
  {\isctx{\ctxempty}}

  \infer[\rulename{ctx-extend}]
  {\isctx{\G} \\
   \istype{\G}{\T}
  }
  {\isctx{\ctxextend{\G}{\x}{\T}}}
\end{mathpar}

\section{Types}
\label{sec:types}

\begin{mathpar}
  \infer[\rulename{ty-universe}]
  {\isctx{\G} \\
   \alpha \in \PP
  }
  {\istype{\G}{\Universe{\alpha}}}

  \infer[\rulename{ty-prod}]
  {\istype{\G}{\T} \\
   \istype{\ctxextend{\G}{\x}{\T}}{\U}
  }
  {\istype{\G}{\Prod{\x}{\T}{\U}}}

  \infer[\rulename{ty-el}]
  {\isterm{\G}{\e}{\Universe{\alpha}}}
  {\istype{\G}{\El{\alpha}{\e}}}

  \infer[\rulename{ty-unit}]
  {\isctx{\G}}
  {\istype{\G}{\Unit}}
\end{mathpar}

\section{Fibered types}
\label{sec:fibered-types}

\begin{mathpar}
  \infer[\rulename{fib-universe}]
  {\isctx{\G} \\
   \alpha \in \PP
  }
  {\isfib{\G}{\Universe{\alpha}}}

  \infer[\rulename{fib-prod}]
  {\isfib{\G}{\T} \\
   \isfib{\ctxextend{\G}{\x}{\T}}{\U}
  }
  {\isfib{\G}{\Prod{\x}{\T}{\U}}}

  \infer[\rulename{fib-el}]
  {\isterm{\G}{\e}{\Universe{\alpha}} \\
   \alpha \in \FF
  }
  {\isfib{\G}{\El{\alpha}{\e}}}

  \infer[\rulename{fib-unit}]
  {\isctx{\G} \\
   \zero\in\FF }
  {\istype{\G}{\Unit}}

\end{mathpar}

\section{Terms}
\label{sec:terms}

\begin{mathpar}
  \infer[\rulename{term-eq}]
  {\isterm{\G}{\e}{\T} \\
   \eqtype{\G}{\T}{\U}
  }
  {\isterm{\G}{\e}{\U}}

  \infer[\rulename{term-var}]
  {\isctx{\G} \\
   (\x{:}\T) \in \G
  }
  {\isterm{\G}{\x}{\T}}

  \infer[\rulename{term-abs}]
  {\isterm{\ctxextend{\G}{\x}{\T}}{\e}{\U}}
  {\isterm{\G}{(\lam{\x}{\T}{\e})}{\Prod{\x}{\T}{\U}}}

  \infer[\rulename{term-app}]
  {\isterm{\G}{\e_1}{\Prod{x}{\T} \U} \\
   \isterm{\G}{\e_2}{\T}
  }
  {\isterm{\G}{\app{\e_1}{\e_2}}{\subst{\U}{\x}{\e_2}}}

  \infer[\rulename{term-star}]
  {\isctx{\G}}
  {\isterm{\G}{\unitTerm}{\Unit}}

  \infer[\rulename{name-unit}]
  {\isctx{\G}}
  {\isterm{\G}{\nUnit}{\Universe{\zero}}}

  \infer[\rulename{name-prod}]
  {\isterm{\G}{\e_1}{\Universe{\alpha}} \\
   \isterm{\ctxextend{\G}{\x}{\El{\alpha}{\e_1}}}{\e_2}{\Universe{\beta}}
  }
  {\isterm{\G}{(\nProd{\alpha,\beta}{\x}{\e_1} \e_2)}{\Universe{\piClose{\alpha}{\beta}}}}

  \infer[\rulename{name-universe}]
  { }
  {\isterm{\G}{\nUniverse{\alpha}}{\Universe{\uClose{\alpha}}}}

  \infer[\rulename{term-coerce}]
  {\isterm{\G}{\e}{\Universe{\alpha}}\\
   \alpha \leq \beta}
  {\isterm{\G}{\coerce{\e}{\alpha}{\beta}}{\Universe{\beta}}}

\end{mathpar}

\section{Type Equality}

\begin{mathpar}
  \infer[\rulename{tyeq-refl}]
  {\istype{\G}{\T}}
  {\eqtype{\G}{\T}{\T}}

  \infer[\rulename{tyeq-sym}]
  {\eqtype{\G}{\U}{\T}}
  {\eqtype{\G}{\T}{\U}}

  \infer[\rulename{tyeq-trans}]
  {\eqtype{\G}{\T}{\T'}\\
   \eqtype{\G}{\T'}{\U}}
  {\eqtype{\G}{\T}{\U}}

  \infer[\rulename{tyeq-el}]
  {\eqterm{\G}{\e_1}{\e_2}{\Universe{\alpha}}}
  {\eqtype{\G}{\El{\alpha}{\e_1}}{\El{\alpha}{\e_2}}}

  \infer[\rulename{tyeq-prod}]
  {\eqtype{\G}{\T_1}{\U_1}\\
   \eqtype{\ctxextend{\G}{\x}{\T_1}}{\T_2}{\U_2}}
  {\eqtype{\G}{\Prod{\x}{\T_1}{\T_2}}{\Prod{\x}{\U_1}{\U_2}}}

  \infer[\rulename{tyeq-el-pi}]
  { }
  {\eqtype{\G}{\El{\piClose{\alpha}{\beta}}{(\nProd{\alpha,\beta}{\x}{\e_1}{\e_2})}}
              {\Prod{\x}{\El{\alpha}{\e_1}}{\El{\beta}{\e_2}}}}

  \infer[\rulename{tyeq-el-unit}]
  { }
  {\eqtype{\G}{\El{\zero}{\nUnit}}{\Unit}}

  \infer[\rulename{tyeq-el-coerce}]
  {\isterm{\G}{\e}{\Universe{\alpha}} }
  {\eqtype{\G}{\El{\beta}(\coerce{\e}{\alpha}{\beta})}
              {\El{\alpha}{e}}}

\end{mathpar}

\section{Term Equality}

\begin{mathpar}

% PER rules

  \infer[\rulename{eq-refl}]
  {\isterm{\G}{\e}{\T}}
  {\eqterm{\G}{\e}{\e}{\T}}

  \infer[\rulename{eq-sym}]
  {\eqterm{\G}{\e_2}{\e_1}{\T}}
  {\eqterm{\G}{\e_1}{\e_2}{\T}}

  \infer[\rulename{eq-trans}]
  {\eqterm{\G}{\e_1}{\e_2}{\T}\\
   \eqterm{\G}{\e_2}{\e_3}{\T}}
  {\eqterm{\G}{\e_1}{\e_3}{\T}}

  \infer[\rulename{eq-eq}]
  {\eqterm{\G}{\e_1}{\e_2}{\T}\\
    \eqtype{\G}{\T}{\U}}
  {\eqterm{\G}{\e_1}{\e_2}{\U}}

  \infer[\rulename{eq-abs}]
  {\eqtype{\G}{\T_1}{\U_1}\\
    \eqterm{\ctxextend{\G}{\x}{\T_1}}{\e_1}{\e_2}{\T_2}}
  {\eqterm{\G}{(\lam{\x}{\T_1}{\e_1})}
              {(\lam{\x}{\U_1}{\e_2})}
              {\Prod{\x}{\T_1}{\T_2}}}

  \infer[\rulename{eq-app}]
  {\eqterm{\G}{\e_1}{\e'_1}{\Prod{\x}{\T}{\U}}\\
   \eqterm{\G}{\e_2}{\e'_2}{\T}}
  {\eqterm{\G}{\app{\e_1}{\e_2}}{\app{\e'_1}{\e'_2}}{\subst{\U}{\x}{\e_2}}}

  \infer[\rulename{eq-beta}]
  {\isterm{\ctxextend{\G}{\x}{\T}}{e_1}{\U}\\
    \isterm{\G}{\e_2}{\T}}
  {\eqterm{\G}{\app{(\lam{\x}{\T}{\e_1})}{\e_2}}
              {\subst{\e_1}{\x}{\e_2}}
              {\subst{\U}{\x}{\e_2}}}

  \infer[\rulename{eq-prod}]
  {\eqterm{\G}{\e_1}{\e'_1}{\Universe{\alpha}}\\
   \eqterm{\ctxextend{\G}{\x}{\El{\alpha}{\e_1}}}{\e_2}{\e'_2}{\Universe{\beta}}}
  {\eqterm{\G}{\nProd{\alpha,\beta}{\x}{\e_1}{\e_2}}{\nProd{\alpha,\beta}{\x}{\e'_1}{\e'_2}}{\Universe{\piClose{\alpha}{\beta}}}}

  \infer[\rulename{eq-coerce}]
  {\eqterm{\G}{\e_1}{\e_2}{\Universe{\alpha}}}
  {\eqterm{\G}{(\coerce{\e_1}{\alpha}{\beta})}
              {(\coerce{\e_2}{\alpha}{\beta})}
              {\Universe{\beta}}}

  \infer[\rulename{eq-coerce-refl}]
  {\isterm{\G}{\e}{\Universe{\alpha}}}
  {\eqterm{\G}{(\coerce{\e}{\alpha}{\alpha})}{e}{\Universe{\alpha}}}

  \infer[\rulename{eq-coerce-trans}]
  {\isterm{\G}{\e}{\Universe{\alpha}} \\
    \alpha \leq \gamma \leq \beta}
  {\eqterm{\G}{\coerce{(\coerce{\e}{\alpha}{\gamma}}{\gamma}{\beta})}
              {\coerce{\e}{\alpha}{\beta}}
              {\Universe{\beta}}}

\end{mathpar}
\bigskip

Question: Assuming that $\piClose{0}{1} = \piClose{1}{1} = 1$, should we be able to prove that
\[ \eqterm{\ctxempty}{(\nProd{0,1}{\_}{\nUnit}{\nUniverse{1}})}{(\nProd{1,1}{\_}{(\coerce{\nUnit}{0}{1})}{\nUniverse{1}})}
                     {\Universe{1}}  ? \]

This would follow if we had a reflection rule
\[
  \infer
   {\eqtype{\G}{\El{\alpha}{\e_1}}{\El{\alpha}{\e_2}}}
   {\eqterm{\G}{\e_1}{\e_2}{\Universe{\alpha}}}
\]
but we definitely need Rule~\rulename{tyeq-el}, for example, to prove that
\[ \eqtype{\ctxempty}
          {\El{\zero}{(\app{(\lam{\x}{\Universe{\zero}}{\x})}{\nUnit})}}
                    {\El{\zero}{\nUnit}}
\]
and I'm a little nervous (algorithmically) about making this a bidirectional inference rule.


\end{document}
