\documentclass{article}

\usepackage{amsmath,amssymb}
\usepackage{mathpartir}
\usepackage[utf8]{inputenc}
% Universe indices

\newcommand{\NN}{\mathbb{N}} % God-given numbers

% Generic entities
\newcommand{\G}{\Gamma} % a context
\newcommand{\D}{\Delta} % another context
\newcommand{\T}{T} % a type
\newcommand{\U}{U} % another type
\newcommand{\x}{x} % a term variable
\newcommand{\y}{y} % another term variable
\newcommand{\z}{z} % a third term variable
\newcommand{\e}{e} % a term
\newcommand{\cmd}{C} % a command
\newcommand{\expr}{E} % an expressions

\newcommand{\rulename}[1]{\text{\textsc{#1}}}

%% Syntax
\newcommand{\bnf}{\ \mathrel{{:}{:}{=}}\ }
\newcommand{\bnfor}{\ \mid\ \ }

%% Syntactic constructs
\newcommand{\ctxempty}{\bullet} % empty context
\newcommand{\ctxextend}[3]{#1,\, #2\, {:}\, #3} % extended context

\newcommand{\subst}[3]{#1[#3/#2]} % substitution
\newcommand{\substs}[2]{#1[#2]} % substitution of many variables

\newcommand{\Type}{\mathsf{Type}} % The type of all types
\newcommand{\Prod}[2]{\mathop{\textstyle\prod_{(#1 {:} #2)}}} % dependent product

\newcommand{\lam}[3]{\lambda #1 {:} #2.{#3}\,.\,} % $\lambda$-abstraction
\newcommand{\app}[5]{#1\mathbin{@^{#2{:}#3.#4}} #5} % application

\newcommand{\abst}[2]{[#1 \,.\, #2]} % abstraction
\newcommand{\ascribe}[2]{#1 \,{:}{:}\, #2} % type ascription

\newcommand{\JuEqual}[3]{\mathsf{Eq}_{#1}(#2,#3)} % Equality type
\newcommand{\juRefl}[1]{{\mathsf{refl}_{#1}}\ }    % Judgmental refl

% orthodox judgments
\newcommand{\isctx}[1]{#1\ \mathsf{ctx}} % well formed context
\newcommand{\istype}[2]{\isterm{#1}{#2}{\Type}} % well formed type
\newcommand{\isterm}[3]{#1 \vdash\,#2\,:\,#3} % well formed term

\newcommand{\eqtype}[3]{\eqterm{#1}{#2}{#3}{\Type}} % equal types
\newcommand{\eqterm}[4]{#1 \vdash #2 \equiv #3 : #4} % equal terms

% commands

\newcommand{\cmdLam}[2]{\lambda #1 {:} #2\,.\,} % $\lambda$-abstraction untagged return type
\newcommand{\cmdLamCurry}[1]{\lambda #1 \,.\,} % $\lambda$-abstraction completely untagged
\newcommand{\cmdApp}[2]{#1\,#2} % application untagged
\newcommand{\cmdType}{\mathsf{Type}} % The type of all types
\newcommand{\cmdProd}[2]{\mathop{\textstyle\prod_{(#1 {:} #2)}}} % dependent product
\newcommand{\cmdAscribe}[2]{#1 \,{:}{:}\, #2} % type ascription
\newcommand{\cmdLet}[2]{\mathsf{let}\;#1 \mathbin{{:}{=}} #2\; \mathsf{in}\;} % let-binding
\newcommand{\cmdEq}[2]{\mathsf{Eq}(#1,#2)} % Equality type
\newcommand{\cmdRefl}{\mathsf{refl}\;}    % Judgmental refl
\newcommand{\cmdReturn}{\mathsf{return}\;}    % Judgmental refl


% algorithmic judgments
\newcommand{\env}{\eta}
\newcommand{\envextend}[4]{#1(#2) \mathbin{{:}{=}} (#3, #4)} % extended context
\newcommand{\ctxenv}{\G;\, \env} % A combined context and environment

\newcommand{\evalc}[4]{#1 \vdash #2 \Leftarrow #3 \mapsto #4} % checking
\newcommand{\evali}[4]{#1 \vdash #2 \Rightarrow (#3, #4)} % inference

\newcommand{\ishints}[2]{#1 \vdash #2 \; \mathsf{hints}} % well-formed context and hints

\newcommand{\istypealg}[2]{#1 \vdash #2 \Leftarrow \mathsf{type}} % well formed type
\newcommand{\isfibalg}[2]{#1 \vdash #2 \Leftarrow \mathsf{fibered}} % is a fibered type
\newcommand{\eqtypealg}[3]{#1 \vdash #2 \thickapprox #3} % equal types
\newcommand{\eqtypepath}[3]{#1 \vdash #2 \thicksim #3} % equal normal-form types
\newcommand{\eqtermalg}[4]{#1 \vdash #2 \thickapprox #3 \Leftarrow #4} % equal terms of normalized type
\newcommand{\eqtermext}[4]{#1 \vdash #2 \simeq #3 \Leftarrow #4} % equal terms w/o eta
\newcommand{\eqpath}[4]{#1 \vdash #2 \thicksim #3}  % equal paths

\newcommand{\equationin}[3]{\mathsf{equation}\; #1 \,{:}\, #2 \,{\equiv}\, #3 \; \mathsf{in} \;} % use equation hint
\newcommand{\rewritein}[3]{\mathsf{rewrite}\; #1 \,{:}\, #2 \,{\equiv}\, #3 \; \mathsf{in} \;} % use rewrite hint
\newcommand{\eqhint}[2]{(#1 \,{\equiv}\, #2)} % equality hint
\newcommand{\rwhint}[2]{(#1 \,{\leadsto}\, #2)} % rewrite hint

\renewcommand{\H}{\mathcal{H}}      % term hint list
\newcommand{\hintempty}{\circ}      % empty hint list

\newcommand{\addhinteq}[3]{#1, \eqhint{#2}{#3}} % extended hint list
\newcommand{\addhintrw}[3]{#1, \rwhint{#2}{#3}} % extended hint list
\newcommand{\ctxs}[2]{#1\,;\,#2}
\newcommand{\GH}{\ctxs{\G}{\H}}           % combined context and hints

% normalization

\newcommand{\tywhnf}[4]{#1 \vdash #2 \leadsto #3 / #4} % "one-step type normalization"
\newcommand{\whnf}[4]{#1 \vdash #2 \leadsto #3 / #4} % "one-step term normalization"

\newcommand{\tywhnfs}[3]{#1 \vdash #2 \leadsto^* #3 \not\leadsto } % "type normalization"
\newcommand{\whnfs}[3]{#1 \vdash #2 \leadsto^* #3 \not\leadsto } % "term normalization"

\newcommand{\inferred}[1]{{\color{magenta}{#1}}}

\newcommand{\simplefib}{\mathop{\mathsf{isfib}}\,}
\newcommand{\true}{\mathsf{true}}
\newcommand{\false}{\mathsf{false}}

\newcommand{\nameof}{{\mathsf{name\_of}}\,}
\newcommand{\typeof}[2]{{\mathsf{type\_of}}(#1,#2)}

\newcommand{\tynorm}[2]{#1 \hookrightarrow #2} % (strong) normalization of terms
\newcommand{\norm}[2]{#1 \hookrightarrow #2} % (strong) normalization of terms


\begin{document}
\title{Andromedan type theory}
\author{
Andrej Bauer \\ University of Ljubljana
\and
Philipp Haselwarter \\ University of Ljubljana
\and
Matija Pretnar \\ University of Ljubljana
\and
Christopher A.~Stone \\ Harvey Mudd College}

\maketitle

\begin{abstract}
  An outline of the type theory implemented by Andromeda.
\end{abstract}

\section{The declarative formulation}
\label{sec:declarative-formulation}

In this section we give the formulation of type theory in a declarative way
which minimizes the number of judgments, is better suited for a semantic account, but is
not susceptible to an algorithmic treatment.

\subsection{Syntax}
\label{sec:syntax}

Contexts:
%
\begin{equation*}
  \G
  \begin{aligned}[t]
    \bnf   {}& \ctxempty & & \text{empty context}\\
    \bnfor {}& \ctxextend{\G}{\x}{\T} & & \text{context $\G$ extended with $\x : \T$}
  \end{aligned}
\end{equation*}
%
Terms ($\e$) and types $(\T, \U)$:
%
\begin{equation*}
  \e, \T, \U
  \begin{aligned}[t]
    \bnf   {}& \Type & & \text{universe}\\
    \bnfor {}& \Prod{x}{\T} \U & & \text{product}\\
    \bnfor {}& \JuEqual{\T}{\e_1}{\e_2} & & \text{equality type} \\
    \bnfor {}&  \x   &&\text{variable} \\
    \bnfor {}&  \lam{\x}{\T_1}{\T_2} \e  &&\text{$\lambda$-abstraction} \\
    \bnfor {}&  \app{\e_1}{\x}{\T_1}{\T_2}{\e_2}  &&\text{application} \\
    \bnfor {}&  \juRefl{\T} \e  &&\text{reflexivity} \\
  \end{aligned}
\end{equation*}
%
Note that $\lambda$-abstraction and application are tagged with extra types not usually
seen in type theory. An abstraction $\lam{\x}{\T_1}{\T_2} \e$ speficies not only the type
$\T_1$ of $\x$ but also the type $\T_2$ of $e$, where $\x$ is bound in $\T_2$ and $\e$.
Similarly, an application $\app{\e_1}{\x}{\T_1}{\T_2}{\e_2}$ specifies that $\e_1$ and
$\e_2$ have types $\Prod{\x}{\T_1} \T_2$ and $\T_2$, respectively. This is necessary
because in the presence of exotic equalities (think ``$\mathsf{nat} \to \mathsf{bool}
\equiv \mathsf{nat} \to \mathsf{nat}$'') we must be \emph{very} careful about
$\beta$-reductions.

The annotations on an application matter also for determining when two
terms are equal. For example, if $X,Y : \Type$, $f : \mathsf{nat}\to X$ and $$e
: \JuEqual{\Type}{\mathsf{nat}{\to}X}{\mathsf{nat}{\to}Y},$$ then
$(\app{f}{\_}{\mathsf{nat}}{X} 0) : X$ and $(\app{f}{\_}{\mathsf{nat}}{Y} 0) : Y$,
so the two identical-but-for-annotations terms have different types and thus
cannot be equivalent.

\subsection{Judgments}
\label{sec:judgments}

\begin{align*}
& \isctx{\G} & & \text{$\G$ is a well formed context} \\
& \isterm{\G}{\e}{\T} & & \text{$\e$ is a well formed term of type $\T$ in context $\G$} \\
& \eqterm{\G}{\e_1}{\e_2}{\T} & & \text{$e_1$ and $e_2$ are equal terms of type $\T$ in context $\G$}
\end{align*}
%
The judgement ``$\T$ is a type in context $\G$'' is a special case of term formation, namely
$\istype{\G}{\T}$. Similarly, equality of terms at $\Type$ is equality of types.

\subsection{Contexts}
\label{sec:contexts}

\begin{mathpar}
  \infer[\rulename{ctx-empty}]
  { }
  {\isctx{\ctxempty}}

  \infer[\rulename{ctx-extend}]
  {\isctx{\G} \\
   \istype{\G}{\T}
  }
  {\isctx{\ctxextend{\G}{\x}{\T}}}
\end{mathpar}

\subsection{Terms and types}

\paragraph{General rules}
\begin{mathpar}
  \infer[\rulename{term-conv}]
  {\isterm{\G}{\e}{\T} \\
   \eqtype{\G}{\T}{\U}
  }
  {\isterm{\G}{\e}{\U}}

  \infer[\rulename{term-var}]
  {\isctx{\G} \\
   (\x{:}\T) \in \G
  }
  {\isterm{\G}{\x}{\T}}
\end{mathpar}

\paragraph{Universe}

\begin{mathpar}
  \infer[\rulename{ty-type}]
  {\isctx{\G}
  }
  {\istype{\G}{\Type}}
\end{mathpar}

\paragraph{Products}

\begin{mathpar}
  \infer[\rulename{ty-prod}]
  {\istype{\G}{\T} \\
   \istype{\ctxextend{\G}{\x}{\T}}{\U}
  }
  {\istype{\G}{\Prod{\x}{\T}{\U}}}

  \infer[\rulename{term-abs}]
  {\isterm{\ctxextend{\G}{\x}{\T}}{\e}{\U}}
  {\isterm{\G}{(\lam{\x}{\T}{\U}{\e})}{\Prod{\x}{\T}{\U}}}

  \infer[\rulename{term-app}]
  {\isterm{\G}{\e_1}{\Prod{x}{\T} \U} \\
   \isterm{\G}{\e_2}{\T}
  }
  {\isterm{\G}{\app{\e_1}{\x}{\T}{\U}{\e_2}}{\subst{\U}{\x}{\e_2}}}
\end{mathpar}

\paragraph{Equality types}
\label{sec:equality}

\begin{mathpar}
  \infer[\rulename{ty-eq}]
  {\istype{\G}{\T}\\
   \isterm{\G}{\e_1}{\T}\\
   \isterm{\G}{\e_2}{\T}
  }
  {\istype{\G}{\JuEqual{\T}{\e_1}{\e_2}}}

  \infer[\rulename{term-refl}]
  {\isterm{\G}{\e}{\T}}
  {\isterm{\G}{\juRefl{\T} \e}{\JuEqual{\T}{\e}{\e}}}
  \end{mathpar}

\subsection{Equality}

\paragraph{General rules}

\begin{mathpar}
  \infer[\rulename{eq-refl}]
  {\isterm{\G}{\e}{\T}}
  {\eqterm{\G}{\e}{\e}{\T}}

  \infer[\rulename{eq-sym}]
  {\eqterm{\G}{\e_2}{\e_1}{\T}}
  {\eqterm{\G}{\e_1}{\e_2}{\T}}

  \infer[\rulename{eq-trans}]
  {\eqterm{\G}{\e_1}{\e_2}{\T}\\
   \eqterm{\G}{\e_2}{\e_3}{\T}}
  {\eqterm{\G}{\e_1}{\e_3}{\T}}

  \infer[\rulename{eq-conv}]
  {\eqterm{\G}{\e_1}{\e_2}{\T}\\
    \eqtype{\G}{\T}{\U}}
  {\eqterm{\G}{\e_1}{\e_2}{\U}}
\end{mathpar}

\paragraph{Equality reflection}
%
\begin{mathpar}
  \infer[\rulename{eq-reflection}]
  {\isterm{\G}{\e}{\JuEqual{\T}{\e_1}{\e_2}}}
  {\eqterm{\G}{\e_1}{\e_2}{\T}}
\end{mathpar}

\paragraph{Computations}

\begin{mathpar}
\infer[\rulename{prod-beta}]
  {\eqtype{\G}{\T_1}{\U_1}\\
    \eqtype{\ctxextend{\G}{\x}{\T_1}}{\T_2}{\U_2}\\\\
    \isterm{\ctxextend{\G}{\x}{\T_1}}{\e_1}{\T_2}\\
    \isterm{\G}{\e_2}{\U_1}}
  {\eqterm{\G}{\bigl(\app{(\lam{\x}{\T_1}{\T_2}{\e_1})}{\x}{\U_1}{\U_2}{\e_2}\bigr)}
              {\subst{\e_1}{\x}{\e_2}}
              {\subst{\T_2}{\x}{\e_2}}}
\end{mathpar}

\paragraph{Extensionality}

%
\begin{mathpar}
  \infer[\rulename{eq-eta}]
  {\isterm{\G}{\e'_1}{\JuEqual{\T}{\e_1}{\e_2}} \\
    \isterm{\G}{\e'_2}{\JuEqual{\T}{\e_1}{\e_2}}
  }
  {\eqterm{\G}{\e'_1}{e'_2}{\JuEqual{\T}{\e_1}{\e_2}}}

  \infer[\rulename{prod-eta}]
  {\isterm{\G}{\e_1}{\Prod{\x}{\T}{\U}}\\
   \isterm{\G}{\e_2}{\Prod{\x}{\T}{\U}}\\\\
   \eqterm{\ctxextend{\G}{\x}{\T}}{(\app{\e_1}{\x}{\T}{\U}{\x})}
          {(\app{\e_2}{\x}{\T}{\U}{\x})}{\U}
  }
  {\eqterm{\G}{\e_1}{\e_2}{\Prod{\x}{\T}{\U}}}
\end{mathpar}

\subsubsection{Congruences}

\paragraph{Type formers}

\begin{mathpar}
  \infer[\rulename{cong-prod}]
  {\eqtype{\G}{\T_1}{\U_1}\\
   \eqtype{\ctxextend{\G}{\x}{\T_1}}{\T_2}{\U_2}}
  {\eqtype{\G}{\Prod{\x}{\T_1}{\T_2}}{\Prod{\x}{\U_1}{\U_2}}}

  \infer[\rulename{cong-eq}]
  {\eqtype{\G}{\T}{\U}\\
   \eqterm{\G}{\e_1}{\e'_1}{\T}\\
   \eqterm{\G}{\e_2}{\e'_2}{\T}
  }
  {\eqtype{\G}{\JuEqual{\T}{\e_1}{\e_2}}
              {\JuEqual{\U}{\e'_1}{\e'_2}}}
\end{mathpar}

\paragraph{Products}

\begin{mathpar}
  \infer[\rulename{cong-abs}]
  {\eqtype{\G}{\T_1}{\U_1}\\
    \eqtype{\ctxextend{\G}{\x}{\T_1}}{\T_2}{\U_2}\\
    \eqterm{\ctxextend{\G}{\x}{\T_1}}{\e_1}{\e_2}{\T_2}}
  {\eqterm{\G}{(\lam{\x}{\T_1}{\T_2}{\e_1})}
              {(\lam{\x}{\U_1}{\U_2}{\e_2})}
              {\Prod{\x}{\T_1}{\T_2}}}

  \infer[\rulename{cong-app}]
  {\eqtype{\G}{\T_1}{\U_1}\\
   \eqtype{\ctxextend{\G}{\x}{\T_1}}{\T_2}{\U_2}\\\\
   \eqterm{\G}{\e_1}{\e'_1}{\Prod{\x}{\T_1}{\T_2}}\\
   \eqterm{\G}{\e_2}{\e'_2}{\T_1}}
  {\eqterm{\G}{(\app{\e_1}{\x}{\T_1}{\T_2}{\e_2})}{(\app{\e'_1}{\x}{\U_1}{\U_2}{\e'_2})}{\subst{\T_2}{\x}{\e_2}}}
\end{mathpar}

\paragraph{Equality types}

%
\begin{mathpar}
\infer[\rulename{cong-refl}]
{\eqterm{\G}{\e_1}{\e_2}{\T}\\
 \eqtype{\G}{\T}{\U}}
{\eqterm{\G}{\juRefl{\T} \e_1}{\juRefl{\U} \e_2}{\JuEqual{\T}{\e_1}{\e_1}}}
\end{mathpar}

%%%%%%%%%%%%%%%%%%%%%%%%%%%%%%%%%%%%%%%%%%%%%%%%%%%%%%%%%

\section{Algorithmic version}

Andromeda should be thought of as a programming language for deriving judgments. At the moment the language is untyped. We can hope to have it \emph{simply typed} one day.

\subsection{Syntax} % (fold)
\label{sub:prog-syntax}

Expressions:
%
\begin{equation*}
  \expr
  \begin{aligned}[t]
    \bnf   {}& \cmdType & & \text{universe}\\
    \bnfor {}& \x   &&\text{variable} \\
  \end{aligned}
\end{equation*}
%
Computations:
%
\begin{equation*}
  \cmd
  \begin{aligned}[t]
    \bnf   {}& \cmdReturn \expr              &&\text{pure expressions} \\
    \bnfor {}& \cmdLet{\x}{\cmd_1} \cmd_2    &&\text{let binding} \\
    \bnfor {}& \cmdAscribe{\cmd}{\expr}      &&\text{ascription} \\
    \bnfor {}& \cmdProd{x}{\expr} \cmd       &&\text{product}\\
    \bnfor {}& \cmdEq{\cmd}{\cmd}            &&\text{equality type} \\
    \bnfor {}& \cmdLam{\x}{\expr} \cmd       &&\text{$\lambda$-abstraction} \\
    \bnfor {}& \cmdApp{\expr}{\cmd}          &&\text{application} \\
    \bnfor {}& \cmdRefl \cmd                 &&\text{reflexivity}
  \end{aligned}
\end{equation*}

The result of a computation is a value, which is a pair $(e,T)$ where $e$ and $T$ are terms of type theory, as described in Section~\ref{sec:syntax}. The correctness guarantee which we want is that a computation only ever evaluates to derivable judgments.

% subsection prog-syntax (end)

\subsubsection{Operational semantics} % (fold)
\label{ssub:operational_semantics}

Operational semantics is given by \emph{two} versions of evaluation of computations, called \emph{inference} and \emph{checking}, of the forms:
%
\begin{align*}
  \text{Inference:}&\quad \evali{\ctxenv}{\cmd}{\e}{\T} \\
  \text{Checking:}&\quad  \evalc{\ctxenv}{\cmd}{\T}{\e}
\end{align*}
%
These are read as ``in the given context $\G$ and environment $\env$ command $\cmd$ infers that $\e$ has type $\T$'' and ``in the given context $\G$ and environment $\env$ command $\cmd$ checks that $\e$ has the given type $\T$.''

[EXPLAIN THAT AN ENVIRONMENT MAPS VARIABLES TO VALUES.]

\begin{mathpar}

  \infer[\rulename{check-infer}]
  {\evali{\ctxenv}{\cmd}{\e}{\U} \\
    \eqtypealg{\ctxenv}{\T}{\U}}
  {\evalc{\ctxenv}{\cmd}{\T}{\e}}

  \infer[\rulename{infer-type}]
  {}
  {\evali{\ctxenv}{\cmdReturn \Type}{\Type}{\Type}}

  \infer[\rulename{infer-product}]
  {\evalc \ctxenv \expr \Type {\U_1} \\
    \evalc {\ctxextend \G \x {\T_1};\, \env} \cmd \Type {\U_2}}
  {\evali \ctxenv {\cmdProd \x \expr \cmd} {\Prod \x {\U_1} {\U_2}} \Type}

  \infer[\rulename{infer-eq}]
  {\evali \ctxenv {\cmd_1} {\e_1} \T \\
    \evalc \ctxenv {\cmd_2} \T {\e_2}}
  {\evali \ctxenv {\cmdEq {\cmd_1} {\cmd_2}} {\JuEqual {\T} {\e_1} {\e_2}} \Type}

  \infer[\rulename{infer-ascription}]
  {\evalc \ctxenv \expr \Type \T \\
    \evalc \ctxenv \cmd \T \e}
  {\evali \ctxenv {\cmdAscribe \cmd \expr} \e \T}

\end{mathpar}

Inference of $\cmdEq{\cmd_1} {\cmd_2}$ keeps the type of the first argument.

Ascription only has an infer rule, and will always switch to a checking phase. It breaks the inward information flow and has to use the check-infer when encountered during checking.

\begin{mathpar}

  \infer[\rulename{infer-var}]
  {\env(\x) = (\e, \T)}
  {\evali{\ctxenv}{\cmdReturn \x}{\e}{\T}}

  % supposedly the user annotated the lambda, so we should use the type \T_1
  % which we get out of \expr in the recursive call, instead of \U_1
  \infer[\rulename{check-$\lambda$-tagged}]
  {\tywhnfs \ctxenv \U {\Prod \x {\U_1} {\U_2}} \\
    \evalc \ctxenv \expr \Type {\T_1} \\
    \eqtypealg \ctxenv {\T_1} {\U_1} \\
    \evalc {\ctxextend \G \x {\T_1};\, \env} \cmd {\U_2} \e}
  {\evalc \ctxenv {\cmdLam \x \expr \cmd} \U {\lam \x {\U_1} {\U_2} \e}}
  % XXX should the result be (x:U1)->U2 or (x:T1->U2)?

  \infer[\rulename{check-$\lambda$-untagged}]
  {\tywhnfs \ctxenv \U {\Prod \x {\U_1} {\U_2}} \\
    \evalc {\ctxextend \G \x {\U_1};\, \env} \cmd {\U_2} \e}
  {\evalc \ctxenv {\cmdLamCurry \x \cmd} \U {\lam \x {\U_1} {\U_2} \e}}

  \infer[\rulename{infer-$\lambda$}]
  {\evalc \ctxenv \expr \Type {\U_1} \\
    \evali {\ctxextend \G \x {\U_1};\, \env} \cmd \e {\U_2}}
  {\evali \ctxenv {\cmdLam \x \expr \cmd} {\lam \x {\U_1} {\U_2} \e} {\Prod \x {\U_1} {\U_2}}}

  \infer[\rulename{infer-app}]
  {\evali{\ctxenv}{\expr}{\e_1}{\T} \\
   \tywhnfs{\ctxenv}{\T}{\Prod{\x}{\U_1} \U_2} \\
   \evalc{\ctxenv}{\cmd}{\U_1}{\e_2}
  }
  {\evali
    {\ctxenv}
    {\cmdApp{\expr}{\cmd}}
    {\app{\e_1}{\x}{\U_1}{\U_2}{\e_2}}
    {\subst{\U_2}{\x}{\e_2}}
  }

  \infer[\rulename{check-app-non-dep}]
  {\evali \ctxenv \cmd {\e_2} \U \\
    \evalc \ctxenv \expr {\Prod \_ \U \T} {\e_1}}
  {\evalc \ctxenv {\cmdApp \expr \cmd} \T
    {\app{\e_1} \_ \U \T {\e_2}}}

  \infer[\rulename{infer-refl}]
  {\evali{\ctxenv}{\cmd}{\e}{\T}}
  {\evali{\ctxenv}{\cmdRefl \cmd}{\juRefl{\T}{\e}}{\JuEqual{\T}{\e}{\e}}}

  \infer[\rulename{check-refl}]
  {\tywhnfs{\ctxenv}{\T}{\JuEqual{\U}{\e_1}{\e_2}} \\
   \evalc{\ctxenv}{\cmd}{\U}{\e} \\
   \eqtermalg{\ctxenv}{\e}{\e_1}{\U} \\
   \eqtermalg{\ctxenv}{\e}{\e_2}{\U}
 }
  {\evalc{\ctxenv}{\cmdRefl \cmd}{\T}{\juRefl \T \e}}

  \infer[\rulename{check-let}]
  {\evali \ctxenv {\cmd_1} {\e_1} \U \\
  \evalc {\G;\, \envextend \env \x \e \U} {\cmd_2} \T {\e_2}}
  {\evalc \ctxenv {\cmdLet \x {\cmd_1} \cmd_2} \T {\e_2}}

  \infer[\rulename{infer-let}]
  {\evali \ctxenv {\cmd_1} {\e_1} \U \\
  \evali {\G;\, \envextend \env \x \e \U} {\cmd_2} {\e_2} \T}
  {\evali \ctxenv {\cmdLet \x {\cmd_1} \cmd_2} {\e_2} \T}

\end{mathpar}

TODO: check the freshness (and other side-conditions?).

% subsubsection operational_semantics (end)



\end{document}