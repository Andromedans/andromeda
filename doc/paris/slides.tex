\documentclass[14pt]{beamer}
\usepackage[utf8]{inputenc}

% Universe indices

\newcommand{\NN}{\mathbb{N}} % God-given numbers

\newcommand{\PP}{\mathbb{P}} % the poset of universe indices
\newcommand{\FF}{\mathbb{F}} % the subposet of fibered universe indices
\newcommand{\zero}{o} % this is not o, it is omicron

\newcommand{\piClose}[2]{\mathsf{max}(#1,#2)}   % the universe containing a pi with given domain and codomain
\newcommand{\uClose}[1]{\mathsf{succ}(#1)}  % the universe containing the given universe's name.

% Generic entities
\newcommand{\G}{\Gamma} % a context
\newcommand{\D}{\Delta} % another context
\newcommand{\T}{T} % a type
\newcommand{\U}{U} % another type
\newcommand{\x}{x} % a term variable
\newcommand{\y}{y} % another term variable
\newcommand{\z}{z} % a third term variable
\newcommand{\e}{e} % a term

\newcommand{\rulename}[1]{\text{\textsc{#1}}}

%% Syntax
\newcommand{\bnf}{\ \mathrel{{:}{:}{=}}\ }
\newcommand{\bnfor}{\ \mid\ \ }

%% Syntactic constructs
\newcommand{\ctxempty}{\bullet} % empty context
\newcommand{\ctxextend}[3]{#1,\, #2\, {:}\, #3} % extended context

\newcommand{\subst}[3]{#1[#3/#2]} % substitution
\newcommand{\substs}[2]{#1[#2]} % substitution of many variables

\newcommand{\Universe}[1]{\mathbb{U}_{#1}} % non-fibered universe
\newcommand{\El}[2]{\mathsf{El}^{#1}\, #2} % the type named by the term

\newcommand{\Unit}{\mathsf{Unit}} % The unit type
\newcommand{\Prod}[2]{\mathop{\textstyle\prod_{(#1 {:} #2)}}} % dependent product

\newcommand{\lam}[3]{\lambda #1 {:} #2.{#3}\,.\,} % $\lambda$-abstraction
\newcommand{\app}[5]{#1\mathbin{@^{#2{:}#3.#4}} #5} % application

\newcommand{\abst}[2]{[#1 \,.\, #2]} % abstraction
\newcommand{\ascribe}[2]{#1 \,{:}{:}\, #2} % type ascription

\newcommand{\unitTerm}{\star} % the element of the unit type

\newcommand{\coerce}[3]{\mathsf{coerce}^{#1{\mapsto}#2} \, #3}

\newcommand{\PrEqual}[3]{\mathsf{Paths}_{#1}(#2,#3)} % Propositional Equality type
\newcommand{\JuEqual}[3]{\mathsf{Id}_{#1}(#2,#3)} % Judgmental Equality type

\newcommand{\PrElim}[6]{\mathsf{J}_{#1}(#2, #3, #4, #5, #6)} % Propositional equality eliminator

\newcommand{\prRefl}[1]{{\mathsf{idpath}_{#1}}\ }  % Propositional refl
\newcommand{\juRefl}[1]{{\mathsf{refl}_{#1}}\ }    % Judgmental refl

% names
\newcommand{\nUnit}{\mathsf{unit}} % name of unit type
\newcommand{\nProd}[4]{\pi^{#1,#2} #3\,{:}\,#4 \,.\ } % name of a dependent product
\newcommand{\nUniverse}[1]{u_{#1}}  % name of a universe
\newcommand{\nPrEqual}[4]{\mathsf{paths}^{#1}_{#2}(#3,#4)} % Propositional Equality type
\newcommand{\nJuEqual}[4]{\mathsf{id}^{#1}_{#2}(#3,#4)} % Judgmental Equality type


% orthodox judgments
\newcommand{\isctx}[1]{#1\;\mathsf{ctx}} % well formed context
\newcommand{\istype}[2]{#1 \vdash #2\;\mathsf{type}} % well formed type
\newcommand{\isfib}[2]{#1 \vdash #2\;\mathsf{fibered}} % is a fibered type
\newcommand{\isterm}[3]{#1 \vdash\,#2\,:\,#3} % well formed term

\newcommand{\eqtype}[3]{#1 \vdash #2 \equiv #3} % equal types
\newcommand{\eqterm}[4]{#1 \vdash #2 \equiv #3 : #4} % equal terms

% algorithmic judgments

\newcommand{\chkterm}[3]{#1 \vdash #2 \Leftarrow #3} % checking
\newcommand{\synterm}[3]{#1 \vdash #2 \Rightarrow #3} % synthesis

\newcommand{\ishints}[2]{#1 \vdash #2 \; \mathsf{hints}} % well-formed context and hints

\newcommand{\istypealg}[2]{#1 \vdash #2 \Leftarrow \mathsf{type}} % well formed type
\newcommand{\isfibalg}[2]{#1 \vdash #2 \Leftarrow \mathsf{fibered}} % is a fibered type
\newcommand{\eqtypealg}[3]{#1 \vdash #2 \thickapprox #3} % equal types
\newcommand{\eqtypepath}[3]{#1 \vdash #2 \thicksim #3} % equal normal-form types
\newcommand{\eqtermalg}[4]{#1 \vdash #2 \thickapprox #3 \Leftarrow #4} % equal terms of normalized type
\newcommand{\eqtermext}[4]{#1 \vdash #2 \simeq #3 \Leftarrow #4} % equal terms w/o eta
\newcommand{\eqpath}[4]{#1 \vdash #2 \thicksim #3}  % equal paths

\newcommand{\equationin}[3]{\mathsf{equation}\; #1 \,{:}\, #2 \,{\equiv}\, #3 \; \mathsf{in} \;} % use equation hint
\newcommand{\rewritein}[3]{\mathsf{rewrite}\; #1 \,{:}\, #2 \,{\equiv}\, #3 \; \mathsf{in} \;} % use rewrite hint
\newcommand{\eqhint}[2]{(#1 \,{\equiv}\, #2)} % equality hint
\newcommand{\rwhint}[2]{(#1 \,{\leadsto}\, #2)} % rewrite hint

\renewcommand{\H}{\mathcal{H}}      % term hint list
\newcommand{\hintempty}{\circ}      % empty hint list

\newcommand{\addhinteq}[3]{#1, \eqhint{#2}{#3}} % extended hint list
\newcommand{\addhintrw}[3]{#1, \rwhint{#2}{#3}} % extended hint list
\newcommand{\ctxs}[2]{#1\,;\,#2}
\newcommand{\GH}{\ctxs{\G}{\H}}           % combined context and hints

% normalization

\newcommand{\tywhnf}[4]{#1 \vdash #2 \leadsto #3 / #4} % "one-step type normalization"
\newcommand{\whnf}[4]{#1 \vdash #2 \leadsto #3 / #4} % "one-step term normalization"

\newcommand{\tywhnfs}[3]{#1 \vdash #2 \leadsto^* #3 \not\leadsto } % "type normalization"
\newcommand{\whnfs}[3]{#1 \vdash #2 \leadsto^* #3 \not\leadsto } % "term normalization"

\newcommand{\inferred}[1]{{\color{magenta}{#1}}}

\newcommand{\simplefib}{\mathop{\mathsf{isfib}}\,}
\newcommand{\true}{\mathsf{true}}
\newcommand{\false}{\mathsf{false}}

\newcommand{\nameof}{{\mathsf{name\_of}}\,}
\newcommand{\typeof}[2]{{\mathsf{type\_of}}(#1,#2)}



\usepackage{pgfpages}
%\setbeameroption{hide notes}
\setbeameroption{show notes on second screen=right}
\setbeamertemplate{note page}{\pagecolor{yellow!10}\insertnote}

\newcommand{\nat}{\mathtt{nat}}
\newcommand{\bool}{\mathtt{bool}}
\newcommand{\UU}{\mathsf{U}}

\renewcommand{\istype}[1]{#1\ \mathsf{type}}
\renewcommand{\isfib}[1]{#1\ \mathsf{fib}}

\newcommand{\refl}[1]{\mathsf{refl}\,#1}
\newcommand{\idpath}[1]{\mathsf{idpath}\,#1}

\newcommand{\Id}[3]{\mathsf{Id}_{#1}(#2, #3)}
\newcommand{\Paths}[3]{\mathsf{Paths}_{#1}(#2, #3)}

\renewcommand{\rulename}[1]{\text{\textsc{\footnotesize #1}}}
\renewcommand{\eqhint}[3]{(#1 \,{\equiv}_{#3}\, #2)} % equality hint
\renewcommand{\rwhint}[3]{(#1 \,{\leadsto}_{#3}\, #2)} % rewrite hint
\renewcommand{\addhinteq}[4]{#1, \eqhint{#2}{#3}{#4}} % extended hint list
\renewcommand{\addhintrw}[4]{#1, \rwhint{#2}{#3}{#4}} % extended hint list
\renewcommand{\equationin}[4]{\mathsf{equation}\; #1 \,{:}\, #2 \,{\equiv}_{#4}\, #3 \; \mathsf{in} \;} % use equation hint
\renewcommand{\rewritein}[4]{\mathsf{rewrite}\; #1 \,{:}\, #2 \,{\equiv}_{#4}\, #3 \; \mathsf{in} \;} % 

\newcommand{\J}[5]{\mathsf{J}_{#1}(#2, #3, #4, #5)}
\newcommand{\transport}[6]{\mathsf{transport}_{#1}^{#2}(#3,#4,#5,#6)}
\usepackage{palatino}
% \usepackage{xypic}
\usepackage{amsfonts}
%\usepackage[llbracket,rrbracket]{stmaryrd}
\usepackage{graphicx}

\usepackage{tikz}
\usetikzlibrary{decorations}

\usepackage{color}
\usepackage{booktabs}
\usepackage{mathpartir}

%% Universe indices

\newcommand{\NN}{\mathbb{N}} % God-given numbers

% Generic entities
\newcommand{\G}{\Gamma} % a context
\newcommand{\D}{\Delta} % another context
\newcommand{\T}{T} % a type
\newcommand{\U}{U} % another type
\newcommand{\x}{x} % a term variable
\newcommand{\y}{y} % another term variable
\newcommand{\z}{z} % a third term variable
\newcommand{\e}{e} % a term
\newcommand{\cmd}{C} % a command
\newcommand{\expr}{E} % an expressions

\newcommand{\rulename}[1]{\text{\textsc{#1}}}

%% Syntax
\newcommand{\bnf}{\ \mathrel{{:}{:}{=}}\ }
\newcommand{\bnfor}{\ \mid\ \ }

%% Syntactic constructs
\newcommand{\ctxempty}{\bullet} % empty context
\newcommand{\ctxextend}[3]{#1,\, #2\, {:}\, #3} % extended context

\newcommand{\subst}[3]{#1[#3/#2]} % substitution
\newcommand{\substs}[2]{#1[#2]} % substitution of many variables

\newcommand{\Type}{\mathsf{Type}} % The type of all types
\newcommand{\Prod}[2]{\mathop{\textstyle\prod_{(#1 {:} #2)}}} % dependent product

\newcommand{\lam}[3]{\lambda #1 {:} #2.{#3}\,.\,} % $\lambda$-abstraction
\newcommand{\app}[5]{#1\mathbin{@^{#2{:}#3.#4}} #5} % application

\newcommand{\abst}[2]{[#1 \,.\, #2]} % abstraction
\newcommand{\ascribe}[2]{#1 \,{:}{:}\, #2} % type ascription

\newcommand{\JuEqual}[3]{\mathsf{Eq}_{#1}(#2,#3)} % Equality type
\newcommand{\juRefl}[1]{{\mathsf{refl}_{#1}}\ }    % Judgmental refl

% orthodox judgments
\newcommand{\isctx}[1]{#1\ \mathsf{ctx}} % well formed context
\newcommand{\istype}[2]{\isterm{#1}{#2}{\Type}} % well formed type
\newcommand{\isterm}[3]{#1 \vdash\,#2\,:\,#3} % well formed term

\newcommand{\eqtype}[3]{\eqterm{#1}{#2}{#3}{\Type}} % equal types
\newcommand{\eqterm}[4]{#1 \vdash #2 \equiv #3 : #4} % equal terms

% commands

\newcommand{\cmdLam}[2]{\lambda #1 {:} #2\,.\,} % $\lambda$-abstraction untagged return type
\newcommand{\cmdLamCurry}[1]{\lambda #1 \,.\,} % $\lambda$-abstraction completely untagged
\newcommand{\cmdApp}[2]{#1\,#2} % application untagged
\newcommand{\cmdType}{\mathsf{Type}} % The type of all types
\newcommand{\cmdProd}[2]{\mathop{\textstyle\prod_{(#1 {:} #2)}}} % dependent product
\newcommand{\cmdAscribe}[2]{#1 \,{:}{:}\, #2} % type ascription
\newcommand{\cmdLet}[2]{\mathsf{let}\;#1 \mathbin{{:}{=}} #2\; \mathsf{in}\;} % let-binding
\newcommand{\cmdEq}[2]{\mathsf{Eq}(#1,#2)} % Equality type
\newcommand{\cmdRefl}{\mathsf{refl}\;}    % Judgmental refl
\newcommand{\cmdReturn}{\mathsf{return}\;}    % Judgmental refl


% algorithmic judgments
\newcommand{\env}{\eta}
\newcommand{\envextend}[4]{#1(#2) \mathbin{{:}{=}} (#3, #4)} % extended context
\newcommand{\ctxenv}{\G;\, \env} % A combined context and environment

\newcommand{\evalc}[4]{#1 \vdash #2 \Leftarrow #3 \mapsto #4} % checking
\newcommand{\evali}[4]{#1 \vdash #2 \Rightarrow (#3, #4)} % inference

\newcommand{\ishints}[2]{#1 \vdash #2 \; \mathsf{hints}} % well-formed context and hints

\newcommand{\istypealg}[2]{#1 \vdash #2 \Leftarrow \mathsf{type}} % well formed type
\newcommand{\isfibalg}[2]{#1 \vdash #2 \Leftarrow \mathsf{fibered}} % is a fibered type
\newcommand{\eqtypealg}[3]{#1 \vdash #2 \thickapprox #3} % equal types
\newcommand{\eqtypepath}[3]{#1 \vdash #2 \thicksim #3} % equal normal-form types
\newcommand{\eqtermalg}[4]{#1 \vdash #2 \thickapprox #3 \Leftarrow #4} % equal terms of normalized type
\newcommand{\eqtermext}[4]{#1 \vdash #2 \simeq #3 \Leftarrow #4} % equal terms w/o eta
\newcommand{\eqpath}[4]{#1 \vdash #2 \thicksim #3}  % equal paths

\newcommand{\equationin}[3]{\mathsf{equation}\; #1 \,{:}\, #2 \,{\equiv}\, #3 \; \mathsf{in} \;} % use equation hint
\newcommand{\rewritein}[3]{\mathsf{rewrite}\; #1 \,{:}\, #2 \,{\equiv}\, #3 \; \mathsf{in} \;} % use rewrite hint
\newcommand{\eqhint}[2]{(#1 \,{\equiv}\, #2)} % equality hint
\newcommand{\rwhint}[2]{(#1 \,{\leadsto}\, #2)} % rewrite hint

\renewcommand{\H}{\mathcal{H}}      % term hint list
\newcommand{\hintempty}{\circ}      % empty hint list

\newcommand{\addhinteq}[3]{#1, \eqhint{#2}{#3}} % extended hint list
\newcommand{\addhintrw}[3]{#1, \rwhint{#2}{#3}} % extended hint list
\newcommand{\ctxs}[2]{#1\,;\,#2}
\newcommand{\GH}{\ctxs{\G}{\H}}           % combined context and hints

% normalization

\newcommand{\tywhnf}[4]{#1 \vdash #2 \leadsto #3 / #4} % "one-step type normalization"
\newcommand{\whnf}[4]{#1 \vdash #2 \leadsto #3 / #4} % "one-step term normalization"

\newcommand{\tywhnfs}[3]{#1 \vdash #2 \leadsto^* #3 \not\leadsto } % "type normalization"
\newcommand{\whnfs}[3]{#1 \vdash #2 \leadsto^* #3 \not\leadsto } % "term normalization"

\newcommand{\inferred}[1]{{\color{magenta}{#1}}}

\newcommand{\simplefib}{\mathop{\mathsf{isfib}}\,}
\newcommand{\true}{\mathsf{true}}
\newcommand{\false}{\mathsf{false}}

\newcommand{\nameof}{{\mathsf{name\_of}}\,}
\newcommand{\typeof}[2]{{\mathsf{type\_of}}(#1,#2)}

\newcommand{\tynorm}[2]{#1 \hookrightarrow #2} % (strong) normalization of terms
\newcommand{\norm}[2]{#1 \hookrightarrow #2} % (strong) normalization of terms


\title{Brazilian type checking}
\author{
\parbox{0.42\textwidth}{Andrej Bauer \\[4pt] \small University of Ljubljana}
\and
\parbox{0.42\textwidth}{Christopher Stone \\[4pt] \small Harvey Mudd College}
}

\date{\small
Formalization of mathematics in proof assistants\\
Institut Henri Poincare\\
May 2014}

% Comment out the following line to avoid overlayed items
% in itemize.
% \beamerdefaultoverlayspecification{<+->}

\mode<presentation>
% \usetheme{Goettingen}
\usecolortheme{rose}
\usefonttheme{serif}
\setbeamertemplate{navigation symbols}{}

% Frame number
\setbeamertemplate{footline}[frame number]{}

\begin{document}

\begin{frame}
  \titlepage

  \note[item]{Thanks to organizers for the invitation. This is promising
    to be a very exciting meeting.}

  \note[item]{The work I will present is being done jointly with Chris Stone,
    who is sitting somewhere in the audience. Hi Chris!}
\end{frame}


\begin{frame}
  \LARGE
  \begin{enumerate}
  \item Brazil
  \item TT
  \end{enumerate}

  \note[item]{My talk is about an ongoing experiment in the \emph{design} of a proof
    checker, called ``Brazil'' for reasons that will become apparent.}

  \note[item]{We are also working on an extension of Brazil, called ``tt''. This is a more
    ambitious piece of work. You can think of it as a special-purpose programming language
    for implementation of unification algorithms, tactics, search procedures, etc.
    Alternatively, you can think of it as a trusted kernel on top of which we can build
    ever more complex tools for formalization.}

  \note[item]{As tt is more experimental than Brazil, I will focus on Brazil and discuss
    tt only briefly at the end. But please talk to Chris and me, and we'll show you both
    prototypes in action. Even better, you can participate on GitHub.}

\end{frame}

\begin{frame}
  \LARGE
  \begin{enumerate}
  \item Brazil
  \item<0> TT
  \end{enumerate}

  \note[item]{Brazil is based on a type theory proposed by VV, so in this sense it is
    closely related to HoTT. However, I think some ideas are relevant also to other kinds
    of foundation. VV's type theory is quite general and is likely able to accommodate
    special cases, such as classical mathematics.}

  \note[item]{Various aspects of Brazil in TT, in particular the programming language
    ideas, are independent of the foundation altogether. So you may be able to take some
    ideas from us and use them for your work. We'd be delighted.}
\end{frame}

\begin{frame}
  
  Strict equality:
  %
  \begin{equation*}
    \infer
    {\istype{A} \\
      a : A \\
      b : A}
    {\istype{\Id{A}{a}{b}}}
    \qquad
    \infer
    {a : A}
    {\refl{a} : \Id{A}{a}{a}}
  \end{equation*}
  %
  Paths:
  \begin{equation*}
    \infer
    {\istype{A} \\
      a : A \\
      b : A}
    {\istype{\Paths{A}{a}{b}}}
    \qquad
    \infer
    {a : A}
    {\idpath{a} : \Paths{A}{a}{a}}
  \end{equation*}

  \note[item]{
    To motivate the setup, recall that there are \emph{two} ways to introduce
    equality in type theory. They go by many different names. Let me call them \emph{strict    
    equality} and \emph{paths}, because I want to use your topological intuitions.
  }

  \note[item]{If you are familiar with type theory you will recognize these by their
    elimination rules on the next slide. If not, think of the strict equality as the usual
    equality, presented as a set. If $a$ and $b$ are equal then $\Id{A}{a}{b}$ contains the element $\refl{a}$, and nothing else. If not, then $\Id{A}{a}{b}$ is empty.}

  \note[item]{$\mathsf{Paths}$ is the type of \emph{paths} between $a$ and $b$ in a space
    $A$. Under this view, all types are ``some sort of spaces''. Here $\idpath{a}$ is
    the constant path at point $a$.
  }

  \note[item]{I am not going to give you pages upon pages of inference rules, you can look
    them up on our GitHub project.}
\end{frame}

\begin{frame}
  
  Equality elimination:
  %
  \begin{equation*}
    \infer
    {x : A \vdash \istype{P(x)} \\
     a : A \\ b : A \\
     e : \Id{A}{a}{b} \\
     u : P(a)
    }
    {u : P(b)}
  \end{equation*}
  %
  Path elimination:
  %
  \begin{equation*}
    \infer
    {x : A \vdash \istype{P(x)} \\
     a : A \\ b : A \\
     p : \Paths{A}{a}{b} \\
     u : P(a)
    }
    {\transport{A}{x.P}{a}{b}{p}{u} : P(b)}
  \end{equation*}

  \note[item]{Here are the \emph{non-dependent} versions of elimination rules. The
    dependent ones are more complex, and need not be considered here.}

  \note[item]{The strict version is just replacement of equals for equals. Since $a$ and
    $b$ are equal, we may use $a$ in place of $b$.}

  \note[item]{The path version can be explained as follows: $a$ and $b$ may not be equal, but
    there is a path $p$ between them. We can \emph{transport} $u$ along $p$ from the fiber
    $P(a)$ to the fiber $P(b)$. We think of $P$ as a topological fibration, i.e., a family
    of spaces indexed by $A$, with a path-lifting property. This is at the core of the
    HoTT.}

  \note[item]{The homotopical way of doing things has many benefits, among others the
    Univalence axiom. But it can also lead to complications.}
\end{frame}


\begin{frame}
  
  Equality elimination:
  %
  \begin{equation*}
    \infer
    {x : A \vdash \istype{P(x)} \\
     a : A \\ b : A \\
     e : \Id{A}{a}{b} \\
     u : P(a)
    }
    {u : P(b)}
  \end{equation*}
  %
  Path elimination:
  %
  \begin{equation*}
    \infer
    {x : A \vdash \istype{P(x)} \\
     a : A \\ b : A \\
     p : \Paths{A}{a}{b} \\
     u : P(a)
    }
    {\transport{A}{x.P}{a}{b}{p}{u} : P(b)}
  \end{equation*}
  
  \note[item]{One such complication is the construction of a model of HoTT inside HoTT.
    (Note that there is no problem with Gödelian phenomena because the inner model has
    fewer universe.)}

  \note[item]{We would like to perform the construction of semisimplicial types to create
    a model. However, even though the construction seems intuitively clear, using paths
    and transport makes everything horribly complicated because the type theory makes us
    write down all the transports, even though they are all trivial.}

  \note[item]{VV therefore proposed to have \emph{both} equality types at the same time.
    Then we can use the strict equality where appropriate, to keep the construction
    manageable.}
\end{frame}

\begin{frame}

  Fibered types:
  %
  \begin{equation*}
    \istype{A}
    \qquad\quad
    \isfib{A}
    \qquad\quad
    \infer{\isfib{A}}{\istype{A}}
  \end{equation*}
  %
  Equality:
  %
  \begin{gather*}
    \infer
    {\istype{A} \\
      a : A \\
      b : A}
    {\istype{\Id{A}{a}{b}}}
    %
    \\[0.75\baselineskip]
    %
    \infer
    {\isfib{A} \\
      a : A \\
      b : A}
    {\isfib{\Paths{A}{a}{b}}}
  \end{gather*}

  \note[item]{We cannot just throw in both equalities -- it would turn out that they
    coincide, and that would then contradict Univalence.}

  \note[item]{Instead, we should distinguish some of the types as \emph{fibered}. These
    behave nicely with respect to paths: they have the path lifting property necessary to
    interpret transport.}
 
  \note[item]{Paths can only be formed and used on fibered types. Since strict equality is
    not fibered, we cannot show that path equality implies strict equality.}

  \note[item]{There are other complications. Let me show how crazy things get.}
\end{frame}

\begin{frame}
  \begin{multline*}
    e : \Id{\UU}{\nat \to \bool}{\nat \to \nat} \\
    \vdash
    (\lambda n : \nat \,.\, n + 3) \, \mathsf{42} : \bool
  \end{multline*}


  \note[item]{
    We may \emph{assume} that the Baire space and the Cantor space
    are equal. Therefore $\lambda n : \nat \,.\, n + 3$ is map from
    $\nat$ to $\bool$, so the application is a $\bool$.
  }

  \note[item]{
    But we \emph{cannot} $\beta$-reduce because on a careful reading
    of the $\beta$-rule we discover that it does not apply in this case.
    This destroys any hope for having normal forms.
  }

  \note[item]{
    This and similar examples force us to equip each application
    with explicit typing information. Luckily, all such annotations
    can be derived because they are unique, if they exist. Nevertheless,
    we need them.
  }

  \note[item]{
    By the way, the assumption that the Baire space and Cantor space
    are equal is \emph{not} inconsistent. As homework, you should find
    \emph{two} models which validate the assumption.
  }
\end{frame}


\begin{frame}
  
  \begin{multline*}
    e : \Id{\UU}{D}{D \to D} \\
    \vdash
    (\lambda x : D \,.\, x \, x) (\lambda x : D \,.\, x \, x) : D
  \end{multline*}


  \note[item]{
    Worse, we may assume that a type is equal to its own function space.
  }

  \note[item]{
    The ``paradoxical'' $\lambda$-term which does not normalize becomes
    well-typed.
  }

  \note[item]{
    Therefore, it is going to be easy to get the proof checker into an
    infinite loop, as long as any sort of normalization is built in.
  }

  \note[item]{
    Is this a problem? Maybe theoretically, but certainly not in practice.
    The typical user is going to hit Ctrl-C well before the Sun becomes a
    red giant.
  }
\end{frame}

\begin{frame}

  \begin{equation*}
    \infer
    {x : A \vdash \istype{P(x)} \\
     a : A \\ b : A \\
     e : \Id{A}{a}{b} \\
     u : P(a)
    }
    {u : P(b)}
  \end{equation*}
  
  \note[item]{
    We would like to send formalized mathematics to people
    in a faraway country, such as Brazil. They will check our development
    independently. [VV tried Bolivia but that didn't work.]
  }

  \note[item]{
    But how will they deal with strict equality elimination? If we send them
    the proof object $u$ and $P(b)$, how can they ever guess what $a$ and $e$ are?
  }

  \note[item]{
    In fact, type checking in the presence of strict equality is undecidable.
    Without some divine help, Brazilians are doomed.
  }

  \note[item]{
    By the way, with path equality there is no problem because transport records
    $a$ and the path from $a$ to $b$.
  }

  \note[item]{
    Thus, we must also send to Brazil \emph{additional hints} about equality, which
    however are not part of the proof object proper. These are \emph{extra}.
  }

\end{frame}

\begin{frame}
  
  \begin{enumerate}
  \item Type theory with strict equality \emph{and} paths.
  \item A trusted proof checker.
  \item \emph{Practical} support for equality hints.
  \item Sacrifice termination and completeness.
  \item Preserve \emph{soundness} guarantee.
  \end{enumerate}


  \note[item]{
    Let us summarize what we want.
  }

  \note[item]{
    By ``practical'' we mean that hints should be used infrequently
    and need not be very large.
  }

  \note[item]{
    The soundness guarantee is: \emph{if} the checker accepts 
    a proof object (with hints), then there is a derivation that
    the proof object has the given type.
  }

  \note[item]{
    By now it is obvious that our proof checker is called ``Brazil''.
  }
\end{frame}

\begin{frame}
  
  Contexts:
  % 
  \begin{equation*}
    \G
    \begin{aligned}[t]
      \bnf   {}& \ctxempty & & \text{empty context}\\
      \bnfor {}& \ctxextend{\G}{\x}{\T} & & \text{context extended with $x : T$}
    \end{aligned}
  \end{equation*}
  % 
  Equality hints:
  % 
  \begin{equation*}
    \H
    \begin{aligned}[t]
      \bnf   {}& \hintempty & & \text{empty hints}\\
      \bnfor {}& \addhinteq{\H}{\e_1}{\e_2}{\T} & & \text{extend hints with an equation} \\
      \bnfor {}& \addhintrw{\H}{\e_1}{\e_2}{\T} & & \text{extend hints with a rewrite}
    \end{aligned}
  \end{equation*}

  \note[item]{The typing contexts are as usual.}

  \note[item]{We also carry around a list of equality hints. These are of two
    kinds: ordinary equations and rewrite rules (directed equations) to be used
    during weak head normalization.}

\end{frame}

\begin{frame}

  Types:
  % 
  \begin{equation*}
    \T, \U
    \begin{aligned}[t]
      \bnf   {}& \Universe{\alpha} & & \text{universe}\\
      \bnfor {}& \El{\inferred\alpha}{\e} & & \text{type named by $e$}\\
      \bnfor {}& \Unit & & \text{the unit type}\\
      \bnfor {}& \Prod{x}{\T} \U & & \text{product}\\
      \bnfor {}& \PrEqual{\inferred{T}}{\e_1}{\e_2} & & \text{path type}\\
      \bnfor {}& \JuEqual{\inferred{T}}{\e_1}{\e_2} & & \text{equality type}
    \end{aligned}
  \end{equation*}

  \note[item]{We use Tarski-style universes. With Russell-style universes and
    cummulativity it is not clear at all how to perform equality checks (at what
    universe -- it matters!). I shall not say more about this, except that contrary
    to folk opinion, Tarski universes are \emph{not} annoying to implement. And
    they are completely hidden from the user.
  }

  \note[item]{The magenta bits are type annotations. They must be present, but luckily
    it is easier to derive them than to require that they be provided. So, the magenta
    bits are not there in the concrete syntax. They get added during type checking
    and synthesis, we will see an example.}
  
\end{frame}

\small

\begin{frame}
  Terms:
%
\begin{equation*}
  \e
  \begin{aligned}[t]
    \bnf   {}&  \x   &&\text{variable} \\
    \bnfor {}&  \equationin{\e_1}{\inferred{\e_2}}{\inferred{\e_3}}{\inferred{T}} e_4 &&\text{use equality hint $\e_1$ in $\e_4$} \\
    \bnfor {}&  \rewritein{\e_1}{\inferred{\e_2}}{\inferred{\e_3}}{\inferred{T}} e_4 &&\text{use rewrite hint $\e_1$ in $\e_4$} \\
    \bnfor {}&  \ascribe{\e}{\T}  &&\text{ascribe type $\T$ to term $\e$} \\
    \bnfor {}&  \lam{\x}{\T_1}{\inferred{\T_2}} \e  &&\text{$\lambda$-abstraction} \\
    \bnfor {}&  \app{\e_1}{\inferred\x}{\inferred{\T_1}}{\inferred{\T_2}}{\e_2}  &&\text{application} \\
    \bnfor {}&  \unitTerm  &&\text{the element of unit type} \\
    \bnfor {}&  \prRefl{\inferred\T}{\e}  &&\text{identity path} \\
    \bnfor {}&  \PrElim{\inferred\T}{\abst{x\,y\,p}{\U}}{\abst{z}{\e_1}}{\e_2}{\inferred{\e_3}}{\inferred{\e_4}}  &&\text{path eliminator} \\
    \bnfor {}&  \juRefl{\inferred\T} \e  &&\text{reflexivity} \\
           {}&  \vdots  &&\text{\vdots}
  \end{aligned}
\end{equation*}

 \note[item]{Here is the abstract syntax of terms, without universe names}

 \note[item]{Again, type annotations in magenta are derived automatically.
   I emphasize that this is \emph{simpler} than having them provided and having
   to check that they are correct.}

 \note[item]{
   The ``equation'' and ``rewrite'' constructs are how hints are added to the
   system. This is a \emph{local} construct, so you can tell Brazilians to use a specific
   hint only in a specific term.
 }

 \note[item]{
   When terms are compared for syntactic equality, the hints constructs are ignored.
   The hints are \emph{not} part of the proof object proper.
 }

\end{frame}

\begin{frame}

  \emph{Synthesize} the type of $e$:
  %
  \begin{equation*}
    \synterm{\Gamma}{e}{T}
  \end{equation*}
  %
  \emph{Check} that $e$ has the given type $T$:
  %
  \begin{equation*}
    \chkterm{\Gamma}{e}{T}
  \end{equation*}
  %
  Passage from synthesis to checking:
  %
  \begin{equation*}
    \infer{\synterm{\Gamma}{e}{T}
      \\ U \equiv T}
    {\chkterm{\Gamma}{e}{U}}
  \end{equation*}
  
  \note[item]{
    I cannot here explain all the technical details of what we did,
    but here is a broad outline of how we formulated VV's type theory
    in an algorithmic way.
  }

  \note[item]{
    For the experts: we use standard synthesis/checking distinction,
    Harper-Stone style equality checking, and we consult an equality hints
    database to overcome limitations of the equality checking algorithm.
  }
  
  \note[item]{
    Let me briefly comment on what that means. We split the typing judgment
    into two -- synthesis and checking. Whether we should use synthesis or checking
    depends on the shape of the term $e$.}

  \note[item]{
    This way the type checking algorithm can be organized in such a way that
    type equality is invoked in precisely one rule. We thus have good control
    over where equality checking happens.
  }
\end{frame}

\begin{frame}

  Application (traditional):
  %
  \begin{equation*}
    \infer[\rulename{term-app}]
      {\isterm{\G}{\e_1}{\Prod{x}{\T} \U} \\
      \isterm{\G}{\e_2}{\T}
    }
    {\isterm{\G}{\app{\e_1}{\x}{\T}{\U}{\e_2}}{\subst{\U}{\x}{\e_2}}}
  \end{equation*}

  Application (synthesis \& checking):
  %
  \begin{equation*}
    \infer[\rulename{syn-app}]
    { \synterm{\GH}{\e_1}{\T_1} \\
      \whnfs{\GH}{\T_1}{\Prod{\x}{\T} \U}\\
      \chkterm{\GH}{\e_2}{\T}
    }
    { \synterm{\GH}{\app{\e_1}{\inferred{\x}}{\inferred\T}{\inferred\U}{\e_2}}{\subst{\U}{\x}{\e_2}} }
  \end{equation*}


  \note[item]{
    For example, consider the traditional application rule and how it is done
    by synthesis and checking.
  }

  \note[item]{
    Application is a synthesizing form. Notice that we need to weak head
    normalize in order to verify that $e_1$ has the correct type. Also,
    the magenta annotations are read off the synthesized parts. This is the
    case with all magenta annotations.}

\end{frame}


\begin{frame}

  Install a new hint:
  %
  \begin{equation*}
    \infer[\rulename{chk-equation-hint}]
    { \synterm{\GH}{\e_1}{\U'} \\
      \whnfs{\GH}{\U'}{\JuEqual{\U}{\e_2}{\e_3}} \\\\
      \chkterm{\ctxs{\G}{(\addhinteq{\H}{\e_2}{\e_3}{\U})}}{\e_4}{\T}
    }
    { \chkterm{\GH}{(\equationin{\e_1}{\inferred{\e_2}}{\inferred{\e_3}}{\inferred{\U}} \e_4)}{\T} }
  \end{equation*}

  Using a hint:
  %
  \begin{equation*}
    \infer[\rulename{chk-eq-hint}]
    {
      \eqhint{\e_1}{\e_2}{\T} \in \H
    }
    { \eqtermalg{\GH}{\e_1}{\e_2}{\T} }
  \end{equation*}

  \note[item]{Next, equality checking is done by an adaptation of an algorithm developed
    by Bob Harper and Chris Stone. The algorithm has function extensionality and various
    $\eta$-rules built in.
  }

  \note[item]{Our version consults the hints. For example, to check that $e_1$ and $e_2$
    are equal at type $T$, we first check whether they are syntactically equal, then we
    consult the hints, then we proceed with the Harper-Stone algorithm (which calls itself
    recursively at a smaller type).
  }

  \note[item]{At a base type the algorithm uses weak-head normalization to compare terms.
    There we use rewrite hints.}
\end{frame}

\begin{frame}[fragile]

Symmetry of equality:
\begin{verbatim}
Definition sym :=
fun (A : Type) (a b : A) (p : a == b) => 
  equation p in (refl a :: b == a).
\end{verbatim}

Strict equality eliminator:
{\small
\begin{verbatim}
Definition G :=
fun (A : Universe 0)
  (P : forall (x y : A), x == y -> Universe 0)
  (r : forall z : A, P z z (refl z))
  (x y : A) (p : x == y)
  =>
    equation p in (r x :: P x y p).
\end{verbatim}
}
\end{frame}

\begin{frame}[fragile]

\small 
\begin{verbatim}
Definition Type := Universe f0.

Parameter bool : Type.
Parameter true false : bool.

Parameter bool_ind : 
  forall (P : bool -> Type) (b : bool),
    P true -> P false -> P b.

Parameter bool_ind_true :
  forall (P : bool -> Type)
         (x : P true) (y : P false),
    bool_ind P true x y == x.

Rewrite bool_ind_true.
\end{verbatim}

\end{frame}

\begin{frame}
  \LARGE
  \begin{enumerate}
  \item<0> Brazil
  \item TT
  \end{enumerate}

  \note[item]{In conclusion let me say a few words about TT.}

  \note[item]{
    We heard yesterday that there is a distinction between
    \emph{functional} or \emph{declarative} style of programming,
    and \emph{procedural} programming.
  }

  \note[item]{
    In the theory of programming languages procedural programming
    is just one case of a more general concept known as \emph{computational effect}.
    All sort of things are computational effects: state, I/O, non-determinism, exceptions, etc.
    The typical operations of a proof assistant may be viewed as computational
    effects: unification, context manipulation, backtracking, etc.
  }

  \note[item]{
    In fact, Brazilian equality hints are a very simple form of computatinal effect, too.
  }
  
\end{frame}

\begin{frame}
  \LARGE
  \begin{enumerate}
  \item<0> Brazil
  \item TT
  \end{enumerate}
  

  \note[item]{
    We are developing a language based on a general theory of so-called algebraic effects
    and handlers that is tailored to the needs of a proof assistant. The handlers are a
    powerful programming concepts, a generalization of exception handlers and delimited
    continuations. They allow us to write purely functional programs in a procedural style
    (and more).
  }

  \note[item]{
    For example, it will be possible in TT to implement meta-variables and
    unification as a \emph{derived} concept. At the same time, tt will have a
    soundness guarantee: it will refuse to generate invalid Brazilian terms.
  }

  \note[item]{
    But it is too early to tell how successful the experiment will be.
    Ask us again on Friday, Chris will be done by then.
  }
\end{frame}


\end{document}